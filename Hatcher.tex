\documentclass[11pt]{article}

\usepackage[margin=1in]{geometry}
\usepackage{amsmath,amsthm,amsfonts}
\usepackage[utf8]{inputenc}
\usepackage{amssymb}
\usepackage[mathscr]{eucal}
\usepackage{graphicx}
\usepackage{listings}
\usepackage{titlesec}
\usepackage{tikz-cd}
\usepackage[shortlabels]{enumitem}
\usepackage{tikz}
\usepackage{xcolor}
\usepackage[OT1]{fontenc}
\usepackage{physics}
\usepackage{xpatch}
\usepackage{hyperref}

\setlength\parindent{0pt}
\newcommand{\forceindent}{\leavevmode{\parindent=1.5em\indent}}

\theoremstyle{definition}
\newtheorem*{defin}{Definition}
\newtheorem*{example}{Example}

\theoremstyle{plain}
\newtheorem*{corollary}{Corollary}
\newtheorem{proposition}{Proposition}[section]
\newtheorem*{lemma}{Lemma}
\newtheorem{theorem}{Theorem}[section]

\theoremstyle{remark}
\newtheorem*{remark}{Remark}

\renewcommand{\thesection}{\Roman{section}}

\setcounter{section}{-0}

\newenvironment{rcases}
  {\left.\begin{aligned}}
  {\end{aligned}\right\rbrace}

\newcommand{\id}{\textup{id}}
\newcommand{\im}{\textup{Im}\,}
\newcommand{\R}{\mathbf{R}}
\newcommand{\Z}{\mathbf{Z}}
\newcommand{\Q}{\mathbf{Q}}
\newcommand{\N}{\mathbf{N}}
\newcommand{\C}{\mathbf{C}}
\newcommand{\cone}{\textup{cone}}
\newcommand{\D}{\mathbf{D}}

\begin{document}

\textbf{\Large{Content}}\bigbreak

\hyperref[1]{\textbf{I The Fundamental Group}}\newline
\forceindent\hyperref[2]{I.1 Basic constructions}\quad\hyperref[3]{I.2 Fundamental groups}\quad\hyperref[4]{I.3 $\pi_1\mathbf{S}^1\cong\Z$ and its consequences}\newline
\forceindent\hyperref[6]{I.4 Induced homomorphisms}\quad\hyperref[7]{I.5 Van Kampen's theorem}\quad\hyperref[8]{I.6 Covering spaces}\newline
\forceindent\hyperref[9]{I.7 Transformations and actions}\medbreak

\hyperref[10]{\textbf{II Homology}}\newline
\forceindent\hyperref[11]{II.1 Simplicial homology and Singular homology}\quad\hyperref[12]{II. 2 Zig-zag lemma and excision}

\newpage
\section{The Fundamental Group}\label{1}

\subsection{Basic constructions}\label{2}

Convention: all spaces are topological, and all maps are continuous.\medbreak

A \textbf{homotopy} is a family of maps $\{f_t:X\to Y\}_{0\leq t\leq1}$ such that $F:X\times[0,1]\to Y$ by $(x,t)\mapsto f_t(x)$ is continuous. Maps $f_0$ and $f_1$ are \textbf{homotopic} relative to $A\subseteq X$, denoted by $f_0\simeq f_1$, if the homotopy $f_t|_A$ is independent of $t$.\medbreak

A \textbf{retraction} is a map $r:X\to X$ such that $r^2=r$. A \textbf{deformation retraction} is a homotopy from the identity map to the retraction, i.e., $\{f_t:X\to X\}$ such that $f_0=\id_X$, $f_1=r$, and $f_t|_{r(X)}=\id_{r(X)}$. $r(X)$ is called the \textbf{deformation retract} of $X$.\medbreak

Every continuous map $f$ factors as an inclusion to the \textbf{mapping cylinder} $M_f$ followed by a retraction $\pi:M_f\to Y$.
\[\begin{tikzcd}
X\arrow[hookrightarrow]{dr}{i}\arrow{rr}{f}&&Y\\
&M_f\arrow[swap]{ur}{\pi}
\end{tikzcd}\]\medbreak

A \textbf{homotopy equivalence} is a bijective map $f:X\to Y$ with inverse $g$ such that $fg\simeq\id_Y$ and $gf\simeq\id_X$. If such $f$ exists, then $X$ and $Y$ are \textbf{homotopic}, denoted by $X\simeq Y$. $X$ is \textbf{contractible} if it is homotopic to a point.\medbreak

An \textbf{$n$-cell} is a topological space homeomorphic to $\R^n$, and a \textbf{$n$-skeleton} is the union of cells of dimension less then or equal to $n$.\medbreak

A \textbf{cell complex} $X$ of dimension $k$ is built up inductively as follows:\begin{enumerate}
    \item Start with a $0$-skeleton consisting of $0$-cells.
    \item Given $(k-1)$-skeleton $X^{k-1}$, attach $k$-cells $e^k_\alpha$ to $X^{k-1}$ through $\phi_\alpha:\partial\D^k\to X^{k-1}$. Define $k$-skeleton $X^k=X^{k-1}\sqcup\{e^k_\alpha\}_\alpha$.
    \item Then assign $X=\cup_kX^k$ with weak topology, namely, $A\subseteq X$ is open if and only if $A\cap X^k$ is open for each $k$.
\end{enumerate}
\begin{example}
Identification space of $\mathbf{T}^2$: the point is a $0$-cell, and edges $a,b,c,d$ are $1$-cells encircling the middle $2$-cell.
\end{example} 

\subsection{Fundamental groups}\label{3}

A \textbf{path} on space $X$ is a map $f:[0,1]\to X$. Two paths $f$ and $g$ are \textbf{path homotopic}, denoted by $f\simeq_pg$, if they are homotopic relative to the endpoints.\medbreak

$f\phi$ is a \textbf{reparametrization} of $f$ if the map $\phi:[0,1]\to[0,1]$ satisfies $\phi(0)=0$ and $\phi(1)=1$. Then $f\simeq_p f\phi$ by $f_t(s)=f((1-t)s+t\phi(s))$. In fact, convex combinations make any two paths with same endpoints in a convex Euclidean space homotopic.\medbreak

A \textbf{constant path} is $f_{x_0}(s)=x_0$ for all $s\in[0,1]$. An \textbf{inverse path} $\overline{f}$ of $f$ is $\overline{f}(s)=f(1-s)$. The \textbf{concatenation} (not composition!) $f\cdot g$ of $f$ and $g$ is
\[f\cdot g(s)=\begin{cases}
f(2s),&s\in[0,1/2]\\g(2s-1),&s\in[1/2,1]
\end{cases}.\]\medbreak

Then Path homotopy is an equivalence relation. The set of path homotopy equivalence classes of paths with endpoints $x_0\in X$ forms a group with concatenation as operation, constant path as identity element, and inverse path as inverse. Such group is called the \textbf{fundamental group} of $X$ at basepoint $x_0$, denoted by $\pi_1(X,x_0)$.\medbreak

A path $h$ in $X$ going from $x_0$ to $x_1$ induces a group isomorphism $\beta_h:\pi_1(X,x_1)\to\pi_1(X,x_0)$ by $[f]\mapsto[h\cdot f\cdot\overline{h}]$. Hence in a path-connected space we write $\pi_1X$ instead of $\pi_1(X,x_0)$.
\medbreak

A space is \textbf{simply connected} if it is path-connected and has trivial fundamental group. A space is simply connected if any two paths with same endpoints are homotopic. Examples include convex Euclidean space.

\begin{theorem}
If $X$ and $Y$ are path-connected, then $\pi_1(X\times Y)=\pi_1(X)\times \pi_1(Y).$
\end{theorem}
\begin{proof}
Continuity is preserved in product topology.
\end{proof}

\subsection{$\pi_1\mathbf{S}^1\cong\Z$ and its consequences}\label{4}

Let's for now assume that $\pi_1\mathbf{S}^1\cong\Z$ and look at its consequences. We will prove it in \hyperref[8]{I.6 Covering spaces}.

\begin{corollary}[Fundamental Theorem of Algebra]
Every $p(z)\in\C[z]$ that is not a constant has a root in $\C$.
\end{corollary}

\begin{corollary}[Brouwer fixed-point, 2d]
Every continuous map $f:\D^2\to\D^2$ has a fixed point.
\end{corollary}

The next theorem tells us that there exists a pair of antipodal points on earth with exact same temperature and humidity and any instance:

\begin{corollary}[Borsuk-Ulam, 2d]
For every continuous map $f:\mathbf{S}^1\to\R^2$ there exists a pair of antipodal points $x,-x\in\mathbf{S}^1$ such that $f(x)=f(-x)$.
\end{corollary}

\subsection{Induced homomorphisms}\label{6}

A map $\phi:X\to Y$ induces a map $\phi_\ast:\pi_1(X,x_0)\to\pi_1(Y,\phi(x_0))$ by $[f]\mapsto[\phi\circ f]$. This map is well-defined, as if $f_t$ is a path homotopy connecting loops $f_0$ and $f_1$, then $\phi\circ f_t$ is a homotopy connecting $\phi\circ f_0$ and $\phi\circ f_1$. Furthermore, $\phi_\ast$ is a group homomorphism, as $\phi\circ([f]\cdot[g])=(\phi\circ[f])\cdot(\phi\circ[g])$ both maps $s$ to $\phi(f(2s))$ when $s\in[0,1/2]$ and $\phi(g(2s-1))$ when $s\in(1/2,1]$.\medbreak

Moreover, notice that $(\psi\circ\phi)_\ast=\psi_\ast\circ\phi_\ast$, and that $(\id_{(X,x_0)})_\ast=\id_{\pi_1(X,x_0)}$. Hence the fundamental group $\pi_1$ is a functor from the category of based topological spaces to the category of groups.

\begin{example}
$\pi\mathbf{S}^n=0$ for $n\geq2$.
\begin{proof}
Let $f:[0,1]\to\mathbf{S}^n$ be a based loop. If there exists a point $x\in\mathbf{S}^n\setminus f([0,1])$, then $\mathbf{S}^n\setminus\{x\}$ is contractible and $f$ is trivial. Since $f$ is chosen arbitrarily, this proves that $\pi\mathbf{S}^n=0$.\medbreak

Let $f$ be based at $x_0$. Choose $x_1$ and an $n$-dimensional ball $B_{x_1}$ centered at $x_1$ with $x_0\not\in B_{x_1}$. $f^{-1}(B_{x_1})$ is covered by a union (possibly infinite) of intervals, but since $f^{-1}(x_1)$ is compact, $f^{-1}(x_1)\subseteq(a_1,b_1)\cup\cdots\cup(a_n,b_n)$ where $f(a_i),f(b_i)\in\partial B_{x_1}$ for all $i$. Since $n\geq2$, $B_{x_1}\setminus\{x_1\}$ is path-connected. Hence any path $f_i$ in $B_{x_1}$ that goes from $f(a_i)$ to $f(b_i)$ is homotopic to $f'_i$ in $B_{x_1}\setminus\{x_1\}$ with the same endpoints. With this modification, $x_1\in\mathbf{S}^n\setminus f([0,1])$, and the argument follows.
\end{proof}
\end{example}


\begin{example}
$\R^2$ is not homeomorphic to $\R^n$ for $n\neq2$. (This is not trivial, since there exists continuous space-filling curves $\R^2\to\R^n$, $n>2$ that are surjective.)
\end{example}

\begin{proof}
Connectedness is preserved by homeomorphisms. Removing a point disconnects $\R$ but not $\R^2$. Hence $\R^2$ is not homeomorphic to $\R$.\medbreak

For $n>2$, if $\R^2$ is homeomorphic to $\R^n$, then $\pi_1\R^2\cong\pi_1\R^n$, then $\pi_1(\R^2\setminus\{0\})=\pi_1(\R^n\setminus\{0\})$. By polar decomposition $\R^n\setminus\{0\}$ is homeomorphic to $\mathbf{S}^{n-1}\times\R$. Hence $\pi_1(\R^n\setminus\{0\})$ is trivial. But $\pi_1(\R^2\setminus\{0\})=\Z$. Hence $\R^2$ is not homeomorphic to $\R^n$.
\end{proof}

In the proof above we deduce $\pi_1\R^2\cong\pi_1\R^n$ from $\R^2$ being homeomorphic to $\R^n$. In fact, the same can be deduced from the weaker condition $\R^2\simeq\R^n$:

\begin{theorem}
Let $X\simeq Y$ via homotopy $\phi_t:X\to Y$. Let $h:[0,1]\to Y$ by $t\mapsto\phi_t(x_0)$ be a path and $\beta_h:[f]\mapsto[h\cdot f\cdot\overline{h}]$ an isomorphism. Then the following diagram commutes:
\[\begin{tikzcd}
\pi_1(X,x_0)\arrow[swap]{dr}{(\phi_0)_\ast}\arrow{r}{(\phi_1)_\ast}&\pi_1(Y,\phi_1(x_0))\arrow{d}{\beta_h}\\
&\pi_1(Y,\phi_0(x_0)).
\end{tikzcd}\]
Hence a homotopic equivalence $X\stackrel{\text{$\phi$}}{\simeq}Y$ induces a group isomorphism $\pi_1(X,x_0)\stackrel{\text{$\phi_\ast$}}{\cong}\pi_1(Y,\phi(x_0))$.
\end{theorem}
\begin{proof}
We want to show that $(\phi_0)_\ast\circ f\simeq_p\beta_h\circ((\phi_1)_\ast\circ f)$ for $[f]\in\pi_1(X,x_0)$. Consider $h_s:[0,1]\to Y$ by $t\mapsto h(st)$, a reparametrization of $h|_{[0,s]}$. Then the homotopy $h_s\cdot((\phi_s)_\ast\circ f)\cdot\overline{h_s}$ does the job. Indeed, it gives $(\phi_0)_\ast\circ f$ when $s=0$ and $\beta_h\circ((\phi_1)_\ast\circ f)$ when $s=1$.\medbreak

It remains to check that $\phi_\ast$ is bijective. Since $\phi^{-1}\circ\phi=\id_X$, $(\phi^{-1}\circ\phi)_\ast=(\phi^{-1})_\ast\circ(\phi_\ast)=\id_{\pi_1(X,x_0)}$ is bijective. Hence $(\phi^{-1})_\ast$ is surjective and $\phi_\ast$ is injective. A mirror argument concludes the proof.
\end{proof}

\subsection{Van Kampen's theorem}\label{7}

Recall the property of free product of groups: for homomorphisms $\phi_1:G_1\to H$ and $\phi_2:G_2\to H$ there exists a unique extension $\Phi:G_1\ast G_2\to H$ by $g_1\cdots g_n\mapsto\phi_{l(1)}(g_1)\cdots\phi_{l(n)}(g_n)$ where $l(i)=1$ if $g_i\in G_1$ and $l(i)=2$ if $g_i\in G_2$.
\[\begin{tikzcd}
G_1\arrow[bend left=15]{drr}{\phi_1}\arrow[swap]{dr}{i_1}\\
&G_1\ast G_2\arrow{r}{\Phi}&H\\
G_2\arrow{ur}{i_2}\arrow[swap,bend right=15]{urr}{\phi_2}
\end{tikzcd}\]\medbreak

Van Kampen's theorem asserts that a pushout in the category of sets induces a \textbf{pushout} in the category of groups.
\[\begin{tikzcd}
A\cap B\arrow{r}{i_1}\arrow[swap]{d}{i_2}&A\arrow{d}{j_1}\\
B\arrow{r}{j_2}&X
\end{tikzcd}\Longrightarrow
\begin{tikzcd}
\pi_1(A\cap B,x_0)\arrow{r}{(i_1)_\ast}\arrow[swap]{d}{(i_2)_\ast}&\pi_1(A,x_0)\arrow{d}{(j_1)_\ast}\\
\pi_1(B,x_0)\arrow{r}{(j_2)_\ast}&\pi_1(X,x_0)
\end{tikzcd}\]

\begin{theorem}[van Kampen]
Let $\{A_\alpha\}$ be an open cover of $X$ such that each $A_\alpha$ is path-connected and contains a common $x_0\in X$. Let $j_\alpha:A_\alpha\hookrightarrow X$ and $i_{\alpha\beta}:A_\alpha\cap A_\beta\hookrightarrow A_\alpha$ be inclusions, and $\Phi:\ast_\alpha\pi_1(A_\alpha,x_0)\to\pi_1(X,x_0)$ be the unique extension of $(j_\alpha)_\ast$. Then\begin{enumerate}[(i)]
    \item if $A_\alpha\cap A_\beta$ is path-connected, then $\Phi$ is surjective;
    \item if $A_\alpha\cap A_\beta\cap A_\gamma$ is path-connected, then $\ker\Phi$ is the smallest normal subgroup $N$ containing $(i_{\alpha\beta})_\ast[f](i_{\beta\alpha})_\ast^{-1}[f]$ where $[f]\in\pi_1(A_\alpha\cap A_\beta)$.
\end{enumerate}
\end{theorem}
\begin{proof}
\textit{Proof of (i).} Given a loop $f$ in $X$ based at $x_0$, choose a partition (invoking the compactness argument) $0=t_0<t_1<\cdots<t_n=1$ such that $f([t_{i-1},t_i])\in A_\alpha$ for some $\alpha$. We shall subsequently call such $A_\alpha$ as $A_i$, and $f_i=f|_{[t_{i-1},t_i]}$. For each $i$ pick a path $g_i$ in $A_i\cap A_{i+1}$ from $x_0$ to $f(t_i)$. Then $f\simeq_p(f_1\cdot\overline{g}_1)\cdot(g_1\cdot f_2\cdot\overline{g}_2)\cdot\cdots\cdot(g_{n-1}\cdot f_n)$. Since each term on the right is contained in a single $A_\alpha$, its image under $\Phi$ is precisely $f$.\medbreak

\textit{Proof of (ii).} We call $(f_1\cdot\overline{g}_1)\cdot(g_1\cdot f_2\cdot\overline{g}_2)\cdot\cdots\cdot(g_{n-1}\cdot f_n)$ a factorization of $f$, and define two factorization to be equivalent if one can be obtained from another by finite combinations of the following operations:\begin{enumerate}[(i)]
    \item word reduction, i.e., combine $[f_i][f_{i+1}]=[f_i\cdot f_{i+1}]$ if $[f_i],[f_{i+1}]\in\pi_1A_\alpha$;
    \item if $[f_i]\in\pi_1A_\alpha$ is the image of $\pi_1(A_\alpha\cap A_\beta)$ under $(i_\alpha)_\ast$, then $[f_i]\in\pi_1A_\beta$.
\end{enumerate}
Note that (ii) does not change the class of elements in $Q=\ast_\alpha\pi_1(A_\alpha,x_0)/N$. If we can show that any two factorization of $f$ are equivalent, then the induced $\phi':Q\to\pi_1(X,x_0)$ is injective, and $\ker\Phi=N$.\medbreak

ewf
\end{proof}

\begin{example}
Van Kampen's theorem provides a quick proof of $\pi_1\mathbf{S}^n=0$ for $n\geq2$. Just take the northern and southern hemisphere with a little overlap to form an open cover of $\mathbf{S}^1$, and each is nullhomotopic. Hence $\pi_1\mathbf{S}^n=0\ast0=0$.
\end{example}

\begin{example}
A surface with genus $n$ has fundamental group $n$ has . For example $\pi_1(\mathbf{T}^2)\cong\langle a,b:aba^{-1}b^{-1}\rangle=F^{ab}(a,b)$
\end{example}

\begin{example}
cell attaching
\end{example}

\subsection{Covering spaces}\label{8}

Suppose $p:\widetilde{X}\to X$ and open $U\subseteq X$. $U$ is \textbf{evenly covered} if $p^{-1}(U)=\sqcup_\alpha V_\alpha$ where each $V_\alpha$ (called \textbf{slice}) is an open subset of $\widetilde{X}$ restricted to which $p$ maps homeomorphically to $U$. If there exists an open cover of $X$ such that each member is evenly covered, then $(\widetilde{X},p)$ \textbf{evenly covers} $X$.\medbreak

A \textbf{lift} of a map $f:Y\to X$ is a map $\widetilde{f}:Y\to\widetilde{X}$ such that $p\widetilde{f}=f$. $\Phi:\pi_1(X,x_0)\to p^{-1}(x_0)$ by $[f]\mapsto\widetilde{f}(1)$ is called the \textbf{lifting correspondence}. The fact that $\Phi$ is well-defined is due to the uniqueness in homotopy-lifting property:

\begin{theorem}
Given a covering map $p:\widetilde{X}\to X$, a homotopy $f_t:Y\to X$, and a lift $\widetilde{f}_0:Y\to\widetilde{X}$ of $f_0$, there exists a unique lift $\widetilde{f}:Y\to\widetilde{X}$ extending $\widetilde{f}_0$.
\[\begin{tikzcd}
Y\times\{0\}\arrow{r}{\widetilde{f}_0}\arrow[hookrightarrow]{d}&\widetilde{X}\arrow{d}{p}\\
Y\times[0,1]\arrow{r}{F}\arrow[dashed]{ur}{!\widetilde{F}}&X
\end{tikzcd}\]
\end{theorem}
\begin{proof}
We first prove that there is an extension $\widetilde{F}:N(y_0)\times[0,1]\to\R$ where $N(y_0)$ is a neighborhood of $y_0\in Y$. Then we will prove the uniqueness determined by such extension.\medbreak

By continuity of $F$, for each $(y_0,t)\in Y\times[0,1]$ there exists a product neighborhood $N(y_0)\times(a,b)$ such that $F(N(y_0)\times(a,b))\subseteq U_\alpha$ where $\{U_\alpha\}$ is an open cover of $\mathbf{S}^1$. By compactness, $\{y_0\}\times[0,1]$ can be covered by finitely many sets. Hence choose a partition $0=t_0<t_1<\cdots<t_m=1$ such that $F(N(y_0)\times[t_{i-1},t_i])\subseteq U_i$.\medbreak

Assume inductively that $\widetilde{F}$ has been defined on $N(y_0)\times[0,t_{i-1}]$, then $\widetilde{F}(y_0,t_{i-1})\in\widetilde{U}_i$ where $\widetilde{U}_i$ is some path-connected component of $p^{-1}(U_i)$. Shrink $N(y_0)$ so that $\widetilde{F}(N(y_0)\times\{t_{i-1}\})\subseteq\widetilde{U}_i$, and define $\widetilde{F}$ on $N(y_0)\times[t_{i-1},t_i]$ by $p^{-1}F$, which is possible since $p:\widetilde{U}_\alpha\to U_\alpha$ is a homeomorphism, given that $\widetilde{U}_\alpha$ is a connected component of $p^{-1}(U_\alpha)$. Continue this process and induction gives an extension $\widetilde{F}:N(y_0)\times[0,1]\to\R$.\medbreak

Suppose for $F:[0,1]\to\mathbf{S}^1$ there exists two lifts $\widetilde{F},\widetilde{F}':[0,1]\to\R$ such that $\widetilde{F}(0)=\widetilde{F}'(0)$. Choose partition $0<t_0<t_1<\cdots<t_m=1$ such that $F([t_i,t_{i+1}])\subseteq U_i$ and assume inductively that $\widetilde{F}=\widetilde{F}'$ on $[t_{i-1},t_i]$. Then $\widetilde{F}([t_i,t_{i+1}])$ and $\widetilde{F}'([t_i,t_{i+1}])$ are in the same path-connected component $\widetilde{U}_i$ since $\widetilde{F}(t_i)=\widetilde{F}'(t_i)$. Since $p:\widetilde{U}_i\to U_i$ is injective, $p\widetilde{F}=p\widetilde{F}'$ gives $\widetilde{F}=\widetilde{F}'$ on $[t_i,t_{i+1}]$. Now by induction $\widetilde{F}=\widetilde{F}'$ on $[0,1]$.\medbreak

Now for an open cover $\{Y_\alpha\}$ of $Y$, construct $\widetilde{F}$ on each $Y_\alpha\times[0,1]$. If $y\times[0,1]$ is in both $Y_\alpha\times[0,1]$ and $Y_\beta\times[0,1]$, then $\widetilde{F}$ and $\widetilde{F}'$ coincide since $\widetilde{F}(y,0)=\widetilde{F}'(y,0)$ is fixed.
\end{proof}

\begin{example}
Now we are finally ready to prove that $\pi_1\mathbf{S}^1\cong\Z$.
\begin{proof}
Let $p:\R\to\mathbf{S}^1$ by $s\mapsto(\cos2\pi s,\sin2\pi s)$ be the projection and $\widetilde{\omega}_n(s)=ns$ be a path in $\R$ going from $0$ to $n$. Then $\omega_n=p\widetilde{\omega}_n$ by $s\mapsto(\cos2\pi ns,\sin2\pi ns)$ gives loops on $\mathbf{S}^1$ based at $(1,0)$. Let $\Phi:\Z\to\pi_1(\mathbf{S}^1,(1,0))$ by $n\mapsto[\omega_n]$. We will show that $\Phi$ is a group isomorphism.\medbreak

\textit{Group homomorphism.} Let $\tau_m:\R\to\R$ by $s\mapsto s+m$, then $\widetilde{\omega}_m\cdot\tau_m\widetilde{\omega}_n$ is a path going from $0$ to $m+n$. Hence $\Phi(m+n)=[\omega_{m+n}]=[p(\widetilde{\omega}_m\cdot\tau_m\widetilde{\omega}_n)]=[p\widetilde{\omega}_m\cdot p\tau_m\widetilde{\omega}_n]=[\omega_m\cdot\omega_n]$.\medbreak

\textit{Surjection.} For any $[f]\in\pi_1(\mathbf{S}^1,(1,0))$ and $f':\{0\}\to\R$ by $f'(0)=0=p^{-1}(1,0)$, apply lemma to $Y$ being a point gives a unique lift $\widetilde{f}:[0,1]\to\R$ of $f$ extending $f'$ such that $\widetilde{f}(0)=0$. Since $p(\widetilde{f}(1))=f(1)=(0,1)$, we have $\widetilde{f}(1)\in p^{-1}(1,0)$. But $p^{-1}(1,0)=\Z$, hence there exists $n\in\Z$ such that $\widetilde{f}=\widetilde{\omega}_n$ and subsequently $\Phi(n)=[\omega_n]=[p\widetilde{\omega}_n]=[f]$.\medbreak

\textit{Injection. }Let $\Phi(m)=\Phi(n)$, then suppose $\omega_m\simeq_p\omega_n$ by path homotopy $f_t:[0,1]\to\mathbf{S}^1$ such that $f_0=\omega_m$ and $f_1=\omega_n$. Apply lemma to Y being $[0,1]$ gives a unique lift of path homotopy $\widetilde{f}_t:[0,1]\to\R$ such that $\widetilde{f}_t(0)=0=p^{-1}(1,0)$. Hence $m=\widetilde{f}_0(1)=\widetilde{f}_1(1)=n$.
\end{proof}
\end{example}

\begin{corollary}
Let $p:(\widetilde{X},\widetilde{x}_0)\to(X,x_0)$ be a covering map, then\begin{enumerate}[(i)]
    \item $p_\ast$ is injective and $p_\ast(\pi_1(\widetilde{X},\widetilde{x}_0))$ is a subgroup of $\pi_1(X,x_0)$;
    \item $p_\ast(\pi_1(\widetilde{X},\widetilde{x}_0))$ consists of the homotopy classes of loops in $(X,x_0)$ that lifts to loops in $(\widetilde{X},\widetilde{x}_0)$;
    \item the number of sheets $|p^{-1}(x_0)|$ is equal to the number of cosets $|\pi_1(X,x_0)/p_\ast(\pi_1(\widetilde{X},\widetilde{x}_0))|$.
\end{enumerate}
\end{corollary}

Note that we did not put any group structure on $p_\ast(\pi_1(\widetilde{X},x_0))/\pi_1(X,x_0)$, as $\pi_1(X,x_0)$ may not be a normal subgroup.\medbreak

A covering space $(\widetilde{X},p)$ of $X$ is \textbf{universal} if $\widetilde{X}$ is simply connected.

\begin{example}
Consider the real projective space $\R\mathbf{P}^n$ consisting of all lines in $\R^{n+1}$ passing through the origin. Alternatively $\R\mathbf{P}^n$ can be constructed by identifying antipodal points on $\mathbf{S}^n$. The natural map $p:\mathbf{S}^n\to\R\mathbf{P}^n$ is a covering map with $\abs{p^{-1}(x_0)}=2$, and since $\mathbf{S}^n$ is simply connected, $\abs{\pi_1(\R\mathbf{P}^n)}=2$ which implies that $\pi_1(\R\mathbf{P}^n)\cong\Z/2\Z$.
\end{example}

locally path-connected
lifting criterion

\begin{theorem}
Let $f:(Y,y_0)\to(X,x_0)$ where $Y$ is path-connected and locally path-connected. Let $p:(\widetilde{X},\widetilde{x}_0)\to(X,x_0)$ be a covering map. Then there exists a lift $\widetilde{f}$ of $f$ if and only if $f_\ast(\pi_1(Y,y_0))\subseteq p_\ast(\pi(\widetilde{X},\widetilde{x}_0))$.
\end{theorem}
\begin{proof}
If a lift $\widetilde{f}$ exists, then trivially $f(\pi_1(Y,y_0))=p\circ\widetilde{f}(\pi_1(Y,y_0))\subseteq p(\pi_1(\widetilde{X},\widetilde{x}_0))$.\medbreak

Conversely, for any $y\in Y$ since $Y$ is path-connected, choose a path $g_y$ in $Y$ from $y_0$ to $y$. Since $f\circ g_y$ is a path in $(X,x_0)$, lift $f\circ g_y$ and define $\widetilde{f}(y)=\widetilde{f\circ g_y}(1)$. Note that the definition is independent of the choice of $g_y$, for if there is another such path $g_y'$, then let $h_0=f\circ(g_y'\cdot\overline{g_y})=(f\circ g_y')\cdot(\overline{f\cdot g_y})$ be a loop in $(X,x_0)$. Since $[h_0]\in f_\ast(\pi_1(Y,y_0))\subseteq p_\ast(\pi_1(\widetilde{X},\widetilde{x}_0))$, $h_0\simeq_pp\circ k$ for some loop $K$ in $(\widetilde{X},\widetilde{x}_0)$. By uniqueness of path-lifting property $\widetilde{h_0}\simeq_pk$, and hence $\widetilde{f}(y)=\widetilde{f\circ g'_y}(1)=\widetilde{\overline{f\circ g_y}}(0)=\widetilde{f\circ g_y}(1)=\widetilde{f}(y)$.\medbreak

few
\end{proof}

$X$ is \textbf{semi-locally simply connected} if for every $x\in X$ there exists a neighborhood $U$ such that $i_\ast(\pi_1U)$ is trivial in $X$ where $i:U\hookrightarrow X$ is the inclusion. Examples include Hawaiian earrings. If $X$ has a universal cover, then $X$ is semi-locally simply connected.

\begin{theorem}
If $X$ is path-connected, locally path-connected, and semi-locally simply connected, then $X$ has a universal cover.
\end{theorem}
\begin{proof}
Observe that if such universal cover $\widetilde{X}$ exists, then there is a one-to-one correspondence between points in $\widetilde{X}$ and path homotopy classes of paths in $\widetilde{X}$ starting at some fixed $\widetilde{x}_0\in\widetilde{X}$, and thus by $p$, the path homotopy classes of paths in $X$ starting at some $x\in X$. This prompts us to construct $\widetilde{X}$ using fundamental groupoid $\mathcal{P}(X)$.\medbreak

Let $\widetilde{X}=\{[f]\in\mathcal{P}(X):f(0)=x_0\}$ and $p:\widetilde{X}\to X$ by $[f]\mapsto f(1)$. We now construct a topology on $\widetilde{X}$ so that $p$ is a covering map and $\widetilde{X}$ is simply connected.\medbreak

Let $\mathcal{B}_x=\{U\subseteq X:U\textrm{ is path-connected and }i_\ast(\pi_1U)\textrm{ is trivial in }\pi_1X\}$, then $X=\cup_{U\in\mathcal{B}_x}U$, and if $U_1,U_2\in\mathcal{B}_x$ and $x\in U_1\cap U_2$ then there exists an open and path-connected $V\subseteq U_1\cap U_2$ such that $(ij)_\ast(\pi_1V)=i_\ast(j_\ast(\pi_1V))\subseteq i_\ast(\pi_1U_1)$. Hence $V\in\mathcal{B}_x$, and $\mathcal{B}_x$ is a basis for the topology on $X$.\medbreak

Given $U_\alpha\in\mathcal{B}_x$, then $p^{-1}(U_\alpha)=\sqcup_{x\in U_\alpha}\{[f]\in\mathcal{P}(X):f(0)=x_0\textrm{ and }f(1)=x\}$. Define equivalence relation $[f_1]\sim_{U_\alpha}[f_2]$ if there exists a path $g$ in $U_\alpha$ such that $g\cdot f_1=f_2$, then $\sim_{U_\alpha}$ breaks $p^{-1}(U_\alpha)=\sqcup_\beta V_{\alpha\beta}$ where $V_{\alpha\beta}=\{[f\cdot g]\in\mathcal{P}(X):g\textrm{ in }U_\alpha\textrm{ and }g(0)=f(1),f\textrm{ fixed}\}$.\medbreak

Observe that $p:V_{\alpha\beta}\to U_\alpha$ is surjective on $V_{\alpha\beta}$ since $U_\alpha$ is path-connected, and it is injective by semi-locally simply connectedness of $X$. Now let $\mathcal{B}_{\widetilde{x}}=\{V_{\alpha\beta}\}$, then $\widetilde{X}=\cup_{\alpha,\beta}V_{\alpha\beta}$. Moreover for $y\in\widetilde{X}$ and $y\in V_{\alpha\beta}\cap V_{\alpha'\beta'}$, there exists $U\in\mathcal{B}_x$ such that $p(y)\in U\subseteq U_\alpha\cap U_\beta=p(V_{\alpha\beta})\cap p(V_{\alpha'\beta'})$. Now let $V=p^{-1}(U)$, then $y\in V$ and $V\subseteq V_{\alpha\beta}\cap V_{\alpha'\beta'}$. Hence $\mathcal{B}_{\widetilde{x}}$ is a basis, and $p:\widetilde{X}\to X$ is a covering map.\medbreak

Finally, let's check that $\widetilde{X}$ is simply connected. Let $[f]\in p_\ast(\pi_1(\widetilde{X},\widetilde{x}_0))$, then $\widetilde{f}:[0,1]\to\widetilde{X}$ by $t\mapsto[f_t]$ is a lift of $f$, where $f_t=f|_{[0,t]}$. From a previous corollary $\widetilde{f}$ is a loop, hence $[f]=\widetilde{f}(1)=\widetilde{x}_0$ is the class of constant loops. Since $p_\ast$ is injective, $\pi_1(\widetilde{X},\widetilde{x}_0)$ has to be trivial.
\end{proof}

\begin{corollary}
If $X$ is path-connected, locally path-connected, and semi-locally simply connected, then for every subgroup $H\subseteq\pi_1(X,x_0)$ there is a covering space $((\widetilde{X}_H,\widetilde{x}_0),p)$ such that $p_\ast(\widetilde{X}_H,\widetilde{x}_0)=H$.
\end{corollary}
\begin{proof}

\end{proof}

\subsection{Transformations and actions}\label{9}


\newpage
\section{Homology}\label{10}

\subsection{Simplicial homology and singular homology}\label{11}

$\Delta$-complex structure $\Rightarrow$ simplicial homology; singular simplices $\Rightarrow$ singular homology.\medbreak

Calculation example of simplicial homology: torus $\mathbf{T}^2$ has $\Delta$-complex structure
\[\begin{tikzcd}[column sep=huge,row sep=huge]
v\arrow{r}{b}&v\\
v\arrow{u}{a}\arrow[swap]{r}{b}\arrow[ur,"U" near end,"L"' near start,"c"]&v\arrow[swap]{u}{a}
\end{tikzcd}\Longrightarrow\begin{cases}
\textrm{2-simplices: }\sigma_U,\sigma_L\\
\textrm{1-simplices: }\sigma_a,\sigma_b,\sigma_c\\
\textrm{0-simplices: }\sigma_v
\end{cases}.\]
Thus the chain complex obtained is as follows:
\[\begin{tikzcd}[row sep=small]
0\arrow{r}{\partial_3}&\Delta_2(\mathbf{T}^2)\arrow[dash]{d}{\cong}\arrow{r}{\partial_2}&\Delta_1(\mathbf{T}^2)\arrow[dash]{d}{\cong}\arrow{r}{\partial_1}&\Delta_0(\mathbf{T}^2)\arrow{r}{\partial_0}\arrow[dash]{d}{\cong}&0\\
&\Z^2&\Z^3&\Z
\end{tikzcd}.\]
Observe that since $\partial_2\sigma_U=\partial_2\sigma_L=b+a-c$, $\ker\partial_2=\im\partial_2\cong\Z$, and hence
$H_2^\Delta(\mathbf{T}^2)=\ker\partial_2\cong\Z$. Since $\partial_1\sigma_a=\partial_1\sigma_b=\partial_1\sigma_c=0$, $\ker\partial_1=\Delta_1(\mathbf{T}^2)\cong\Z^3$, and $H_1^\Delta(\mathbf{T}^2)=\ker\partial_1/\im\partial_2\cong\Z^3/\Z=\Z^2$. Finally, it's easy to see that $H_n^\Delta(\mathbf{T}^2)=0$ for $n\in\{0,3\}$.

\begin{lemma}
If $X_\alpha$ are path-connected components of $X$, then $H_n(X)\cong\oplus_\alpha H_n(X_\alpha)$.
\end{lemma}

\textbf{Reduced homology groups} $\widetilde{H}_n(X)$ strip a copy of $\Z$ off $H_0(X)$:
\[\begin{tikzcd}
\cdots\arrow{r}&C_1(X)\arrow{r}{\partial_1}&C_0(X)\arrow{r}{\epsilon}&\Z\arrow[r]&0
\end{tikzcd}.\]

\begin{lemma}
If $X$ is path-connected, then $H_0(X)\cong\Z$.
\end{lemma}
\begin{proof}
wrew
\end{proof}

Given $f:X\to Y$, we obtain $f_\#:(C.(X),\partial.)\to(C.(Y),\partial.)$ by $\sigma\mapsto f\circ\sigma$ and noticing that $f_\#(\partial\sigma)=\partial f_\#(\sigma)$. Then we obtain for each $n$ the induced $f_\ast:H_n(X)\to H_n(Y)$ (sometimes denoted by $H_n(f)$) such that $\partial_n^Bh_n=h_{n-1}\partial_n^A$. Notice that since $(f\circ g)_\ast=f_\ast\circ g_\ast$ and $(\id_X)_\ast=\id_{H.(X)}$, $H_n$ acts as a functor from the category of chain complexes to the category of abelian groups.
\[\begin{tikzcd}
\Delta^n\arrow{dr}{\sigma}\arrow{rr}{f_\#\sigma}&&Y\\
&X\arrow{ur}{f}
\end{tikzcd}\quad\quad\quad\begin{tikzcd}
\cdots\arrow{r}&C_n(A)\arrow{d}{h_n}\arrow{r}{\partial_n^A}&C_{n-1}(A)\arrow{d}{h_{n-1}}\arrow{r}&\cdots\\
\cdots\arrow{r}&C_n(B)\arrow{r}{\partial_n^B}&C_{n-1}(B)\arrow{r}&\cdots
\end{tikzcd}\]

That a homeomorphism $f:X\to Y$ induces an isomorphism $H_n(f):H_n(X)\to H_n(Y)$ is immediate from definition. But we also expect, like before, that homotopy equivalence is enough.\medbreak

\begin{theorem}
If $f,g:X\to Y$ and $f\simeq g$, then $f_\ast=g_\ast$.
\end{theorem}
\begin{proof}
blah
\end{proof}

Now since $(f\circ g)_\ast=f_\ast\circ g_\ast$ and $\id_\ast=\id$, we have the following:

\begin{corollary}
If $X\simeq Y$ by $f$, then $H_n(X)\cong H_n(Y)$ by $H_n(f)$. In particular,     $\widetilde{H}_n(X)=0$ for all $n$ if $X$ is contractible.
\end{corollary}

\subsection{Zig-zag lemma and Excision}\label{12}

\begin{theorem}[zig-zag lemma]\label{zig-zag lemma}
A short exact sequence of chain complexes
\[\begin{tikzcd}[column sep=small]
&\vdots\arrow{d}&\vdots\arrow{d}&\vdots\arrow{d}\\
0\arrow{r}&C_n(A)\arrow{d}{\partial}\arrow{r}{i}&C_n(B)\arrow{d}{\partial}\arrow{r}{j}&C_n(C)\arrow{d}{\partial}\arrow{r}&0\\
0\arrow{r}&C_{n-1}(A)\arrow{d}\arrow{r}{i}&C_{n-1}(B)\arrow{d}\arrow{r}{j}&C_{n-1}(C)\arrow{d}\arrow{r}&0\\
&\vdots&\vdots&\vdots
\end{tikzcd}\]
induces a long exact sequence of homology groups
\[\begin{tikzcd}[column sep=scriptsize]
\cdots\arrow{r}&H_{n+1}(C)\arrow{r}{\partial}&H_n(A)\arrow{r}{i_\ast}&H_n(B)\arrow{r}{j_\ast}&H_n(C)\arrow{r}{\partial}&H_{n-1}(A)\arrow{r}&\cdots
\end{tikzcd}.\]
\end{theorem}
\begin{proof}
We only need to construct the \textbf{connecting homomorphism} $\partial:H_{n+1}(C.)\to H_n(A.)$. Let $[c]\in H_{n+1}(C.)$, then $c\in C_{n+1}$ and $\partial c=0$. Since $j$ is surjective, there exists $b\in B_{n+1}$ such that $j(b)=c$ and $j(\partial b)=\partial j(b)=0$ by commutativity. Now since $\partial b\in\ker j\cong\im i$, there exists $a\in A_n$ such that $i(a)=\partial b$. Finally, $i(\partial a)=\partial i(a)=\partial\partial b=0$, and since $i$ is injective, $\partial a=0$. Hence $\partial[c]=[a]\in H_n(A.)$, after checking that the choices of $[c]$ and $b$ does not affect the result.
\[\begin{tikzcd}
&&B_{n+1}\arrow{d}{\partial}\arrow{r}{j}&C_{n+1}\arrow{d}{\partial}\arrow{r}&0\\
&A_n\arrow{d}{\partial}\arrow{r}{i}&B_n\arrow{d}{\partial}\arrow{r}{j}&C_n\\
0\arrow{r}&A_{n-1}\arrow{r}{i}&B_{n-1}
\end{tikzcd}.\]
Finally, check that the sequence of homology groups obtained is indeed exact. This method of proving is referred to as diagram chasing.
\end{proof}

Let $A\subseteq X$, then $(C.(A),\partial)$ is a subcomplex of $(C.(X),\partial)$. Let $C_n(X,A)=C_n(X)/C_n(A)$ be the \textbf{relative} chain group, and the inherited $\partial:C_n(A)\to C_{n-1}(A)$ makes $C.(X,A)$ a chain complex. Elements of the \textbf{relative} homology group $H_n(X,A)$ are \textbf{relative} cycles $\alpha\in C_n(X)$ such that $\partial\alpha\in C_{n-1}(A)$. $\alpha$ is trivial in $H_n(X,A)$ when $\alpha=\gamma+\partial\beta$ where $\beta\in C_{n+1}(X)$ and $\gamma\in C_n(A)$.\medbreak

Apply the \hyperref[zig-zag lemma]{zig-zag lemma} on the exact sequence $\begin{tikzcd}[column sep=scriptsize]
0\arrow{r}&C.(A)\arrow[hookrightarrow]{r}{i}&C.(X)\arrow[twoheadrightarrow]{r}{j}&C.(X,A)\arrow{r}&0
\end{tikzcd}$:
\[\begin{tikzcd}[column sep=scriptsize]
\cdots\arrow{r}&H_{n+1}(X,A)\arrow{r}{\partial}&H_n(A)\arrow{r}{i_\ast}&H_n(X)\arrow{r}{j_\ast}&H_n(X,A)\arrow{r}{\partial}&H_{n-1}(A)\arrow{r}&\cdots
\end{tikzcd}.\]
where the connecting homomorphism $\partial:H_{n+1}(X,A)\to H_n(A)$ corresponds with the boundary map since $\partial[\alpha]=[\partial\alpha]$.

Notice that $\widetilde{H}_n(X,A)\cong H_n(X,A)$ for all $n$, since
\[\begin{rcases}\begin{tikzcd}[column sep=small, row sep=small]
\cdots\arrow{r}&C_1(A)\arrow{r}&C_0(A)\arrow[r]&\Z\arrow[r]&0\\
\cdots\arrow{r}&C_1(X)\arrow{r}&C_0(X)\arrow[r]&\Z\arrow[r]&0
\end{tikzcd}\end{rcases}\Longrightarrow\begin{tikzcd}[column sep=small, row sep=small]
\cdots\arrow{r}&C_1(X,A)\arrow{r}&C_0(X,A)\arrow[r]&0\arrow[r]&0
\end{tikzcd}\]

Computation of $\widetilde{H}_n(\mathbf{S}^m)$: noticing that $\mathbf{D}^m/\partial\mathbf{D}^m\cong\mathbf{D}^m$ and $\widetilde{H}_n(\mathbf{D}^m)=0$ since it is contractible,
\[\begin{tikzcd}[column sep=scriptsize]
0\arrow[r]&C.(\partial\mathbf{D}^m)\arrow{r}{i}&C.(\mathbf{D}^m)\arrow{r}{j}&C.(\mathbf{D}^m/\partial\mathbf{D}^m)\arrow[r]&0
\end{tikzcd}\]
\[\textrm{induces }\begin{tikzcd}
0\arrow{r}{j_\ast}&\widetilde{H}_n(\mathbf{S}^m)\arrow{r}{\partial}&\widetilde{H}_{n-1}(\mathbf{S}^{m-1})\arrow{r}{i_\ast}&0
\end{tikzcd}\]
where $\partial$ is bijective. Hence $\widetilde{H}_n(\mathbf{S}^m)\cong\widetilde{H}_{n-1}(\mathbf{S}^{m-1})$. Now since $\widetilde{H}_0(\mathbf{S}^0)=\Z$ and $\widetilde{H}_n(\mathbf{S}^1)=0$ for $n>0$, we have that $\widetilde{H}_n(\mathbf{S}^n)=\Z$ and $\widetilde{H}_n(\mathbf{S}^m)=0$ for $m\neq n$. In place of $C.(\mathbf{D}^m/\partial\mathbf{D}^m)$ write $C.(\mathbf{D}^m,\partial\mathbf{D}^m)$, we also obtain $\widetilde{H}_n(\mathbf{D}^m,\partial\mathbf{D}^m)\cong\mathbf{H}_{n-1}(\mathbf{S}^{m-1})$.\medbreak

Hence $\R^m$ is not homeomorphic to $\R^n$.

\begin{corollary}[Brouwer fixed-point theorem]
Every map $f:\mathbf{D}^n\to\mathbf{D}^n$ has a fixed point.
\end{corollary}
\[\begin{tikzcd}
\partial\mathbf{D}^n\arrow[swap]{dr}{\id}\arrow{r}{i}&\mathbf{D}^n\arrow{d}{r}\\
&\partial\mathbf{D}^n
\end{tikzcd}\Longrightarrow\begin{tikzcd}
\widetilde{H}_{n-1}(\mathbf{S}^{n-1})\cong\Z\arrow[swap]{dr}{\id}\arrow{r}{i_\ast}&\widetilde{H}_{n-1}(\mathbf{D}^n)\cong0\arrow{d}{r_\ast}\\
&\Z
\end{tikzcd}\]

If $X$ has a $\Delta$-complex structure such that $A,X\setminus Z,A\setminus Z$ are $\Delta$-subcomplexes, then
\[\begin{tikzcd}
\Delta_n(X\setminus Z)\arrow[bend right=25]{rr}{\varphi}\arrow{r}&\Delta_n(X)\arrow[r]&\Delta_n(X,A)
\end{tikzcd}.\]
$\varphi$ is surjective since a basis of $\Delta_n(X,A)$ is given by subcomplexes of $X\setminus A\subseteq X\setminus Z$. Hence $\varphi$ induces an isomorphism $\Delta_n(X,A)\cong\Delta_n(X\setminus Z)/\ker\varphi=\Delta_n(X\setminus Z)/\Delta_n(A\setminus Z)=\Delta_n(X\setminus Z,A\setminus Z)$.

\begin{theorem}[excision theorem]
If $\overline{Z}\subseteq A^\circ$, then $H_n(X,A)\cong H_n(X\setminus Z,A\setminus Z)$ for all $n$.
\end{theorem}
\begin{proof}
Let $\mathcal{U}=\{U_\alpha\}_{\alpha\in A}$ be an open cover of $X$. Define $C_n^\mathcal{U}(X)\subseteq C_n(X)$ to be the subcomplex generated by $n$-simplices of $X$ such that $\sigma(\Delta^n)\subseteq U_\alpha$ for some $\alpha$. The boundary map inherited from $\partial:C_n(X)\to C_{n-1}(X)$ makes $(C.^\mathcal{U}(X),\partial)$ into a chain complex.\medbreak

By so-called barycentric subdivision $S:C.(X)\to C.(X)$ one can divide simplices so that each small simplex lies inside some $U_\alpha$, and by showing that $S$ is chain homotopic to the identity map, make sense of $H_n(C.^\mathcal{U}(X))\cong H_n(C.(X))$ for all $n$.\medbreak

Now let $Y=X\setminus Z$ and $\mathcal{U}=\{Y,A\}$. We have $C_n(X\setminus Z)/C_n(A\setminus Z)=C_n(Y)/C_n(Y\cap A)$, which is generated by the simplices that lie in $X\setminus A$. Hence $C_n(Y)/C_n(Y\cap A)\cong C_n^\mathcal{U}(X)/C_n(A)$. Now we have the exact sequences of homology by the zig-zag lemma:
\[\begin{tikzcd}[column sep=scriptsize]
H_n(C.(A))\arrow{d}{\cong}\arrow[r]&H_n(C.^\mathcal{U}(X))\arrow{d}{\cong}\arrow[r]&H_n(C.^\mathcal{U}(X)/C.(A))\arrow[d,dashed]\arrow[r]&H_{n-1}(C.(A))\arrow{d}{\cong}\arrow[r]&H_{n-1}(C.^\mathcal{U}(X))\arrow{d}{\cong}\\
H_n(C.(A))\arrow[r]&H_n(C.(X))\arrow[r]&H_n(C.(X)/C.(A))\arrow[r]&H_{n-1}(C.(A))\arrow[r]&H_{n-1}(C.(X))
\end{tikzcd}.\]
By the five lemma $H_n(C.^\mathcal{U}(X)/C.(A))\cong H_n(C.(X)/C.(A))$. Hence $H_n(X\setminus Z,A\setminus Z)\cong H_n(X,A)$.
\end{proof}

Equivalently, if $X=A^\circ\cup B^\circ$, then $(B,A\cap B)\hookrightarrow(X,A)$ induces $H_n(B,A\cap B)\cong H_n(X,A)$ for all $n$. To see this let $Z=X\setminus B$ and $B=X\setminus Z$ for the converse.\medbreak

For a good pair, $\widetilde{H}_n(X,A)\cong\widetilde{H}_n(X/A)$; for arbitrary pairs, $\widetilde{H}_n(X,A)\cong\widetilde{H}_n(X\cup\cone_A)$.\medbreak

Now immediately from this, we have [hatcher corollary 2.25]

\begin{lemma}
$H_n(\Delta^n,\partial\Delta^n)\cong\Z$ is generated by identity maps $\id:\Delta^n\to\Delta^n$.
\end{lemma}
\begin{proof}
When $n=0$ the statement is trivial. Proceed by induction on $n$.\medbreak

Let $\Lambda^n$ be $\partial\Delta^n$ setminus the last fact (in the case of a triangle $\Delta^2$, $\Lambda^2$ looks exactly like a triangle without the bottom side) and consider the exact sequence of the triple $(\Delta^{n+1},\partial\Delta^{n+1},\Lambda^{n+1})$
\[\begin{tikzcd}[column sep=small]
H_{n+1}(\Delta^{n+1},\Lambda^{n+1})\arrow{d}{\cong}\arrow[r]&H_{n+1}(\Delta^{n+1},\partial\Delta^{n+1})\arrow{r}{\partial}&H_n(\partial\Delta^{n+1},\Lambda^{n+1})\arrow{d}{\cong}\arrow[r]&H_n(\Delta^{n+1},\Lambda^{n+1})\arrow{d}{\cong}\\
0&&H_n(\Delta^n,\partial\Delta^n)&0
\end{tikzcd}\]
where $H_{n+1}(\Delta^{n+1},\Lambda^{n+1})\cong0$ because $\Delta^{n+1}$ deform retracts to $\Lambda^{n+1}$, and $H_n(\Delta^{n+1},\Lambda^{n+1})\cong0$ because $H_n(\Delta^{n+1})=0$. Hence $\partial$ is an isomorphism. Let $\id_{n+1}\in H_{n+1}(\Delta^{n+1},\partial\Delta^{n+1})$, then
\begin{align*}
\partial \id_{n+1}=\sum(-1)^k\id_{n+1}|_{k^{\textrm{th}}\textrm{ face}}=\pm\id_n\quad\quad\quad\textrm{(the last face)}
\end{align*}
which is a generator of $H_n(\Delta^n,\partial\Delta^n)$ by induction hypothesis. Since $\partial$ is an isomorphism, $\id_{n+1}$ is a generator of $H_{n+1}(\Delta^{n+1},\partial\Delta^{n+1})$.
\end{proof}

\begin{theorem}
Let $X$ be equipped with a $\Delta$-complex structure, then the inclusion $\Delta.(X)\hookrightarrow C.(X)$ of chain complexes induces an isomorphism $H_n^\Delta(X)\cong H_n(X)$ for all $n$.
\end{theorem}
\begin{proof}
First assume that $X$ is finite-dimensional.
\end{proof}

\bigbreak

2.1 basic concepts: simplicial homology, singular homology, cellular homology, relative homology, chain homotopy, reduced homology

example: torus, path-connected=direct sum, path-connected H0, etc

theorem 1 (homotopy)
theorem 2 (equivalence)
theorem 3 (cellular)

example:

2.2 tools

theorem 1 zig zag lemma
theorem 2 excision theorem
theorem 3 5 lemma
theorem 4 m-v theorem

example;

2.3 applications

brouwer fixed-point, degree, hairy ball, borsuk-ulam


2.4 axiomatic homology

\subsection{manifold recap}

An \textbf{n-dimensional manifold} is a second countable Hausdorff space that is locally homeomorphic to $\R^n$.\medbreak

Let $p\in U\subseteq M$ such that $U$ is homeomorphic to $\R^n$, then $H_k(U,U\setminus\{p\})\cong\widetilde{H}_k(\R^n,\R^n\setminus\{p\})$. Considering the pair $(\R^n,\R^n\setminus\{p\})$ we have $\widetilde{H}_k(\R^n,\R^n\setminus\{p\})\cong \widetilde{H}_k(\R^n\setminus\{p\})\cong\widetilde{H}_k(\mathbf{S}^{n-1})$. Finally by excision, $H_k(M,M\setminus\{p\})\cong H_k(U,U\setminus\{p\})$. Together we have 
\[\widetilde{H}_k(M,M\setminus\{p\})\cong\widetilde{H}_k(\mathbf{S}^{n-1})\cong\begin{cases}
\Z,&k=n\\
0,&\textrm{otherwise}
\end{cases}.\]
Hence if two manifolds $M$ and $N$ are homeomorphic, then $\dim M=\dim N$.\medbreak

The \textbf{local orientation} $\mu_p$ of a manifold $M$ at $p\in M$ is a choice of generator $\{\pm1\}$ of $H_n(M,M\setminus\{p\})\cong\Z$. Two orientations are \textbf{consistent} if. \textbf{global orientation}

\newpage
\subsection{cellular homology}

The cellular homology of $X$ helps the computation when $X$ is a CW-complex.\medbreak

We make the following observations when $X$ is a CW-complex:
\begin{enumerate}[(i)]
\item $H_n(X^k,X^{k-1})\cong\begin{cases}\bigoplus\Z\ (\textrm{one for each }n\textrm{-cell of }X),&n=k\\0,&\textrm{otherwise}
\end{cases}$.

\item $i_\ast:H_k(X^k)\to H_k(X)$ induced by $i:X^k\hookrightarrow X$ is surjective.

\item $H_n(X^k)\cong\begin{cases}H_n(X)\ (\textrm{induced by }i:X^k\hookrightarrow X),&n<k\\0,&n>k
\end{cases}$.
\end{enumerate}
(i) follows from $H_n(X^k,X^{k-1})\cong\widetilde{H}_n(X^k/X^{k-1})\cong\widetilde{H}_n(\vee_i\mathbf{S}^k)$ since $(X^k,X^{k-1})$ is a good pair. (ii) and (iii) when $X$ is finite-dimensional follows from considering the exact sequence of $(X^k,X^{k-1})$ and induction on $H_k(X^0)=0$ when $k>0$. When $X$ is infinite-dimensional recall that
\begin{align*}
\textrm{\textcolor{red}{a singular chain in $X$ has compact image, hence meets only finitely many cells of $X$.}}
\end{align*}

Define $C_n^{CW}(X)=H_n(X^n,X^{n-1})=F^{ab}(n\textrm{-cells of }X)$ to be the \textbf{cellular chain group}, and with $d_n=j_{n-1}\partial_n$ we obtain a \textbf{cellular chain complex} ($d^2=0$):
\[\begin{tikzcd}[column sep=small]
&&0\arrow[d]&&&&&&\\
&&H_n(X^n)\arrow[rrd, "j_n"]\arrow[rr]&&H_n(X^{n+1})\cong H_n(X)\arrow[rr]&&0&&\\
\cdots\arrow[rr]&&{H_{n+1}(X^{n+1},X^n)}\arrow[u,"\partial_{n+1}"]\arrow[rr,"d_{n+1}"]&& {H_n(X^n,X^{n-1})}\arrow[rr, "d_n"] \arrow[d, "\partial_n"] &  & {H_{n-1}(X^{n-1},X^{n-2})} \arrow[rr]&&\cdots\\
&&&&H_{n-1}(X^{n-1})\arrow[rru,"j_{n-1}"']&&&&\\
&&&&0\arrow[u]&&&&       
\end{tikzcd}.\]
The corresponding homology groups are the \textbf{cellular homology groups} $H_n^{CW}(X)$.

\begin{theorem}
The inclusion $C.^{CW}(x)\hookrightarrow C.(X)$ induces an isomorphism
$H_.^{CW}(X)\cong H.(X)$.
\end{theorem}
\begin{proof}
Since $j_n$ is injective, $\im\partial_{n+1}=\im d_{n+1}$ and $H_n(X^n)=\im j_n$. By exactness $\im j_n=\ker\partial_n$, and since $j_{n-1}$ is injective, $\ker\partial_n=\ker d_n$. Together we have $H_n^{CW}(X)\cong H_n(X^n)/\im\partial_{n+1}$, which by exactness is precisely $H_n(X)$.
\end{proof}

Observe that $d_{n+1}:H_{n+1}(X^{n+1},X^n)\to H_n(X^n,X^{n-1})$ sends $e_\alpha^{n+1}\mapsto\sum d_{\alpha\beta}e^n_\beta$. The claim is that $d_{\alpha\beta}=\deg f_{\alpha\beta}$ where $f_{\alpha\beta}=q_\beta\circ\varphi_\alpha$, with $\varphi_\alpha:\partial\mathbf{D}^n\to X$ the attaching map and $q_\beta:X\to\mathbf{S}^n_\beta$ collapsing all of $X$ except for $e_\beta^n$.

\[\begin{tikzcd}
\widetilde{H}_{n+1}(\mathbf{D}_\alpha^{n+1},\partial\mathbf{D}^{n+1}_\alpha)\arrow{d}{\Phi_{\alpha\ast}}\arrow{r}{\partial}&\widetilde{H}_n(\partial\mathbf{D}_\alpha^{n+1})\arrow{d}{\varphi_{\alpha\ast}}\arrow{r}{f_{\alpha\beta\ast}}&\widetilde{H}_n(\mathbf{S}_\beta^n)\\
\widetilde{H}_{n+1}(X^{n+1},X^n)\arrow[swap]{drr}{d_{n+1}}\arrow{r}{\partial}&\widetilde{H}_n(X^n)\arrow{ur}{q_{\beta\ast}}\arrow{r}{q_{\beta_1\ast}}&\widetilde{H}_n(X^n/X^{n-1})\arrow[swap]{u}{q_{\beta_2\ast}}\\
&&\widetilde{H}_n(X^n,X^{n-1})\arrow[equal]{u}
\end{tikzcd}\]

Take a generator $[\mathbf{D}^{n+1}]\in\widetilde{H}_{n+1}(\mathbf{D}_\alpha^{n+1},\partial\mathbf{D}^{n+1}_\alpha)$, $\partial$ sends it to a generator in $\widetilde{H}_n(\partial\mathbf{D}_\alpha^{n+1})$ which has the image $\deg f_{\alpha\beta}$ under $f_{\alpha\beta\ast}$. On the other hand, the characteristic map (extension of attaching map) $\Phi_{\alpha\ast}$ sends $[\mathbf{D}^{n+1}]$ to $[e_\alpha^{n+1}]$, and $d_{n+1}$ further sends it to $\sum d_{\alpha\beta}e_\beta^n$, which projects to the $\beta^{\mathrm{th}}$ factor $d_{\alpha\beta}$ by $q_{\beta_2\ast}$. Since the diagram commutes, $d_{\alpha\beta}=\deg f_{\alpha\beta}$.\medbreak

Now we no longer have to subdivide spaces to obtain a $\Delta$-complex structure like we did when computing simplicial homology of $\mathbf{T}^2$ and $\mathbf{RP}^2$.\medbreak

For $\mathbf{M}_g$, blah\medbreak

For $\mathbf{RP}^n$, blah\medbreak

For $\mathbf{CP}^n$: recall that $\mathbf{CP}^n=\mathbf{S}^{2n+1}/\sim$ where $v\sim\lambda v$ when $\abs{\lambda}=1$. The ``upper hemisphere'' of $\mathbf{CP}^n$ consists of points $(\omega,(1-\abs{\omega}^2)^{1/2})$ where $\omega\in\mathbf{C}^n\cong\mathbf{D}^{2n}$, the boundary of which corresponds to $(\omega,0)$ with the identification of $\mathbf{CP}^{n-1}$. Hence inductively $\mathbf{CP}^n$ is obtained from $\mathbf{CP}^{n-1}$ by attaching a $2n$-cell: $\mathbf{CP}^n=e_0\cup e_2\cup\cdots\cup e_{2n}$. Hence the cellular chain complex is an alternation between $0$ and $\Z$ with trivial boundary maps.
\[H_k(\mathbf{CP}^n)\cong\begin{cases}
\Z,&k=0,2,\cdots,2n\\0,&\textrm{otherwise}
\end{cases}.\]

\begin{lemma}
If $0\xrightarrow{ }A\xrightarrow{\alpha}B\xrightarrow{\beta}C\xrightarrow{ }0$ is exact, then so is $0\xrightarrow{ }A\otimes\Q\xrightarrow{\alpha\otimes\id}B\otimes\Q\xrightarrow{\beta\otimes\id}C\otimes\Q\xrightarrow{ }0$, given that $A,B,C$ are abelian groups, i.e., $\Q$ is flat over $\Z$.
\end{lemma}
\begin{proof}
If $(\alpha\otimes\id)(a\otimes q)=\alpha(a)\otimes q=0$, then there exists $n\in\N$ such that $n\alpha(a)=\alpha(na)=0$. Since $\alpha$ is injective, $na=0$ and thus $a\otimes q=0$, proving that $\ker(\alpha\otimes\id)=0$. Since $\beta\circ\alpha=0$, $(\beta\otimes\id)\circ(\alpha\otimes\id)=(\beta\circ\alpha)\otimes\id=0$, and $\im(\alpha\otimes\id)\subseteq\ker(\beta\otimes\id)$. Check the other direction. The surjectivity of $\beta\otimes\id$ inherits from the surjectivity of $\beta$.
\end{proof}

Euler Char: alternating sum of rank

theorem: number of k-cells,

proof: flatness of q, split exact sequence.

eg. $\chi(\mathbf{M}_g)=2-2g$, $\chi(\mathbf{S}^n)=\begin{cases}0,&n\textrm{ odd}\\2,&\textrm{otherwise}
\end{cases},\chi(\mathbf{RP}^n)=\begin{cases}0,&n\textrm{ odd}\\1,&\textrm{otherwise}
\end{cases}$,$\chi(\mathbf{CP}^n)=2n$




\end{document}