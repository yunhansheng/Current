\documentclass[hidelinks,11pt]{article}

\usepackage[margin=1in]{geometry}
\usepackage{amsmath,amsthm,amsfonts}
\usepackage[utf8]{inputenc}
\usepackage{amssymb}
\usepackage[mathscr]{eucal}
\usepackage{graphicx}
\usepackage{listings}
\usepackage{xcolor}
\usepackage[OT1]{fontenc}
\usepackage{physics}
\usepackage{tikz-cd}
\usepackage{enumitem}
\usepackage{xpatch}
\usepackage{nicefrac}
\usepackage{mathtools}
\usepackage{environ}

\setlength\parindent{0pt}
\setcounter{section}{-0}

\newtheoremstyle{dotless}{}{}{\itshape}{}{\bfseries}{}{ }{\thmname{#1}\thmnumber{ #2}\thmnote{ (#3)}}

\theoremstyle{definition}
\newtheorem*{defin}{DEF}

\theoremstyle{dotless}
\newtheorem*{corollary}{Corollary}
\newtheorem{prop}{PROP}[section]

\theoremstyle{remark}
\newtheorem*{remark}{Remark}

\usepackage{hyperref}
\hypersetup{colorlinks=true,linkcolor=magenta}

\DeclareMathOperator{\Var}{Var}
\DeclareMathOperator{\E}{\mathbb{E}}
\DeclareMathOperator{\R}{\mathbb{R}}
\DeclareMathOperator{\1}{\mathbf{1}}
\DeclareMathOperator{\p}{\mathbb{P}}

%\NewEnviron{killcontents}{}\let\proof\killcontents\let\endproof\endkillcontents



\begin{document}

\tableofcontents

\section{Weak convergence}
Consider a metric space $(S,\rho)$ with topology induced by $\rho$, the Borel $\sigma$-algebra $\mathcal{S}$ of $S$, and a probability measure $\mathbb{P}$ on $(S,\mathcal{S})$.
\begin{defin}
$\p_n$ \textbf{converges weakly} to $\p$, or $\p_n\xrightarrow{w}\p$, if $\p_nf\to\p f$ for every bounded and continuous $f:S\to\R$, where $\p_nf=\int_Sf\,d\p_n$.
\end{defin}
\begin{prop}\label{Prop 1.1}
Every probability measure $\p$ on $(S,\mathcal{S})$ is regular.
\end{prop}
\begin{proof}
Since $F^\epsilon\searrow F$ for closed $F$, $\p$ is regular on closed sets. Since the $\sigma$-algebra generated by all closed sets is $\mathcal{S}$, we only need to show that the collection of sets which $\p$ is regular on is a $\sigma$-algebra.
\end{proof}
This means that $\p$ is determined by the value $\p F$ for all closed $F$.
\begin{prop}[The Portmanteau Theorem]\label{Portmanteau Theorem}The following formulations are equivalent to $\p_n\xrightarrow{w}\p$.
\begin{enumerate}[label=\textup{(\roman*)}]
    \item $\p_nf\to\p f$ for all bounded and uniformly continuous $f:S\to\mathbb{R}$.
    \item $\limsup_{n\to\infty}\p_nF\leq\p F$ for all closed $F$.
    \item $\liminf_{n\to\infty}\p_nU\geq\p U$ for all open $U$.
    \item $\p_nA\to\p A$ for all $A$ that is $\p$-continuous: $\p(\partial A)=0$.
\end{enumerate}
\end{prop}
\begin{proof}
If $\p_n\to\p$, then by definition (i) holds.\smallbreak
(i) to (ii)/(iii): consider  bounded and uniformly continuous $f(x)=(1-\epsilon^{-1}\rho( x,F))^+$ such that $\1_F(x)\leq f(x)\leq\1_{F^\epsilon}(x)$ for closed $F$. Then $\limsup_{n\to\infty}\p_nF\leq\limsup_{n\to\infty}\p_nf=\p f\leq\p F^\epsilon$ for closed $F$. (iii) holds by taking complement.\smallbreak
(ii) \& (iii) to (iv): If $\p(\partial A)=0$ then $\overline{A}=A^\circ$, yet
\begin{align*}
\p\overline{A}&\geq\limsup_{n\to\infty}\p_n\overline{A}\geq\limsup_{n\to\infty}\p_nA\\
&\geq\liminf_{n\to\infty}\p_nA\geq\liminf_{n\to\infty}\p_nA^\circ\geq\p A^\circ.
\end{align*}
If (iv) holds, let $A_x=f^{-1}((-\infty,x])$. Then $\p(\partial A_x)=0$ since $\partial(-\infty,x]=\{x\}$. By Dominated Convergence Theorem
\[\p_nf=\int_{\R}\p_nA_x\,dx\to\int_{\R}\p A_x\,dx=\p f.\]
\end{proof}
We call a subclass $\mathcal{A}\subset\mathcal{S}$ a \textbf{separating class} if $\p=\p'$ on $\mathcal{A}$ implies $\p=\p'$ on $\mathcal{S}$. By Dynkin's Theorem $\mathcal{A}$ is a separating class if it is a $\pi$-system that generates $\mathcal{S}$. For example if $S=C[0,1]$ and $\mathcal{S}=\mathcal{C}$ with uniform metric $\norm{f}=\sup_x\abs{f(x)}$, then the class $\mathcal{C}_f\subset\mathcal{C}$ of all finite-dimensional sets $\pi_{t_1t_2\cdots t_k}^{-1}H$ is a separating class, where $H\in\mathcal{R}^k$ and $k\in\mathbb{N}$.
\[\begin{tikzcd}[row sep=large,column sep=large]
{(C[0,1],\mathcal{C})}\arrow[d,"\p"]\arrow[r,"\pi_{t_1t_2\cdots t_k}"]&{(\mathbb{R}^k,\mathcal{R}^k)}\arrow[ld,"\pi_{t_1t_2\cdots t_k\ast}\p"]\\
{[0,1]}&            
\end{tikzcd}\]
We also call a subclass $\mathcal{A}\subset\mathcal{S}$ a \textbf{convergence-determining class} if for every sequence $\{\p_n\}_{n\geq1}$, $\p_nA\to\p A$ for all $\p$-continuity sets $A\in\mathcal{A}$ implies $\p_n\xrightarrow{w}\p$. A convergence-determining class is obviously a separating class by \hyperref[Portmanteau Theorem]{Portmanteau Theorem}.
\begin{prop}
Let $\mathcal{P}$ be a $\pi$-system, then $\mathcal{P}$ is a separating class if
\begin{enumerate}[label=\textup{(\roman*)}]
    \item every open set is a countable union of $\mathcal{P}$-sets, or
    \item $S$ is separable and for every $x\in S$ and $\epsilon>0$ there exists $A\in\mathcal{P}$ such that $x\in A^\circ\subset A\subset B_\epsilon(x)$.
\end{enumerate}
If further that $\partial A$ contains either $\varnothing$ or uncountably many disjoint sets, then $\mathcal{P}$ is a convergence-determining class.
\end{prop}
If $h$ maps $(S,\mathcal{S},\p)$ to $(S',\mathcal{S}')$, then continuous $h$ induces $\p'=\p h^{-1}$ by $\p'(A)=\p(h^{-1}A)$. By continuity $\p_n\xrightarrow{w}\p$ implies $\p_n'\xrightarrow{w}\p'$.
\[\begin{tikzcd}
{(S,\mathcal{S})}\arrow[dr,"\p"']\arrow[rr,"h"]&&{(S',\mathcal{S}')}\arrow[dl,"\p'=\p h^{-1}"]\\
&{[0,1]}
\end{tikzcd}\]
In fact, the continuity can be weakened to almost continuity.
\begin{prop}[The Mapping Theorem]
If $\p_n\xrightarrow{w}\p$ and $\p D_h=0$ for $D_h$ the set of discontinuities, then $\p_n'\xrightarrow{w}\p'$.
\end{prop}
product space blahblahblah
\begin{prop}
product space
\end{prop}

\newpage
\section{Relationship to other convergence}

\section{Tightness}
\begin{defin}
$\p$ is tight if for each $\epsilon>0$ there exists a compact $K\in\mathcal{S}$ such that $\p K>1-\epsilon$.
\end{defin}
\begin{prop}
If $S$ is separable and complete, then every probability measure $\p$ on $(S,\mathcal{S})$ is tight.
\end{prop}
\begin{proof}
We need a topological lemma: totally bounded set in a complete metric space is compact.\smallbreak
Since $S$ is separable, for each $k$ we are able to choose $n_k$ such that $\p(\cup_{i\leq n_k}B_{k_i})>1-2^{-k}\epsilon$ for balls $B_{k_i}$ with radii $k^{-1}$. Then $K=\cap_{k\geq1}\cup_{i\leq n_k}B_{k_i}$ is totally bounded. Hence it is compact and $\p(K)>1-\epsilon$.
\end{proof}



\end{document}