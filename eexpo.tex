\documentclass[psamsfonts]{amsart}

\usepackage{amssymb,amsfonts}
\usepackage[all,arc]{xy}
\usepackage{enumerate}
\usepackage[shortlabels]{enumitem}
\usepackage{siunitx}
\usepackage{physics}
\usepackage{tikz-cd}
\usetikzlibrary{decorations.pathmorphing}

\newtheorem{thm}{Theorem}[section]
\newtheorem*{thm*}{Theorem}
\newtheorem{cor}[thm]{Corollary}
\newtheorem{prop}[thm]{Proposition}
\newtheorem{lem}[thm]{Lemma}
\newtheorem{conj}[thm]{Conjecture}

\theoremstyle{definition}
\newtheorem{defn}[thm]{Definition}
\newtheorem{defns}[thm]{Definitions}
\newtheorem{exmp}[thm]{Example}
\newtheorem{exmps}[thm]{Examples}
\newtheorem{notn}[thm]{Notation}
\newtheorem{notns}[thm]{Notations}
\newtheorem{addm}[thm]{Addendum}
\newtheorem{exer}[thm]{Exercise}

\theoremstyle{remark}
\newtheorem{rem}[thm]{Remark}
\newtheorem{rems}[thm]{Remarks}
\newtheorem{warn}[thm]{Warning}

\makeatletter
\newcommand{\lim@}[2]{%
  \vtop{\m@th\ialign{##\cr
    \hfil$#1\operator@font lim$\hfil\cr
    \noalign{\nointerlineskip\kern1.5\ex@}#2\cr
    \noalign{\nointerlineskip\kern-\ex@}\cr}}%
}
\newcommand{\filtlim}{%
  \mathop{\mathpalette\lim@{\rightarrowfill@\textstyle}}\nmlimits@
}
\newcommand{\inverselim}{%
  \mathop{\mathpalette\lim@{\leftarrowfill@\textstyle}}\nmlimits@
}
\makeatother

\newcommand{\Q}{\mathbb{Q}}
\newcommand{\s}{\mathbf{S}}
\newcommand{\Z}{\mathbb{Z}}
\newcommand{\W}{\mathrm{W}}
\newcommand{\Ab}{\mathbf{Ab}}
\newcommand{\Hom}{\textup{Hom}}
\newcommand{\CDGA}{\mathbf{CDGA}}
\newcommand{\cros}{\mathrm{cr}}
\newcommand{\CDGcoA}{\mathbf{CDGcoA}}
\newcommand{\DGLie}{\mathbf{DGLie}}
\newcommand{\simp}{\mathbf{\Delta}}
\newcommand{\Ho}{\mathbf{Ho}}
\newcommand{\Sp}{\mathbf{Sp}}
\newcommand{\N}{\mathbb{N}}
\newcommand{\Id}{\mathrm{Id}}
\newcommand{\C}{\mathrm{C}}
\newcommand{\Top}{\mathbf{Top}}
\newcommand{\ch}{\textup{Ch}}
\newcommand{\F}{\mathbb{F}}
\newcommand{\Cof}{\mathrm{Cof}}
\newcommand{\Fib}{\mathrm{Fib}}
\newcommand{\dgcAlg}{\mathbf{dgcAlg}}
\newcommand{\cof}{\mathrm{cof}}
\newcommand{\fib}{\mathrm{fib}}
\newcommand{\Obj}{\textup{Obj}}
\newcommand{\id}{\mathrm{id}}
\newcommand{\sSet}{\mathbf{sSet}}

\DeclareMathOperator{\coker}{coker}
\DeclareMathOperator{\Sym}{Sym}
\DeclareMathOperator{\Etr}{Etr}
\DeclareMathOperator*\colim{colim}
\DeclareMathOperator{\hocolim}{hocolim}
\DeclareMathOperator{\holim}{holim}
\DeclareMathOperator{\Char}{char}
\DeclareMathOperator{\im}{im}
\DeclareMathOperator{\Tor}{Tor}
\DeclareMathOperator{\B}{B}

\makeatletter
\let\c@equation\c@thm
\makeatother
\numberwithin{equation}{section}

\bibliographystyle{plain}

\title{On Realizing Rational and Polynomial Cohomology Rings}

\author{Yunhan (Alex) Sheng}

\begin{document}

\begin{abstract}

This expository paper written during the $\textrm{\smash{2020 REU}}$ program at the University of Chicago focuses on the problem: which commutative graded $R$-algebras occur as a cohomology ring $\smash{H^\ast(X;R)}$ of a space $X$ with coefficients in $R$? Rational homotopy theory developed by Sullivan and Quillen gives a solution for $R=\Q$.
For polynomial algebras over $\Z$ and $\F_p$ for prime $p$ the problem was solved by Andersen-Grodal \cite{Andersen-Grodal}, based on the works of many predecessors. The problem for the general case remains open.

\end{abstract}

\maketitle

\tableofcontents

\section{Introduction}

Some historical information
%Homotopy groups $\pi_n(X)$ of a $0$-connected space $X$ are abelian for $n\geq2$, as are its homology groups $H_n(X)$. Consider the corresponding torsion-free versions $\pi_n(X)\otimes\Q$ and $H_n(X;\Q)$. It turns out that a space $X$ has a rationalization $X_\Q$ satisfying $\pi_n(X_\Q)=\pi_n(X)\otimes\Q$, and two spaces $X$ and $Y$ are rationally homotopy equivalent (denoted $\smash{X\sim_\Q Y}$) if $X_\Q\sim Y_\Q$.

\newpage
\section{The Case $R=\Q$}

In this section we prove, when restricting to rational, $1$-connected objects of finite type, an equivalence of homotopy theories of $\Top_{\Q,1,f}$ and $\CDGA_{\Q,1,f}$. Therefore, any $1$-connected CDGA$_\Q$ of finite type can occur as a rational cohomology ring $H^\ast(X;\Q)$ of some space $X$.

To accomplish the goal we first describe appropriate model structures on $\Top$, $\sSet$, and $\CDGA_k$. We then construct adjunctions between them:
\[\begin{tikzcd}
	{\Top} & {\sSet} & {\CDGA_k}
	\arrow[""{name=0, anchor=center, inner sep=0},"\mathcal{S}_\bullet",shift left=1.3, from=1-1, to=1-2]
	\arrow[""{name=1, anchor=center, inner sep=0},"\abs{-}",shift left=1.4, from=1-2, to=1-1]
	\arrow["\dashv"{anchor=center, rotate=-90}, draw=none, from=0, to=1]
	\arrow[""{name=0, anchor=center, inner sep=0},"\Omega_\textrm{PLdR}", shift left=1.3, from=1-2, to=1-3]
	\arrow[""{name=1, anchor=center, inner sep=0},"\mathcal{K}_\bullet",shift left=1.4, from=1-3, to=1-2]
	\arrow["\dashv"{anchor=center, rotate=-90}, draw=none, from=0, to=1]
\end{tikzcd}.\]
which preserve model structures and induces Quillen equivalences.

\subsection{Rational homotopy and rationalization via localization}

In this section we lay down some preparatory work necessary for later stages.

\begin{defns}
A collection of abelian groups $\mathfrak{C}$ is a \textbf{Serre class} if for a short exact sequence of abelian groups
\[0\to A\to B\to C\to0,\] $A\in\mathfrak{C}$ and $C\in\mathfrak{C}$ implies $B\in\mathfrak{C}$. A map $f\in\Hom_{\Ab}(A,B)$ is a \textbf{$\mathfrak{C}$-isomorphism} if both $\ker f\in\mathfrak{C}$ and $\coker f\in\mathfrak{C}$.
\end{defns}

One can think of a Serre class as the class of groups we wish to kill. In particular, the Serre class of torsion abelian groups is of our interest.

\begin{prop}[Serre-Whitehead]
Let $\mathfrak{C}$ be a Serre class and $f:X\to Y$ be a map between $1$-connected spaces. Then $\pi_i(f):\pi_i(X)\to\pi_i(Y)$ is a $\mathfrak{C}$-isomorphism if and only if $H_i(f):H_i(X)\to H_i(Y)$ is a $\mathfrak{C}$-isomorphism for $i<n$.
\end{prop}

A proof using the Serre spectral sequence is given in Berglund\cite{Berglund}.

\begin{defns}
A map $f:X\to Y$ between $1$-connected spaces is a \textbf{rational homotopy equivalence} if $\pi_i(f)\otimes\Q$ is an isomorphism for all $i$. A space $X$ is \textbf{rational} if $\pi_i(X)$ is uniquely divisible for all $i$.
\end{defns}

Note that $\pi_i(f)\otimes\Q$ being an isomorphisms is equivalent to $\pi_i(f)$ being a ${\mathfrak{C}\textrm{-isomorphism}}$, taking $\mathfrak{C}$ to be the class of torsion abelian groups

\begin{defn}
A \textbf{rationalization} of a space $X$ is a rational space $X_\Q$ and a rational homotopy equivalence $r:X\to X_\Q$.
\end{defn}

Rationalization can be formulated via localization.

\begin{defns}
Let $\mathcal{C}$ be a category and $\mathrm{L}$ a subclass of morphisms in $\mathcal{C}$. An object $A$ of $\mathcal{C}$ is \textbf{$\mathrm{L}$-local} if every $f\in\Hom_\mathcal{C}(X,Y)$ with $f\in\mathrm{L}$ induces a bijection $f^\ast:\Hom_\mathcal{C}(Y,A)\to\Hom_\mathcal{C}(X,A)$. A map $\ell:A\to A_\mathrm{L}$ is a \textbf{$\mathrm{L}$-localization} if $\ell\in\mathrm{L}$ and $A_\mathrm{L}$ is $\mathrm{L}$-local.
\[\begin{tikzcd}
Y\arrow[dr]\arrow[rr,"f^\ast"]&&X\arrow[dl]\\
&A
\end{tikzcd}\]
\end{defns}

Take the category of $1$-connected topological spaces $\Top_1$ and let $\mathrm{Q}$ be the class of rational homotopy equivalences in $\Top_1$.

\begin{prop}
An object of $\Top_1$ is rational if and only if it is $\mathrm{Q}$-local. Hence a map is a rationalization if and only if it is a $\mathrm{Q}$-localization.
\end{prop}
\begin{proof}
\textcolor{red}{wef}
\end{proof}

\begin{cor}
Every $1$-connected space $X$ admits a rationalization.
\end{cor}

\subsection{Equivalence between homotopy categories of $\Top$ and $\sSet$} We construct the pair of adjoint functors between $\Top$ and $\sSet$ which induces Quillen equivalence.\medbreak

Let $\mathbf{\Delta}$ be the category with order sets $[n]=\{0,1,\cdots,n\}$ as objects and functions $\varphi:[m]\to[n]$ with $\varphi(i)\leq\varphi(i+1)$ for all $i\in[m]$ as morphisms. Each $\varphi$ is composed with unique face maps $\partial_i:[n-1]\to[n]$ and degeneracy maps $s_i:[n+1]\to[n]$.

There is a covariant functor $\Phi:\mathbf{\Delta}\to\Top$ sending $[n]\to\Delta^n$ and $\varphi:[m]\mapsto[n]$ to $\varphi_\ast:\Delta^m\to\Delta^n$ where $\Delta^m$ and $\Delta^n$ are the standard topological simplices.

\begin{defn}
A \textbf{simplicial set} is a contravariant functor $\Delta:\mathbf{\Delta}\to\mathbf{Set}$
\end{defn}

Let $\sSet$ be the category with simplicial sets as objects and natural transformations between simplicial sets as morphisms.\medbreak

Define the singular complex functor $\mathcal{S}_\bullet:\Top\to\sSet$ by
\begin{align*}
\mathcal{S}_n:&X\mapsto\Hom_{\Top}(\Delta^n,X),&&X\in\Top\\
&f\mapsto f\psi,&&f\in\Hom_{\Top}(X,Y),\ \psi\in\Hom_{\Top}(\Delta^n,X)
\end{align*}
This functor carries simplicial data as follows:
\[\begin{tikzcd}
{[m]}\arrow[d,"\varphi"]\arrow[r]&\Delta^m\arrow[r]\arrow[d,"\varphi_\ast"]&\mathcal{S}_m(X)=\Hom_{\Top}(\Delta^m,X)\\
{[n]}\arrow[r]&\Delta^n\arrow[r]&\mathcal{S}_n(X)=\Hom_{\Top}(\Delta^n,X)\arrow[u,"\varphi^\ast"']
\end{tikzcd}\]
where $\varphi^\ast$ is given by $\varphi^\ast(\psi)=\psi\varphi_\ast$. Conversely, define the geometric realization functor $\abs{-}:\sSet\to\Top$ to be the Kan extension of $\Phi$ along $\Delta$.
\[\begin{tikzcd}
\mathbf{\Delta}\arrow[dr,"\Delta"']\arrow[rr,"\Phi"]&&\Top\\
&\sSet\arrow[ur,"\abs{-}"']
\end{tikzcd}\]
Explicitly, $\abs{-}$ is defined by $X\mapsto\Hom_{\sSet}(X,\Phi)$ or equivalently by Yoneda lemma $X\mapsto\lim_{\Delta[n]\to X}\Phi[n]$. Written in this form, the adjunction is immediate:
\begin{align*}
\Hom_{\Top}(Y,\abs{X})&\cong\Hom_{\Top}(Y,\textstyle{\lim_{\Delta[n]\to X}\Phi[n]})\quad\textrm{(Yoneda lemma)}\\
&\cong\colim_{\Delta[n]\to X}\mathcal{S}_n(Y)\\
&\cong\colim_{\Delta[n]\to X}\Hom_{\sSet}(\mathcal{S}_\bullet(Y),\Delta[n])\quad\textrm{(Yoneda lemma)}\\
&\cong\Hom_{\sSet}(\mathcal{S}_\bullet(Y),X).
\end{align*}
We remark that a more geometric definition of $\abs{-}$ is available:
\[\abs{-}:X\mapsto\coprod_{n\geq0}(X[n]\times\Delta^n)/_{\sim}\]
where we identify $(\varphi^\ast(x),t)\sim(x,\varphi_\ast(t))$. One associates an $n$-simplex to each $X[n]$ and they are glued together by the aforementioned identification.\medbreak

Now we discuss these functors on model-categorical level.

\begin{prop}
The following defines a model category structure on $\Top$:
\begin{enumerate}[(i)]
    \item weak homotopy equivalences as weak equivalences;
    \item Serre fibrations as fibrations;
    \item the set of maps $I=\{S^{n-1}\hookrightarrow D^n:n\geq0\}$ generating cofibrations.
\end{enumerate}
\end{prop}

\begin{prop}
The following defines a model category structure on $\sSet$:
\begin{enumerate}[(i)]
    \item morphisms whose geometric realizations induces weak homotopy equivalences as weak equivalences;
    \item Kan fibrations as fibrations;
    \item degreewise inclusions as cofibrations.
\end{enumerate}
\end{prop}

The detailed proof of the above propositions is in the Appendix.

\begin{defns}
Let $\mathbf{C}$ and $\mathbf{D}$ be model categories. An adjunction $\mathcal{F}\dashv\mathcal{G}$ is a \textbf{Quillen adjunction} if it satisfy the following equivalent conditions:\begin{enumerate}[(i)]
    \item $\mathcal{F}$ preserves cofibrations and acyclic cofibrations;
    \item $\mathcal{G}$ preserves fibrations and acyclic fibrations.
\end{enumerate}
Moreover, $\mathcal{F}\dashv\mathcal{G}$ induces \textbf{Quillen equivalence} if $X^{\textrm{cof}}\to\mathcal{G}(Y^{\textrm{fib}})$ is a weak equivalence in $\mathbf{C}$ if and only if $\mathcal{F}(X^{\textrm{cof}})\to Y^{\textrm{fib}}$ is a weak equivalence in $\mathbf{D}$ for every cofibrant $X^{\textrm{cof}}\in\mathbf{C}$ and fibrant $Y^{\textrm{fib}}\in\mathbf{D}$.
\end{defns}

If $\mathcal{F}\dashv\mathcal{G}$ is a Quillen equivalence, then the derived functors $\mathbf{L}\mathcal{F}\dashv\mathbf{R}\mathcal{G}$ induces an equivalence between homotopy categories $\Ho(\mathbf{C})$ and $\Ho(\mathbf{D})$, where homotopy categories are obtained by localizing weak equivalences.

\begin{prop}
The singular complex functor $\mathcal{S}_\bullet$ and the geometric realization functor $\abs{-}$ induce a Quillen equivalence between $\Top$ and $\sSet$.
\end{prop}
\begin{proof}
We first prove the Quillen adjunction, then the Quillen equivalence.\medbreak

\textit{Quillen adjunction.} Since both $\Top$ and $\sSet$ are cofibrantly generated, by Proposition A.88 we only need to check that $\abs{i}\in\Cof_{\Top}$ for $i\in I_{\sSet}$ and $\abs{j}\in\Cof_{\Top}\cap\W_{\Top}$ for $j\in J_{\sSet}$. Both are immediate since $\abs{\partial\Delta[n]}=S^{n-1}$ and $\abs{\Delta[n]}=D^n$.\medbreak

\textit{Quillen equivalence.} By Proposition A.87, we only need to check the following:\begin{enumerate}[(i)]
    \item if $\abs{f}\in\W_{\Top}$ for cofibrant objects in $\Top$, then $f\in\W_{\sSet}$ for cofibrant objects in $\sSet$;
    \item the map $\smash{\abs{\mathcal{S}_\bullet(Y)^{\cof}}\to Y}$ is a weak equivalence in $\Top$ for $Y\in\Fib_\Top$
\end{enumerate}
(i) is immediate by definition of $\W_{\sSet}$. 
\end{proof}

\subsection{The $q$-model structure on $\CDGA_k$}

In this section we describe the $q$-model structure on $\CDGA_\Q$ and introduce the Sullivan minimal model.

\begin{defns}
A \textbf{differential graded module} over $R$ (or DGM$_R$) $M$ is a $R$-module $M$ together with a differential $d$ of grading degree $+1$ satisfying $d^2=0$. Let $N$ be a DGM$_R$, a map $f:M\to N$ is called a \textbf{chain map} if $d_Nf=fd_M$.
\end{defns}

Let $\mathbf{DGM}_R$ be the category with DGM$_R$ as objects and chain maps as morphisms. Then $\mathbf{DGM}_R$ is a symmetric monoidal category with graded tensor product $M\otimes_RN$ and symmetry given by $a\otimes b=(-1)^{\abs{a}\abs{b}}b\otimes a$. In addition, the Leibniz rule $\smash{d(a\otimes b)=da\otimes b+(-1)^{\abs{a}}a\otimes db}$ satisfied.

\begin{defn}
A \textbf{commutative differential graded algebra} over $R$ (or CDGA$_R$) $A$ is a commutative monoid in $\mathbf{DGM}_R$.
\end{defn}

Explicitly, $A$ is an object of $\mathbf{DGM}_R$ together with a unit $\eta:R\to A$ and an associative product $\mu:A\otimes_RA\to R$ given by $a\otimes b\to a\cdot b$. The commutativity comes from the symmetric braiding $\smash{a\cdot b=(-1)^{\abs{a}\abs{b}}b\cdot a}$.
\[\begin{tikzcd}[row sep=0.8em]
	{(A\otimes_RA)\otimes_RA} & {A\otimes_RA} \\
	&& A \\
	{A\otimes_R(A\otimes_RA)} & {A\otimes_RA}
	\arrow["\alpha"', from=1-1, to=3-1]
	\arrow["{\mu\otimes_R\id}"', from=1-1, to=1-2]
	\arrow["{\id\otimes_R\mu}", from=3-1, to=3-2]
	\arrow["\mu"', from=3-2, to=2-3]
	\arrow["\mu", from=1-2, to=2-3]
\end{tikzcd}\quad
\begin{tikzcd}[column sep=1.25em]
	& {M\otimes_RM} \\
	{R\otimes_RM} & M & {M\otimes_RR}
	\arrow["{\eta\otimes_R\id}", from=2-1, to=1-2]
	\arrow["{\id\otimes_R\eta}"', from=2-3, to=1-2]
	\arrow["\ell"', from=2-1, to=2-2]
	\arrow["r", from=2-3, to=2-2]
	\arrow["\mu", from=1-2, to=2-2]
\end{tikzcd}\]

Let $A'$ be a CDGA$_R$, a chain map $f:A\to A'$ respects the units $f\eta=\eta'$ and products $f\mu=\mu'(f\otimes f)$. This defines a category $\CDGA_R$.\medbreak

From now on we take $k$ to be a field with characteristic $0$.

\begin{prop}
The following defines a model structure on $\CDGA_k$:\begin{enumerate}[(i)]
    \item quasi-isomorphisms as weak equivalences;
    \item degreewise surjections as fibrations;
    \item the naturally determined $(\Fib\cap\W)^r$ as cofibrations.
\end{enumerate}
\end{prop}

Before proving the theorem, we need CDGA$_k$ models that generalizes the notion cochain complexes of spheres and disks.

The cochain complex $S(n)$ of $S^n$ is the CDGA$_k$ $(\Lambda x,dx=0)$ with $\abs{x}=n$. The cochain complex $D(n-1)$ of $D^{n-1}$ is the CDGA$_k$ $(\Lambda(x,y),dx=0,dy=x)$ with $\abs{x}=n$ and $\abs{y}=n-1$. Then for any CDGA$_k$ $A$, the following holds:\begin{enumerate}[(i)]
    \item $S(n)\hookrightarrow D(n-1)$ is a natural inclusion;
    \item $\Hom_{\CDGA_k}(S(n),A)\cong Z^n(A)$;
    \item $\Hom_{\CDGA_k}(D(n-1),A)\cong A^{n-1}$.
\end{enumerate}

For any graded vector space $V$, define the CDGA$_k$ $S(V)=(\Lambda V,dv=0)$ for $v\in V$, and the CDGA$_k$ $D(V)=(\Lambda(V\oplus sV),dv=0,d(sv)=v)$ for $v\in V$. Then similarly the following holds for any CDGA$_k$ $A$:\begin{enumerate}[(i)]
    \item $S(V)\hookrightarrow D(V)$ is a natural inclusion;
    \item $\Hom_{\CDGA_k}(S(V),A)\cong\Hom_k(V,Z(A))$;
    \item $\Hom_{\CDGA_k}(D(V),A)\cong\Hom_k(V,A)$.
\end{enumerate}

\begin{lem}
For any graded vector spaces $V$ and $V'$ and any cdg-algebra $A$, the morphism $i:A\to A\otimes_kS(V)\otimes_kD(V')$ given by $a\mapsto a\otimes\eta\otimes\eta$ is a cofibration.
\end{lem}
\begin{proof}
The morphism $j:A\to A\otimes_kD(V)$ given by $a\mapsto a\otimes\eta$ is a weak equivalence. The inclusion $S(V)\hookleftarrow D(V)$ is a cofibration.
\end{proof}

\begin{proof}[Proof of Proposition 2.15]
We choose to follow the proof in \cite{Gelfand-Manin}.\medbreak

\textit{Limit Axiom.} We show that pushout and pullback exist in $\CDGA_k$. It's standard to verify that $\{(b,c):f(b)=g(c)\}$ is the pullback of $\smash{B\xrightarrow{f}A\xleftarrow{g}C}$. For pushout of $\smash{B\xleftarrow{f}A\xrightarrow{g}C}$, consider $B\otimes_AC$ together with inclusions $\eta\otimes\id:C\to B\otimes_AC$ and $\id\otimes\eta:B\to B\otimes_AC$ where $\eta$ is the unit.\medbreak

\textit{Factorization Axiom.} Let $f:A\to B$ be a morphism in $\CDGA_k$. We show that $\Fib\circ(\Cof\cap\W)=f=(\Fib\cap\W)\circ\Cof$.

For the second equality consider the base case
\[A\xrightarrow{j_0}C_0=A\otimes_kS(Z(B))\otimes_kD(B)\xrightarrow{q_0}B\]
where $j_0=\id\otimes\eta\otimes\eta$ and $q_o:a\otimes z\otimes b\mapsto f(a)\alpha(z)\beta(b)$ and $\alpha$ corresponds to $\id:Z(B)\to Z(B)$ by previous discussion. Then $j_0\in\Cof$ by Lemma 2.16 and $q_0\in\Fib$ by definition. But instead of $q_0\in\W$, we can only guarantee that $q_0$ is surjective on cohomology.

To fix this issue, we add generators to $C_n$ to kill cocycles in $C_n$. Suppose $\smash{A\xrightarrow{j_n}C_n\xrightarrow{q_n}B}$ has already been constructed. Let
\[V_n=\{(c,b)\in C_n\oplus B:dc=0\textrm{ and }q_n(c)=db\}\textrm{ (with grading }\abs{(c,b)}=\abs{c})\]
contain the information we want to erase. We add generators to $C_n$ to construct $C_{n+1}$ in analogy to adding generators to $S(V_n)$ to construct $D(V_n)$. Formally, we construct $C_{n+1}$ as the pushout
\[\begin{tikzcd}[column sep=large,row sep=large]
S(V_n)\arrow[d,hookrightarrow]\arrow{r}{\alpha'}&C_n\arrow{d}{i_n}\arrow[ddr,bend left,"q_n"]&A\arrow[l,"j_n"']\\
D(V_n)\arrow[drr,bend right,"\beta'"]\arrow[r]&C_{n+1}\arrow[ul,phantom,"\ulcorner",very near start]\arrow[dr,dashed,"q_{n+1}"']\\
&&B
\end{tikzcd}\]
where $\alpha'$ is the composition $S(V_n)\to S(V_n')\to C_n$ and $V_n'=\im(V_n\to C_n)$. Let $j_{n+1}=i_nj_n:A\to C_{n+1}$. The map $q_{n+1}:C_{n+1}\to B$ comes from the universality of pushout. The map $\beta'$ is the composition $D(V_n)\to D(V_n'')\to B$ where $V_n''=\im(V_n\to B)$. We have thus constructed $\smash{A\xrightarrow{j_{n+1}}C_{n+1}\xrightarrow{q_{n+1}}B}$.

Finally let $C=\lim_n C_n$ and $j=\filtlim j_n:A\to C$. By Lemma 2.16 $\lambda\in\Cof$, and by xxx $i_n\in\Cof$. Hence $j_n\in\Cof$ for all $n$. Similarly let $q=\inverselim q_n:C\to B$, then $q$ is surjective on cohomology. By our construction $q$ is also injective on cohomology: any kernel in $H^\cdot(q_n)$ is becomes a boundary in $C_{n+1}$, and hence is trivial in $C$.
\medbreak

\textit{Determination Axiom.} We forced $\Cof=(\Fib\cap\W)^r$ in the construction. It remains to show that $\smash{\Fib=(\Cof\cap\W)^l}$ and $\smash{\W=\Fib^r\circ\Cof^l}$. Let $f:A\to B$ and factorize $f$ to $\smash{A\xrightarrow{i}C\xrightarrow{p}B}$ where $i\in\Cof\cap\W$ and $p\in\Fib$. If $f\in(\Cof\cap\W)^r$, then there exists a lift $h:C\to A$ such that $p=fh$. Since $p$ is surjective, $f$ is surjective and $f\in\Fib$. Hence $(\Cof\cap\W)^r\subseteq\Fib$.

For the reverse inclusion, factorize $j:A\to B$ through $\smash{A\xrightarrow{i}A\otimes_kD(B)\xrightarrow{p}B}$. If $j\in\Cof\cap\W$, then for any surjection $f:C\to D$ we construct a lift $h:B\to C$ as follows:
\[\begin{tikzcd}[column sep=huge, row sep=scriptsize]
A\arrow[d,"i"']\arrow[r,"h"]&C\arrow[dd,"f"]\\
A\otimes_kD(B)\arrow[ur,"k", near start]\arrow[d,"p"]\\
B\arrow[uur,"\ell"']\arrow[u,bend left,"q"]\arrow[uur,dashed,bend right,"h"']\arrow[r,"g"']&D
\end{tikzcd}\]
Since $f$ is surjective, there exists a degreewise map $\ell:B\to C$ such that $g=f\ell$. Then $k:A\otimes_kD(B)\to C$ is determined by restricting it to $h$ and $\ell$ on $A$ and $B$ respectively. The map $q:B\to A\otimes_kD(B)$ is constructed as the lift in the diagram
\[\begin{tikzcd}
A\arrow[d,"j"]\arrow[r,"i"]&A\otimes_kD(B)\arrow[d,"p"]\\
B\arrow[ur,dashed]\arrow[r,equal]&B
\end{tikzcd}\]
We define $h=kq$. It's standard to check that $h$ is indeed the lift required. Hence $f\in(\Cof\cap\W)^r$ and $\Fib\subseteq(\Cof\cap\W)^r$.

The inclusion $\smash{\W\subseteq\Fib^r\circ\Cof^l}$ is immediate. For the converse, we need to prove that $\smash{\Cof^r\subseteq\W}$ and $\smash{\Fib^l\subseteq\W}$. Factorize $f:A\to B$ through $\smash{A\xrightarrow{i}C\xrightarrow{p}B}$ where $i\in\Cof$ and $p\in\Fib\cap\W$. If $f\in\Cof^r$, then it follows that $f\in\W$. The other inclusion is proven similarly.\medbreak

\textit{2-out-of-3 Axiom.} All isomorphisms of cdg-algebras are quasi-isomorphisms. The 2-out-of-3 property holds for quasi-isomorphisms.
\end{proof}

We want to be able to describe the cofibrations in $\CDGA_k$ explicitly, instead of defining it
to satisfy the axioms of model category. These cofibrations are so-called relative Sullivan algebras.

\begin{defns}
A \textbf{relative Sullivan algebra} is an inclusion $A\hookrightarrow A\otimes_k\Lambda V$ of cdga$_k$ where $V=\{V^i\}_{i\geq1}$ is a DGM$_k$ with filtration
\[0=V(-1)\subseteq V(0)\subseteq V(1)\subseteq\cdots\]
satisfying $V=\cup_{j\geq0}V(j)$ and $dV(n)\subseteq A\otimes_k\Lambda V(n-1)$ for all $n\geq0$.

A map $f:A\otimes_k\Lambda V\to A'\otimes_k\Lambda V'$ is a morphism of relative Sullivan algebras if it's restriction to $A$ and $A'$ is a morphism of cdga$_k$s:
\[\begin{tikzcd}
A\arrow[d,hookrightarrow]\arrow[r,"f|_A"]&A'\arrow[d,hookrightarrow]\\
A\otimes_k\Lambda V\arrow[r,"f"]&A'\otimes_k\Lambda'
\end{tikzcd}.\]

A \textbf{Sullivan algebra} is a relative Sullivan algebra with $A=k$. A relative Sullivan algebra is \textbf{minimal} if $\im d\subseteq\Lambda^{\geq2}V$, subspace of words of length at least two.
\end{defns}

Relative Sullivan algebras can also be built inductively similar to CW complexes by gluing disks algebras along sphere algebras. In the proof of factorization axiom in Proposition 2.15 this equivalent to adding generators to kill cocycles: if $A$ is an differential graded module over $k$ with $x\in Z^n(A)$, then $A\otimes_k\Lambda V$ with $V=\{y\}$ and $dy=x$ is the pushout of $D(V)\leftarrow S(V)\rightarrow A$.

A relative Sullivan algebra is an inclusion $A\hookrightarrow X$ of differential graded algebras over $k$ with a filtration
\[A=X(0)\subseteq X(1)\subseteq\cdots\]
satisfying $X=\cup_{j\geq0}X(j)$, where each $X(j)$ is obtained as the pushout of $D(V^j)\leftarrow S(V^j)\rightarrow X(j-1)$ for some differential graded module $V=\{V^i\}_{i\geq1}$.

\begin{prop}
Relative Sullivan algebras are cofibrations and Sullivan algebras are cofibrant in the $q$-model structure of $\CDGA_k$.
\end{prop}

\begin{defn}
Let $f\in\Hom_{\CDGA_k}(A,B)$. A \textbf{Sullivan model} for $f$ is a relative Sullivan algebra $A\hookrightarrow M$ together with a quasi-isomorphism $M\sim B$.
\[\begin{tikzcd}
A\arrow[dr,hookrightarrow]\arrow[rr,"f"]&&B\\
&M\arrow[ur,"\sim"']
\end{tikzcd}\]
A Sullivan model is \textbf{minimal} if $A$
\end{defn}

Minimal Sullivan model plays.

\begin{prop}
Every $f\in\Hom_{\CDGA_k,1}(A,B)$ admits a minimal Sullivan model which is unique up to isomorphism.
\[\begin{tikzcd}[row sep=large]
	A && B \\
	M && {M'}
	\arrow["f","\cong"', from=1-1, to=1-3]
	\arrow["\sim" near start, from=2-1, to=1-3]
	\arrow["\sim"' near start, from=2-3, to=1-1]
	\arrow["\cong"',dashed, from=2-1, to=2-3]
	\arrow[from=1-1,hookrightarrow, to=2-1]
	\arrow[from=1-3,hookrightarrow, to=2-3]
\end{tikzcd}\]
\end{prop}

\newpage
\subsection{Equivalence between homotopy categories of $\sSet$ and $\CDGA_\Q$}

In this section we prove the main equivalence. We begin by constructing a pair of adjoint functors between $\sSet$ and $\CDGA_\Q$.

\begin{defn}
The \textbf{polynomial differential form} $\Omega^\bullet_\textrm{poly}$ is the simplicial object in $\CDGA_k$ given by
\begin{align*}
\mathbf{\Delta}^\textrm{op}&\to\CDGA_k\\
\Delta[n]&\mapsto\Omega^n_\textrm{poly}=\Lambda(t_0,\cdots,t_n,dt_0,\cdots,dt_n)/\left(\sum_{i=0}^nt_i=1,\sum_{i=0}^ndt_i=0\right)
\end{align*}
where $\abs{t_i}=0$ and $\abs{dt_i}=1$.
\end{defn}

The superscript $^\bullet$ encodes the simplicial information, which is specified by face maps $\smash{\delta_i:\Omega^n_\textrm{poly}\to\Omega^{n-1}_\textrm{poly}}$ and degeneracy maps $\smash{\partial_i:\Omega^n_\textrm{poly}\to\Omega^{n+1}_\textrm{poly}}$ below
\[\delta_i:t_k\mapsto\begin{cases}
t_k,&k<i\\0,&k=i\\t_{k-1},&k>i
\end{cases},\quad s_i:t_k\mapsto\begin{cases}
t_k,&k<i\\t_k+t_{k+1},&k=i\\t_{k+1},&k>i
\end{cases}.\]
Apart form the simplicial structure, $\Omega^\bullet_\textrm{poly}$ as also carries a grading structure. We denote the $k$-module of homogeneous elements of degree $p$ by $\Omega^\bullet_{\textrm{poly},p}$.

\begin{lem}
For any $p\geq0$, $\Omega^\bullet_{\mathrm{poly},p}$ is a trivial Kan complex.
\end{lem}
\begin{proof}
For any $p\geq0$, $\Omega^\bullet_{\mathrm{poly},p}$ is a simplicial $k$-module, and thus a Kan complex. By a theorem of Eilenberg-Mac Lane, the homotopy groups of $\Omega^\bullet_{\textrm{poly},p}$ corresponds to the homology groups of the alternating face map complex
\[\begin{tikzcd}
\cdots\arrow[r]&\Omega^2_{\textrm{poly},p}\arrow[r,"\partial_2"]&\Omega^1_{\textrm{poly},p}\arrow[r,"\partial_1"]&\Omega^0_{\textrm{poly},p}\end{tikzcd}\]
with differential given by $\partial_n=\sum_{i=1}^n(-1)^i\delta_i$. Define shifting chain maps $\{s_n:\Omega^n_{\textrm{poly},p}\to\Omega^{n+1}_{\textrm{poly},p}\}_{n}$ given by $s_n(1)=(1-t_0)^2$ and $s_n(t_i)=(1-t_0)t_{i+1}$ for $0\leq i\leq n$. It's standard to verify that $s_n$ is a degeneracy map, Goerss-Jardin 201. 
\end{proof}

Now we can extend $\Omega^\bullet_\textrm{poly}:\mathbf{\Delta}^\textrm{op}\to\dgcAlg_k$ along $\Delta:\mathbf{\Delta}^\textrm{op}\to\sSet$ via left Kan extension, which establishes a functor $\sSet\to\dgcAlg_k$.
\[\begin{tikzcd}
\mathbf{\Delta}^\textrm{op}\arrow[swap]{dr}{\Delta[-]}\arrow{rr}{\Omega^\bullet_\textrm{poly}}&&\dgcAlg_k\\
&\sSet\arrow[swap,near start,dashed]{ur}
\end{tikzcd}\]

\begin{defn}
The \textbf{piecewise-linear de Rham complex} $\Omega^\bullet_\textrm{PLdR}$ is the hom-functor specified by
\begin{align*}
\sSet&\to\dgcAlg_k^\mathrm{op}\\
S&\mapsto\Hom_\sSet(S,\Omega^\bullet_\textrm{poly}).
\end{align*}
\end{defn}

The density theorem in category theory gives a limit construction of $\smash{\Omega^\bullet_\textrm{PLdR}}$:
\[\smash{\Hom_\sSet(S,\Omega^\bullet_\textrm{poly})=\lim_{\Delta[n]\to S}\Omega^\bullet_\textrm{poly}}.\]\bigbreak

The classical de Rham theorem asserts that the Stokes' correspondence between de Rham complex and singular cochcain complex is an equivalence on the cohomology level. The piecewise-linear analog is stated below.

Given a simplex $x\in S$ and a polynomial $k$-form $f\in\Omega^k_\textrm{PLdR}(S)$, there is an integration map $\oint:\Omega^\bullet_\textrm{PLdR}(S)\to C^\bullet(S;k)$ given by
\[(\oint f)(x)=\int_{\Delta^n}f(x)\]
where $f(x)=\hat{f}(x)dt_1\cdots dt_n$ and $\hat{f}(x)\in\Q[t_1,\cdots,t_n]$. This map is simplicial $\{\oint_k\}_{k\geq0}$, and each $\oint_k$ is a linear chain map. The Stokes' theorem applies.

\begin{prop}[PL de Rham theorem]
For a simplicial set $S$, the integration map $\oint:\Omega_\mathrm{PLdR}^\bullet(S)\to C^\bullet(S;k)$ is a quasi-isomorphism.
\end{prop}
\begin{proof}
We split the proof into two steps. The first step provides the basis for the skeleta induction in the second step.\medbreak

\textit{First step.} The goal is to prove the theorem for $S=\Delta[n]$ for any $n$. The limit construction gives $\smash{\Omega^\bullet_\textrm{PLdR}(\Delta[n])=\Omega^n_\textrm{poly}}$, and the natural identification
\begin{align*}
\Lambda(t_0,\cdots,t_n,dt_0,\cdots,dt_n)/\left(\sum_{i=0}^nt_i=1,\sum_{i=0}^ndt_i=0\right)=\Lambda(t_1,\cdots,t_n,dt_1,\cdots,dt_n)
\end{align*}
identifies $\Omega^n_\textrm{poly}$ with the tensor product of $n$ copies of $\Lambda(t,dt)=\Omega^1_\textrm{poly}$. Since $\Omega^1_\textrm{poly}$ has trivial cohomology, so does $\smash{\Omega^n_\textrm{poly}}$ by the Künneth formula. The chain map $\smash{\oint}_n$ takes the trivial class in $\smash{\Omega^n_\textrm{poly}}$ to the trivial class in $\smash{C^n(\Delta[n],k)}$, thus inducing isomorphism on cohomology.\medbreak

\textit{Second step.}
Any simplicial set $S$ has a skeletal filtration
\[\emptyset=S^{(-1)}\subseteq S^{(0)}\subseteq\cdots\subseteq S=\textstyle{\colim_nS^{(n)}}\]
where $S^{(n)}$ is generated by non-degenerate simplices of $S$ of degree at most $n$. The base case $\Omega^\bullet_\textrm{PLdR}(S^{(0)})\simeq C^\bullet(S^{(0)};k)$ is trivial. Now suppose the claim is true for $S^{(n-1)}$, then $S^{(n)}$ is obtained by gluing copies of $\Delta[n]$, which appears as a pushout of the form $\coprod\Delta[n]\leftarrow\coprod\partial\Delta[n]\rightarrow S$. Consider the cube diagram:
\[\begin{tikzcd}[column sep=scriptsize]
\Omega^\bullet_\textrm{PLdR}(S^{(n)})\arrow[dd]\arrow[dr]\arrow[rr]&&\Omega^\bullet_\textrm{PLdR}(\Delta[n])\arrow[dd,dashed,two heads]\arrow[dr,"\simeq"]\\
&C^\bullet(S^{(n)};k)\arrow[dd]\arrow[rr]&&C^\bullet(\Delta[n];k)\arrow[dd,two heads]\\
\Omega^\bullet_\textrm{PLdR}(S^{(n-1)})\arrow[dr,"\simeq"']\arrow[rr,dashed]&&\Omega^\bullet_\textrm{PLdR}(\partial\Delta[n])\arrow[dr,dashed,"\simeq"]\\
&C^\bullet(S^{(n-1)};k)\arrow[rr]&&C^\bullet(\partial\Delta[n];k)
\end{tikzcd}.\]
The front and back squares are pullbacks, since both $\Omega^\bullet_\textrm{PLdR}$ and $C^\bullet$ are contravariant functors that preserve (co)limits. The contravariance also sends inclusions (which are cofibrations in $\sSet$) to surjections (which are fibrations in $\dgcAlg_k$. The front and back squares are connected by $\oint$. The bottom horizontal quasi-isomorphisms are obtained from inductive assumption (noticing that $\partial\Delta[n]$ is an $(n-1)$-skeleton. The upper-right quasi-isomorphism follows from the First step. The cube lemma then completes the induction $\Omega^\bullet_\textrm{PLdR}(S^{(n)})\simeq C^\bullet(S^{(n)};k)$.
\end{proof}

We define the right adjoint functor $\mathcal{K}_\bullet$ of $\Omega_\textrm{PLdR}$ as another hom-set functor:
\begin{align*}
\dgcAlg_k^\textrm{op}&\to\sSet\\
A&\mapsto\Hom_{\dgcAlg_k}(A,\Omega^\bullet_\textrm{poly})
\end{align*}
which forces the adjunction
\[\begin{tikzcd}
	{\Omega_\textrm{PLdR}:\sSet} & {\dgcAlg_k^\textrm{op}:\mathcal{K}_\bullet}
	\arrow[""{name=0, anchor=center, inner sep=0}, shift left=1.5, from=1-1, to=1-2]
	\arrow[""{name=1, anchor=center, inner sep=0}, shift left=1.5, from=1-2, to=1-1]
	\arrow["\dashv"{anchor=center, rotate=-90}, draw=none, from=0, to=1]
\end{tikzcd}\]
by the following calculation:
\begin{align*}
\Hom_{\CDGA_k}(A,\Omega_\textrm{PLdR}(S))&\cong\Hom_{\CDGA_k}(A,\textstyle{\lim_{\Delta[n]\to S}\Omega_\textrm{poly}^\bullet}) \\
&\cong\colim_{\Delta[n]\to S}\mathcal{K}_n(A)\\
&\cong\colim_{\Delta[n]\to S}\Hom_{\sSet}(\mathcal{K}(A),\Delta[n])\quad\textrm{(Yoneda lemma)}\\
&\cong\Hom_{\sSet}(\mathcal{K}(A),S).
\end{align*}
The bullet subscript $_\bullet$ implies that the functor carries simplicial data in accordance with $\Omega^\bullet_\textrm{poly}$, so that $\mathcal{K}_n(A)=\Hom_{\dgcAlg_k}(A,\Omega^n_\textrm{poly})$.

\begin{prop}
The adjunction $\Omega_\mathrm{PLdR}\dashv\mathcal{K}_\bullet$ forms a Quillen adjunction.
\end{prop}
\begin{proof}
We show that $\Omega_\textrm{PLdR}$ preserves cofibrations and acyclic cofibrations. We only need to check this for generating sets $I$ and $J$.\medbreak

\textit{Preserving cofibrations.} We need to show that the inclusions $\partial\Delta[n]\hookrightarrow\Delta[n]$ are sent to degreewise surjections in $\dgcAlg_k$. Moerdijk 29.\medbreak

\textit{Preserving acyclic cofibrations.} f
\end{proof}

Since every simplicial set is cofibrant in $\sSet$ and every CDGA$_k$ is fibrant in $\CDGA_k$, we obtain a pair of derived functors
\[\begin{tikzcd}
	{\mathbb{L}\Omega_\textrm{PLdR}:\Ho(\sSet)} & {\Ho(\CDGA_k):\mathbb{R}\mathcal{K}_\bullet}
	\arrow[""{name=0, anchor=center, inner sep=0}, shift left=1.5, from=1-1, to=1-2]
	\arrow[""{name=1, anchor=center, inner sep=0}, shift left=1.5, from=1-2, to=1-1]
	\arrow["\dashv"{anchor=center, rotate=-90}, draw=none, from=0, to=1]
\end{tikzcd}\]
which allows us to compare the homotopy theories of $\sSet$ and $\CDGA_k$.

\begin{prop}
The map $A\mapsto M[\Omega_\textrm{PLdR}(\mathcal{K}_\bullet(A))]$ is a weak equivalence for minimal Sullivan models $A\in\CDGA_{\Q,1,f}$.  
\end{prop}
\begin{proof}
By $2$-out-of-$3$ property of weak equivalence, we only need to prove the statement for $A\mapsto\Omega_\textrm{PLdR}(\mathcal{K}_\bullet(A))$, and we do that by induction.\medbreak

\textit{Base case.} Let $A=\Lambda x$ with $\abs{x}=n$. Then $H^\ast(A)=\Q[x]$, and
\[H^\ast(\Omega_\textrm{PLdR}(\mathcal{K}_\bullet(A)))\stackrel{(1)}{=}H^\ast(\mathcal{K}_\bullet(A))\stackrel{(2)}{=}H^\ast(K(\mathcal{Q}^\ast,n))=\Q[x]\]
where $(1)$ is induced by $\oint$ and $(2)$ is a consequence of \textcolor{red}{blah}.\medbreak

\textit{Inductive step.} Let $A$ be a Sullivan algebra and assume $A\mapsto\Omega_\textrm{PLdR}(\mathcal{K}_\bullet(A))$ is a weak equivalence.
Apply $\Omega_\textrm{PLdR}\mathcal{K}_\bullet$ to the back square where $B$ is the pushout:
\[\begin{tikzcd}
	{\Lambda\ch(S^n)} && A\arrow[dd,phantom,"\ulcorner",very near end, shift right=4ex] \\
	& {\Omega_\textrm{PLdR}[\mathcal{K}^\bullet(\Lambda\ch(S^n))]} && {\Omega_\textrm{PLdR}(\mathcal{K}^\bullet(A))} \\
	{\Lambda\ch(D^{n-1})} && B \\
	& {\Omega_\textrm{PLdR}[\mathcal{K}^\bullet(\Lambda\ch(D^{n-1}))]} && {\Omega_\textrm{PLdR}(\mathcal{K}^\bullet(B))}
	\arrow["\simeq"{description}, from=1-1, to=2-2]
	\arrow[from=1-1, to=1-3]
	\arrow[from=2-2,crossing over, to=2-4]
	\arrow["\simeq"{description}, from=1-3, to=2-4]
	\arrow[from=1-1, to=3-1]
	\arrow[from=2-2,crossing over, to=4-2]
	\arrow["\simeq"{description}, from=3-1, to=4-2]
	\arrow[from=2-4, to=4-4]
	\arrow[from=3-1, to=3-3]
	\arrow[from=1-3, to=3-3]
	\arrow[from=3-3, to=4-4]
	\arrow[from=4-2, to=4-4]
\end{tikzcd}.\]
The three weak equivalences are by 
We see that $B\mapsto\Omega_\textrm{PLdR}(\mathcal{K}_\bullet(B))$ is a weak equivalence.

Recall that every $1$-connected
\end{proof}

\begin{prop}
The map $\mathcal{K}_\bullet[M(\Omega_\textrm{PLdR}(-))]$ is a weak equivalence in ${\sSet_{\Q,1,f}}$.
\end{prop}

\newpage
\section{The Case $R=\Z$ or $\Z_p$ for prime $p$}

\subsection{Completion of finite loop space at prime $p$}

A principal $G$-bundle $p:E\to B$ where $G$ acts trivially on $B$ satisfies the local triviality condition: for a well chosen cover $\mathcal{U}$ of $B$ there is a $G$-homeomorphism $\varphi_U:p^{-1}(U)\to U\times G$ for any $U\in\mathcal{U}$ such that the diagram on the left commutes.
\[\begin{tikzcd}
p^{-1}(U)\arrow[d,"p"']\arrow[r,"\varphi_U"]&U\times G\arrow[dl]\\
U
\end{tikzcd}\]
In other words, a principal $G$-bundle $E\to B$ consists of a locally trivial free $G$-space $E$ with orbit space $B$.

There exists a universal $G$-bundle $EG\to B$ such that for any space $X$ there is an isomorphism $\varphi:[X,B]\to P_G$ by $f\mapsto f^\ast EG$ where $P_GX$ is the isomorphism class of principal $G$-bundles over $X$. Such $B$ is called the classifying space of $G$ and denoted $\B G$.

Under mild conditions (such as paracompactness or countability) $G\to EG\to\B G$ is a fibration, and from
\[\begin{tikzcd}
G\arrow[r]\arrow[d]&EG\arrow[r]\arrow[d]&\B G\arrow[d]\\
\Omega\B G\arrow[r]&\mathcal{P}\B G\arrow[r]&\B G
\end{tikzcd}\]
we see that $G$ is weak homotopy equivalent to $\Omega\B G$, taking $EG$ to be contractible.

A finite loop space consists of a space $X$ and its classifying space $\B X$ such that $X$ is homotopy equivalence to $\Omega\B X$. A $p$-compact group $X$ is a finite loop space such that\begin{enumerate}[(a)]
    \item $\B X$ is $\mathrm{F_p}$-local
    \item $H^i(X,\F_p)$ is finite for all $i$
\end{enumerate}
taking $\mathrm{F_p}$ to be the class of $\F_p$-equivalences.

\subsection{Arithmetic fracture theorem and local-global transform}

\begin{prop}
Let $R$ be a commutative Noetherian ring of finite Krull dimension and $p\in R$ a non-unit prime. If $H^\ast(X;R)$ is a polynomial $R$-algebra of finite type, then $H^\ast(X;\F_p)$ is a polynomial $\F_p$-algebra of the same type.
\end{prop}
\begin{proof}
The key identity of the proof is
\begin{equation}\label{key} H^\ast(X;R)\otimes_R (R/pR)\cong H^\ast(X;R/p)\cong H^\ast(X;\F_p)\otimes_{\F_p}(R/pR).
\end{equation}
Assume (\ref{key}) is correct, \medbreak

The first identity of (\ref{key}) is ring-theoretical. Apply universal coefficient theorem to both sides, the $\mathrm{Ext}$-terms vanish since $H^\ast(X;R)$ is free over $R$
\end{proof}


\begin{prop}
Let $I$ be a set of primes and $J$ the set of primes not in $I$. If for each $p\in I$ there is a space $B_p$ such that $H^\ast(B_p;\F_p)$ is a polynomial $\F_p$-algebra of finite even type, then there exists a $1$-connected space $Y$ of finite type such that $H^\ast(Y;\Z[J^{-1}])$ is a polynomial $\Z[J^{-1}]$-algebra of the same type.
\end{prop}

Let $B_p$ be $\F_p$-complete and $H^\ast(B_p;\Z_p)$ be a polynomial algebra over $\Z_p$ with generators of degrees $2d_1,2d_2,\cdots,2d_r$, where $\Z_p$ is the ring of $p$-adic integers. If $p>\max\{d_1,\cdots,d_r\}$, then
\[\pi_n(B_p)\cong\pi_{n-1}\left((S^{2d_1-1}\times\cdots\times S^{2d_r-1})_p^\wedge\right)\]\medbreak

\noindent Let $P$ be a set of primes and $Y$ be the homotopy pullback of in the following:
\[\begin{tikzcd}
Y\arrow[r]\arrow[d]&\prod_{p\in P}B_p\arrow[d]\\
K\arrow[r,"f"]&(\prod_{p\in P}B_p)_\Q
\end{tikzcd}\]
where $B_p$ is $\F_p$-complete, $K=K(\Z[P^{-1}],2d_1)\times\cdots\times K(\Z[P^{-1}],2d_r)$, and $f$ is induced by $\Z[P^{-1}]\to\Q\to(\prod_{p\in P}\Z_p)\otimes\Q$. We want to find a description of $Y$ using Mayer-Vietoris sequence of homotopy.\medbreak

\noindent The correct answer: $\pi_n(Y)=(\bigoplus_{i,2d_i=n}\Z)\oplus(\bigoplus_{p\in P}\Tor(\Q/\Z,\pi_n(B_p)))$.\medbreak

\noindent My attempt: apply Mayer-Vietoris sequence to the pullback square, we have the following long exact sequence:
\[\begin{tikzcd}[column sep=small]
\cdots\arrow[r]&\pi_{n+1}((\prod_{p\in P}B_p)_\Q)\arrow[r]&\pi_n(Y)\arrow[r,"h"]&\pi_n(K)\times\pi_n(\prod_{p\in P}B_p)\arrow[r]&\pi_n((\prod_{p\in P}B_p)_\Q)\arrow[r]&\cdots
\end{tikzcd}\]
We also know from a previous result that $\pi_n((\prod_{p\in P}B_p)_\Q)$ is only nonzero for even $n$. If $h$ is an isomorphism, then
\[\textstyle{\pi_n(Y)=(\bigoplus_{i,2d_i=n}\Z[P^{-1}])\oplus(\bigoplus_{p\in P}\pi_n(B_p))}\]
which is similar to the correct answer. The difference is $\Z$ instead of $\Z[P^{-1}]$ and the Tor group.

\subsection{Classification of $p$-compact group: an overview}

\newpage
\section{Rational Homotopy Theory: Heuts' approach}

In 1960 in his seminal paper \cite{Quillen}, Quillen proved the following equivalence among homotopy categories
\[\Ho(\Top_{\Q,1})\simeq\Ho(\DGLie_{\Q,0})\simeq\Ho(\CDGcoA_{\Q,1})\]
where $\DGLie_\Q$ is the category of differential graded Lie algebras over $\Q$ and $\CDGcoA_\Q$ is the category of co-commutative differential graded coalgebras over $\Q$. The subscripts $1$ and $0$ denote the restriction to $1$-connected and connected objects respectively.

Similar to Sullivan's approach, Quillen formed adjunctions between a chain of model categories and proved that these adjunctions induce Quillen equivalences. Quillen's approach is slightly more general as it does not need the finiteness condition, but it is also more involved and less computable in nature.

Instead of following Quillen's original treatment which is also outlined in \cite{Murillo}, we will take a more modern and categorical-formal route developed by Heuts \cite{Heuts} in which he adopted the language of $\infty$-categories of Lurie's.

\subsection{Goodwillie calculus on $\infty$-categories}
We provide background on Goodwillie calculus and $\infty$-categories necessary for Heut's approach.\medbreak

The concept of homotopy (co)limit and its special cases: homotopy orbit, homotopy fiber, and homotopy pushout/pullbacks.

\begin{defns}
With respect to $f:W\to X$ and $g:W\to Y$,\begin{enumerate}[(i)]
    \item the \textbf{standard homotopy pushout} is the space $M=X\vee(W\wedge[0,1])\vee Y$ with identifications $f(w)\sim(w,0)$ and $g(w)\sim(w,1)$;
    \item the \textbf{homotopy pushout} is a space $Z\simeq M$ with $hf\simeq kg$ for $h:X\to Z$ and $k:Y\to Z$.
\end{enumerate}
\end{defns}
(Give an example of standard homotopy pushout) (Illustrate)\medbreak

The Goodwillie calculus of functors was introduced by Goodwillie \cite{Goodwillie} in the context of spaces of spectra. In the following, let $\mathcal{C}$ and $\mathcal{D}$ be model categories that are pointed (admit an object $\ast$ that is both initial and final), and let $F:\mathcal{C}\to\mathcal{D}$ be a homotopy functor, that is, a functor that preserves weak equivalences.

\begin{defn}
A functor $F:\mathcal{C}\to\mathcal{D}$ is \textbf{$1$-excisive} if it sends homotopy pushout squares in $\mathcal{C}$ to homotopy pullback squares in $\mathcal{D}$.
\end{defn}

Examples of $1$-excisive functors include $\Sigma^\infty:\Top_\ast\to\Sp$ and $\Omega^\infty:\Sp\to\Top_\ast$, since in $\Sp$ homotopy pushout squares coincides with homotopy pullback squares.\medbreak

Consider the homotopy pushout $\Sigma X$ of the diagram $\C X\leftarrow X\rightarrow\C X$ in $\mathcal{C}$ where $\C X\simeq\ast$. For a general $F:\mathcal{C}\to\mathcal{D}$, we wish to find a best approximation of $F$ that is $1$-excisive. Define the degree-$1$ approximation of $F$ as
\[P_1(F)(X)=\hocolim(F(X)\to T_1F(X)\to T_1^2(F)(X)\to\cdots)\]
where $T_1(F)(X)$ is the homotopy pullback of $F(\C X)\rightarrow F(\Sigma X)\leftarrow F(\C X)$ in $\mathcal{D}$, and $T_1^2(F)(X)$ is obtained via iterating the same procedure on the functor $T_1(F)$. The observation is that $P_1(F)$ is $1$-excisive.\medbreak

For example, since the functor $\Id_{\Top_\ast}$ is not $1$-excisive. The homotopy pullback is $T_1(\Id_{\Top_\ast})(X)=\Omega\Sigma X$. Hence $P_1(\Id_{\Top_\ast})=\hocolim_n\Omega^n\Sigma^nX=\Omega^\infty\Sigma^\infty X$, which is $1$-excisive.

\begin{defn}
A functor $F:\mathcal{C}\to\mathcal{D}$ is \textbf{$n$-excisive} if it sends strongly homotopy cocartesian cubes $(n+1)$-cubes (homotopy pushout at $\bigcup_{1\leq i\leq n+1}X_i$)
\[\begin{tikzcd}
	{X_0} && {X_1} \\
	& \cdots && {X_1\cup_{X_0}X_2} \\
	{X_{n+1}} && {X_1\cup_{X_0}X_{n+1}} \\
	& {X_{n+1}\cup_{X_0}X_n} && {\bigcup_{1\leq i\leq n+1}X_i}
	\arrow[from=1-1, to=3-1]
	\arrow[from=1-1, to=2-2]
	\arrow[dashed, from=3-1, to=3-3]
	\arrow[from=1-1, to=1-3]
	\arrow[from=2-2, to=2-4]
	\arrow[from=1-3, to=2-4]
	\arrow[dashed, from=1-3, to=3-3]
	\arrow[from=3-1, to=4-2]
	\arrow[from=2-2, to=4-2]
	\arrow[from=4-2, to=4-4]
	\arrow[from=2-4, to=4-4]
	\arrow[dashed, from=3-3, to=4-4]
\end{tikzcd}\]
to homotopy cartesian $(n+1)$-cubes (homotopy pullback at $F(X_0)$)
\[\begin{tikzcd}[column sep=small, row sep=scriptsize]
	{F(X_0)} && {F(X_1)} \\
	& \cdots && {F(X_1\cup_{X_0}X_2)} \\
	{F(X_{n+1})} && {F(X_1\cup_{X_0}X_{n+1})} \\
	& {F(X_{n+1}\cup_{X_0}X_n)} && {F(\bigcup_{1\leq i\leq n+1}X_i)}
	\arrow[from=1-1, to=3-1]
	\arrow[from=1-1, to=2-2]
	\arrow[dashed, from=3-1, to=3-3]
	\arrow[from=1-1, to=1-3]
	\arrow[from=2-2, to=2-4]
	\arrow[from=1-3, to=2-4]
	\arrow[dashed, from=1-3, to=3-3]
	\arrow[from=3-1, to=4-2]
	\arrow[from=2-2, to=4-2]
	\arrow[from=4-2, to=4-4]
	\arrow[from=2-4, to=4-4]
	\arrow[dashed, from=3-3, to=4-4]
\end{tikzcd}.\]
\end{defn}

By the very same homotopy colimit construction we obtain the higher-dimensional analogue of degree $1$-approximation: the degree $n$-approximation of $F:\mathcal{C}\to\mathcal{D}$
\[P_n(F)(X)=\hocolim(F(X)\to T_n(F)(X)\to T_n^2(F)(X)\to\cdots)\]
where $T_n(F)(X)$ is the homotopy limit of the cube formed by $n+1$ copies of $F(X)\to F(C(X))$. The key observation is that $P_n(F)$ is $n$-excisive.

The degree-$n$ approximation is universal in the sense that for any $n$-excisive functor $G$, the map $F\to G$ factors through $P_n(F)$. Another observation is that $P_n(P_k(F))$ is equivalent to $\smash{P_{\min\{n,k\}}(F)}$.\medbreak

The approximations of different degrees thence assemble into a tower
\[\begin{tikzcd}
&F\arrow[d]\arrow[dr]\arrow[drr]\\
F(\ast)&P_1(F)\arrow[l]&P_2(F)\arrow[l]&P_3(F)\arrow[l]&\cdots\arrow[l]\\
&D_1(F)\ar[u,equal]&D_2(F)\arrow[u]&D_3(F)\arrow[u]
\end{tikzcd}\]
where the map $P_n(F)\to P_{n-1}(F)$ is given by the universal property of $P_n(F)$ along with the observation that $P_{n-1}(F)$ is also $n$-excisive. The layer $D_n(F)$ denotes the homotopy fiber of $P_n(F)\to\ P_{n-1}(F)$, which is homogeneous of degree $n$.

\begin{defn}
An $n$-excisive functor $F$ is \textbf{homogeneous} if $P_{n-1}(F)\simeq\ast$.
\end{defn}

The functor $\Top_\ast\to\Sp$ given by $X\mapsto(C\wedge X^{\wedge n})_{h\Sigma_n}$ is homogeneous of degree $n$, where $C$ is a spectrum with an action by symmetric group $\Sigma_n$. This is because the homotopy orbit as a homotopy colimit preserves coCartesian cubes.\medbreak

We study the layers $D_n(F)$ more closely. For simplicity, consider finitary functors $F$ with codomain $\Sp$, that is, functors that preserve filtered homotopy colimits.

Define the first derivative of $F$ at $\ast$ to be $\partial^{(1)}(F)(\ast)=D_1(F)(\mathbb{S})$ where $\mathbb{S}$ is the sphere spectrum $\Sigma^\infty S^0$. Then $\partial^{(1)}(F)(\ast)\wedge K=D_1(F)(K)$ for any spectrum $K$.\medbreak

Just as there is a bijection between symmetric bilinear forms and quadratic forms, there is a bijection between symmetric multilinear functors $L:\mathcal{C}^n\to\mathcal{D}$ and $n$-homogeneous functors $F:\mathcal{C}\to\mathcal{D}$. Here the linear means $1$-homogeneous. One direction is easy: for $L:\mathcal{C}^n\to\mathcal{D}$, there is an $n$-homogeneous functor $(L\Delta)_{h\Sigma_n}$ where $\Delta:\mathcal{C}^n\to\mathcal{C}$ is the diagonal functor. The $\Sigma_n$-action captures the symmetry.

\begin{defn}
The \textbf{$n$-th cross effect} $\cros_n(F)(X_1,\cdots,X_n)$ of a functor $F$ with $n$ variables is the fiber of $F(X_1\wedge\cdots\wedge X_n)\to\holim F(\chi)$ where $\chi$ is the cube spanned by $\{X_i\to\ast\}_{1\leq i\leq n}$ with the first vertex removed.
\end{defn}

The $n$-th cross effect $\cros_n(F)$ measures the failure of $F$ to be an $(n-1)$-excisive functor. For example, if $F(X,Y)$ is $2$-excisive, then $F(X\wedge Y)=\holim F(\chi)$ where $\chi=X\rightarrow\ast\leftarrow Y$, and the fiber $\cros_2(F)$ is trivial.

The key observation is that $\cros_n(D_n(F))$ is a symmetric multilinear functor. Define the $n$-th derivative of $F$ at $\ast$ to be $\partial^{(n)}(F)(\ast)=\cros_n(D_n(F))$, then the multilinearity gives $\partial^{(n)}F(\ast)\wedge X_1\wedge\cdots\wedge X_n=\cros_nD_nF(X_1,\cdots,X_n)$. Associate $\cros_nD_n(F)$ with the $n$-homogeneous functor $D_n(F)$, we obtain $D_n(F)(X)\simeq(\partial^{(n)}F(\ast)\wedge X^{\wedge n})_{h\Sigma_n}$

\medbreak
We now reformulate Goodwillie calculus in the language of $\infty$-categories. The main reference is Lurie \cite{Lurie}.

\subsection{Goodwillie approximation to $\infty$-categories} We develop in this section the central construction of Heuts' construction.

\newpage
\section{Closing Remarks}

The Serre class enjoys many pereserveance properties, see the beautiful work by Serre\cite{Serre}

For a more general statement of Whitehead theorem mod $p$ that works with nilpotent spaces see Schiffman\cite{Schiffman}

For many more interesting interplay between algebra and topology see \cite{Avramov}.

A possible generalization and still currently in research is the survey Behrens-Rezk \cite{Behrens-Rezk}

A classical proof of the Quillen adjunction between $\mathcal{S}_\bullet$ and $\abs{-}$ can be found in May\cite{May2}

For an encyclopedic survey on rational homotopy theory see the book by Félix, Halperin, and Thomas\cite{Felix}. A more concise note by the same authors see Félix-Halperin\cite{Felix-Halperin}. The three lecture notes on rational homotopy theory that I learned from are: Berglund\cite{Berglund}, Hess\cite{Hess}, Moerman\cite{Moerman}

Two places for model categories are May\cite{May} and Goerss-Schemmerhorn\cite{Goerss-Schemmerhorn}

A slight generalization in Jardin\cite{Jardin} gives model strucutre on mod-$p$ CDGA

\newpage
\section*{Appendix}

Model category:\begin{enumerate}
    \item closed under limits and colimits
    \item 2 classes determines the third \begin{enumerate}
        \item $\W=\Fib^r\Cof^\ell$
        \item $\Fib=(\Cof\cap\W)^r$
        \item $\Cof=(\Fib\cap\W)^\ell$
    \end{enumerate}
    \item factorization
    \item $\W$ contains all iso, 2-out-of-3 property for $\W$
\end{enumerate}

\section*{Acknowledgements}

It is a pleasure to thank my mentor, Danny Xiaolin Shi, for blah. I also thank blah for helping 
me understand blah.

\newpage
\begin{thebibliography}{2}

\bibitem{Andersen-Grodal}
Andersen, K. K., \& Grodal, J. (2008). \textit{The Steenrod problem of realizing polynomial cohomology rings}. Journal of Topology, 1(4), 747–760. https://doi.org/10.1112/jtopol/jtn021 

\bibitem{Avramov}
Avramov, L. L., Christensen, J. D., Dwyer, W. G., Mandell, M. A., \& Shipley, B. E. (2007). \textit{Interactions between homotopy theory and algebra}. American Mathematical Society.

\bibitem{Behrens-Rezk}
Behrens, M., \& Rezk, C. (2020). \textit{Spectral Algebra Models of Unstable $v_n$-Periodic Homotopy Theory}. https://doi.org/10.1007/978-981-15-1588-0\_10 

\bibitem{Berglund}
Berglund, A. (2012). \textit{Rational homotopy theory}. University of Copenhagen lecture notes.

\bibitem{Felix}
Félix, Y., Halperin, S., Thomas, J.-C.. \textit{Rational homotopy theory}. Graduate Texts in Mathematics, vol. 205. Springer-Verlag, 2001.

\bibitem{Felix-Halperin}
Félix, Y., \& Halperin, S. (2017). \textit{Rational Homotopy Theory via Sullivan Models: A Survey}. https://doi.org/10.4310/iccm.2017.v5.n2.a3

\bibitem{Goodwillie} Goodwillie

\bibitem{Heuts} Heuts, Gij

\bibitem{Hirschhorn}
Hirschhorn, P. S. (2019). \textit{The Quillen model category of topological spaces}. Expositiones Mathematicae, 37(1), 2–24. https://doi.org/10.1016/j.exmath.2017.10.004 

\bibitem{Goerss-Schemmerhorn}
Goerss, P., \& Schemmerhorn, K. (2007). \textit{Model categories and simplicial methods}. Interactions between Homotopy Theory and Algebra, 3–49. https://doi.org/10.1090/conm/436/08403 

\bibitem{Hess}
Hess, K. (2007). \textit{Rational homotopy theory: a brief introduction}. Interactions between Homotopy Theory and Algebra, 175–202. https://doi.org/10.1090/conm/436/08409

\bibitem{Lurie} Lurie, Jacob. Higher algebra

\bibitem{May2} May, J. P.. \textit{Simplicial Objects in Algebraic Topology}. 1967

\bibitem{May} 
May, J. P., \& Ponto, K.. \textit{More Concise Algebraic Topology}. 2011.

\bibitem{Moerman} lah

\bibitem{Murillo} Murillo, Ancieto. (2017). \textit{Quillen rational homotopy theory revisited}

\bibitem{Quillen} Quillen, Dan. Rational Homotopy Theory

\bibitem{Schiffman}
Schiffman, S. J. (1981). A mod $p$ Whitehead Theorem. Proceedings of the American Mathematical Society, 82(1), 139. https://doi.org/10.2307/2044332 

\bibitem{Serre}
Serre, J.-P. (1953). Groupes D'Homotopie Et Classes De Groupes Abeliens. The Annals of Mathematics, 58(2), 258. https://doi.org/10.2307/1969789 

\bibitem{Steenrod}
Steenrod. N. E. (1962). \textit{The cohomology algebra of a space}. Enseignement Math, (2)7:153–178.

\bibitem{Sullivan} ewf

\end{thebibliography}


\end{document}

