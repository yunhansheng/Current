\documentclass[psamsfonts]{amsart}

\usepackage{amssymb,amsfonts}
\usepackage[all,arc]{xy}
\usepackage{enumerate}
\usepackage[shortlabels]{enumitem}
\usepackage{physics}
\usepackage{tikz-cd}
\usetikzlibrary{decorations.pathmorphing}

\newtheorem{thm}{Theorem}[section]
\newtheorem*{thm*}{Theorem}
\newtheorem{cor}[thm]{Corollary}
\newtheorem{prop}[thm]{Proposition}
\newtheorem{lem}[thm]{Lemma}
\newtheorem{conj}[thm]{Conjecture}

\theoremstyle{definition}
\newtheorem{defn}[thm]{Definition}
\newtheorem{defns}[thm]{Definitions}
\newtheorem{exmp}[thm]{Example}
\newtheorem{exmps}[thm]{Examples}
\newtheorem{notn}[thm]{Notation}
\newtheorem{notns}[thm]{Notations}
\newtheorem{addm}[thm]{Addendum}
\newtheorem{exer}[thm]{Exercise}

\theoremstyle{remark}
\newtheorem{rem}[thm]{Remark}
\newtheorem{rems}[thm]{Remarks}
\newtheorem{warn}[thm]{Warning}

\newcommand{\Q}{\mathbb{Q}}
\newcommand{\s}{\mathbf{S}}
\newcommand{\Z}{\mathbb{Z}}
\newcommand{\Ab}{\mathbf{Ab}}
\newcommand{\Hom}{\textup{Hom}}
\newcommand{\CDGA}{\mathbf{CDGA}}
\newcommand{\simp}{\mathbf{\Delta}}
\newcommand{\Ho}{\mathbf{Ho}}
\newcommand{\N}{\mathbb{N}}
\newcommand{\Top}{\mathbf{Top}}
\newcommand{\Obj}{\textup{Obj}}
\newcommand{\id}{\textup{id}}
\newcommand{\sSet}{\mathbf{sSet}}

\DeclareMathOperator{\coker}{coker}
\DeclareMathOperator{\Sym}{Sym}
\DeclareMathOperator{\Etr}{Etr}

\makeatletter
\let\c@equation\c@thm
\makeatother
\numberwithin{equation}{section}

\bibliographystyle{plain}

\title{The Realization of Rational and Polynomial Cohomology Rings}

\author{Yunhan (Alex) Sheng}

\begin{document}

\begin{abstract}

This expository paper written during the $\textrm{\smash{2020 REU}}$ program at the University of Chicago focuses on a realization problem: which commutative graded $R$-algebras can be realized as the cohomology ring $\smash{H^\ast(X;R)}$ of a space $X$ with coefficient in $R$? The case $R=\Q$ can be solved via Sullivan's approach to rational homotopy theory developed in \cite{Sullivan}. For polynomial algebras the question was resolved completely by Andersen-Grodal\cite{Andersen-Grodal}, and partially by many others. Surprisingly, for other cases, the problem remains open.

\end{abstract}

\maketitle

\tableofcontents

\section{Introduction}

Historical information
%Homotopy groups $\pi_n(X)$ of a $0$-connected space $X$ are abelian for $n\geq2$, as are its homology groups $H_n(X)$. Consider the corresponding torsion-free versions $\pi_n(X)\otimes\Q$ and $H_n(X;\Q)$. It turns out that a space $X$ has a rationalization $X_\Q$ satisfying $\pi_n(X_\Q)=\pi_n(X)\otimes\Q$, and two spaces $X$ and $Y$ are rationally homotopy equivalent (denoted $\smash{X\sim_\Q Y}$) if $X_\Q\sim Y_\Q$.

\section{The Case $R=\Q$}

They main goal of this section is to show, restricting to rational, $1$-connected objects of finite type, an equivalence between homotopy categories:
\[\Ho(\Top_{\Q,1,f})\cong\Ho(\CDGA_{\Q,1,f}).\]

We aim to do so by constructing Quillen adjunctions $S_\bullet$-$\abs{-}$ and $\mathcal{A}^\ast$-$\mathcal{K}_\bullet$ between
\[\begin{tikzcd}
\Top\arrow[r,shift left,rightsquigarrow,"S_\bullet"]&\sSet\arrow[l,shift left,rightsquigarrow,"\abs{-}"]\arrow[r,shift left,rightsquigarrow,"\mathcal{A}^\ast"]&\CDGA_\Q\arrow[l,shift left,rightsquigarrow,"\mathcal{K}_\bullet"]
\end{tikzcd}.\]

Therefore, any $1$-connected CDGA$_\Q$ of finite type can be realized as a cohomology ring $H^\ast(X;\Q)$ of some rational homotopy type.

\newpage
\subsection{Rational homotopy type and $\mathcal{Q}$-localization}

We begin by recalling the classical theorems of Hurewicz and Whitehead, with proofs omitted.

\begin{thm}[Hurewicz]
If a pair $(X,A)$ is $(n-1)$-connected for $n>1$, then $h:\pi_i(X,A)\to H_i(X,A)$ by $[f]\mapsto f_\ast(\alpha)$ is an isomorphism for $i\leq n$ with $\alpha$ being a generator of $H_n(\mathbf{D}^n,\s^{n-1})=\Z$.
\end{thm}

\begin{thm}[Whitehead]
For a map $f:X\to Y$ between $1$-connected spaces, the following are equivalent for $n\geq1$:\begin{enumerate}[(i)]
    \item The induced $\pi_i(f):\pi_i(X)\to\pi_i(Y)$ is an isomorphism for $i<n$ and an epimorphism for $i=n$;
    \item the induced $H_i(f):H_i(X)\to H_i(Y)$ is an isomorphism for $i<n$ and and epimorphism for $i=n$.
\end{enumerate}
\end{thm}

Before we consider the rational version of these theorems, let us consider a more general formulation of Serre classes.

\begin{defns}
A class of abelian groups $\mathfrak{C}\subseteq\Ab$ is a \textbf{Serre class} if the following is true: given a short exact sequence of abelian groups
\[0\to A\to B\to C\to0,\]
the groups $B\in\mathfrak{C}$ if and only if $A\in\mathfrak{C}$ and $C\in\mathfrak{C}$.

A homomorphism $f\in\Hom_{\Ab}(A,B)$ is a \textbf{$\mathfrak{C}$-monomorphism} if $\ker f\in\mathfrak{C}$, a \textbf{$\mathfrak{C}$-epimorphism} if $\coker f\in\mathfrak{C}$, and a \textbf{$\mathfrak{C}$-isomorphism} if both are satisfied.
\end{defns}

One can check that the Serre class has nice algebraic properties. For instance,

Now for mod-$\mathfrak{C}$ versions of Hurewicz and Whitehead theorems, we need the following lemma.

\begin{lem}
Let $\mathfrak{C}$ be a Serre class and $X$ be $1$-connected. Then $\pi_i(X)\in\mathfrak{C}$ for all $i$ if and only if $H_i(X)\in\mathfrak{C}$ for all $i$.
\end{lem}

\begin{thm}[mod-$\mathfrak{C}$ Hurewicz]
Let $\mathfrak{C}$ be a Serre class and $(X,A)$ be $1$-connected. If $\pi_i(X,A)\in\mathfrak{C}$ for $i<n$, then $H_i(X,A)\in\mathfrak{C}$ for $i<n$ and the Hurewicz map $h:\pi_i(X,A)\to H_i(X,A)$ is a $\mathfrak{C}$-isomorphism for $i\leq n$.
\end{thm}

\begin{thm}[mod-$\mathfrak{C}$ Whitehead]
Let $\mathfrak{C}$ be a Serre class and $f:X\to Y$ be a map between $1$-connected spaces. The following are equivalent for $n\geq1$:\begin{enumerate}[(i)]
    \item The induced $\pi_i(f):\pi_i(X)\to\pi_i(Y)$ is a $\mathfrak{C}$-isomorphism for $i<n$ and a $\mathfrak{C}$-epimorphism for $i=n$;
    \item the induced $H_i(f):H_i(X)\to H_i(Y)$ is a $\mathfrak{C}$-isomorphism for $i<n$ and a $\mathfrak{C}$-epimorphism for $i=n$.
\end{enumerate}
\end{thm}

The proofs of these mod-$\mathfrak{C}$ theorems utilizes Serre spectral sequences. For alternative proofs see Moerman\cite{Moerman} (which uses simplicial fibrations) or Klaus-Kreck\cite{Klaus-Kreck}.

Now we take $\mathcal{C}\subseteq\Ab$ to be the Serre class of torsion groups. Then $\pi_i(X)\in\mathcal{C}$ is equivalent to $\pi_i(X)\otimes\Q=0$, and $H_i(X)\in\mathcal{C}$ is equivalent to $H_i(X;\Q)=0$. Also $f\in\Hom_{\Ab}(A,B)$ is a $\mathcal{C}$-isomorphism if and only if $f\otimes\Q$ is an isomorphism.

\begin{defn}
A map $f:X\to Y$ is a \textbf{rational homotopy equivalence} if $\pi_n(f)\otimes\Q$ is an isomorphism for all $n$. In this case write $X\sim_\Q Y$.
\end{defn}

\begin{defns}
A $1$-connected space $X$ \textbf{rational} if $\pi_i(X)$ is uniquely divisible for all $i$. Let $X_\Q$ be a rational space, a map $r:X\to X_\Q$ is a \textbf{rationalization} if is a rational homotopy equivalence.
\end{defns}

There are a few different approaches to rationalization, with $\mathcal{Q}$-localization being a clean one. We begin by recalling basic notions of localization.

\begin{defns}
Let $\mathcal{C}$ be a category with $\mathcal{L}$ a subclass of morphisms in $\mathcal{C}$. An object $A\in\Obj(C)$ is \textbf{$\mathcal{L}$-local} if every $f\in\Hom_\mathcal{C}(X,Y)$ with $f\in\mathcal{L}$ induces a bijection $f^\ast:\Hom_\mathcal{C}(Y,A)\to\Hom_\mathcal{C}(X,A)$. Let $A_\mathcal{L}$ be $\mathcal{L}$-local, a map $\ell:A\to A_\mathcal{L}$ is a \textbf{$\mathcal{L}$-localization} if $\ell\in\mathcal{L}$.
\end{defns}

Note that blah

One can already see the similarities between localization and rationalization. Let $\mathcal{Q}$ be the class of rational homotopy equivalences in the homotopy category of $1$-connected spaces $\Ho(\Top_1)$.

\begin{lem}
If an object $X$ of $\Ho(\Top_1)$ is rational, then it is $\mathcal{Q}$-local. Hence a rationalization is a $\mathcal{Q}$-localization.
\end{lem}

With this one-sided identification we can prove the key theorem of this section:

\begin{thm}
Every $1$-connected space $X$ admits a rationalization.
\end{thm}

Finally, with the following converse of Lemma 2.10, we complete the identification.

\begin{lem}
If an object $X$ of $\Ho(\Top_1)$ is $\mathcal{Q}$-local, then it is rational. Hence a $\mathcal{Q}$-rationalization is a rationalization.
\end{lem}

With this identification, we note blah.

For a more visual construction of rationalization that starts with the rationalization of spheres $\s^n$ see Moerman\cite{Moerman}.

\newpage
\subsection{Quillen adjunction from $\sSet$ to $\Top$}

Here we recall some useful facts regarding simplicial sets. For a more detailed treatment see

Define singular complex functor $S_\bullet:\Top\rightsquigarrow\sSet$ by $S_n:T\mapsto\Hom_{\Top}(\Delta^n,T)$ where $\Delta^n$ is the standard topological $n$-simplex. For any $\varphi\in\Hom_{\simp}([m],[n])$ we have $\varphi_\ast:\Delta^m\to\Delta^n$ by $(t_0,\cdots,t_m)\mapsto(s_0,\cdots,c_n)$ where $\smash{s_i=\sum_{i\in\varphi^{-1}(i)}t_i}$. This induces a map $\smash{\phi^\ast:S_n(T)\to S_m(T)}$ by $\varphi^\ast(f)=f\phi$.

\[\begin{tikzcd}[row sep=small]
{[m]}\arrow[swap]{dd}{\varphi}\arrow{r}&\Delta^m\arrow[swap]{dd}{\varphi_\ast}\arrow{r}&S_m(T)=\Hom_{\Top}(\Delta^m,T)\\
&\textrm{ }\arrow[r,dashrightarrow]&\textrm{ }\\
{[n]}\arrow{r}&\Delta^n\arrow{r}&S_n(T)=\Hom_{\Top}(\Delta^n,T)\arrow[swap]{uu}{\varphi^\ast}
\end{tikzcd}\]

\newpage
\subsection{Sullivan model and the $q$-model structure on $\CDGA_\Q$}

The ``$q$'' in the naming refers to Quillen. Other choices of model structures include $h$-model and $m$-model.\medbreak

We first describe the main algebraic object of our interest. In the following let $R$ be a commutative ring.

\begin{defns}
A \textbf{differential graded module} over $R$ (or DGM$_R$) $(M,d)$ is a $R$-module $M$ together with a differential $d$ of grading degree $1$ satisfying $d^2=0$. Let $N$ be a DGM$_R$, a map $f:M\to N$ is called a \textbf{chain map} if $d_Nf=fd_M$.
\end{defns}

The category $\mathbf{DGM}_R$ is a symmetric monoidal category with tensor product $M\otimes_RN$ preserving grading structure, and symmetry given by the Leibniz rule:
\[\smash{d(a\otimes b)=da\otimes b+(-1)^{\abs{a}}\otimes db}.\]
By the graded nature, DGM$_R$ is often referred to as cochain algebra.

\begin{defn}
A \textbf{commutative differential graded algebra} over $R$ (or CDGA$_R$) $(A,d)$ is a commutative monoid in $\mathbf{DGM}_R$.

Equivalently, $(A,d)$ is an object of $\mathbf{DGM}_R$ together with a unit $\eta:R\to(A,d)$ and an associative, graded-commutative multiplication $\mu:(A,d)\otimes_R(A,d)\to(A,d)$ given by $a\otimes b\to a\cdot b$. The differential $d$ satisfies the Leibniz rule. 
\end{defn}

Let $A'$ be CDGA$_R$, a map $f:A\to A'$ satisfies $f\eta=\eta'$ and $f\mu=\mu'(f\otimes f)$, and respects the differential. This defines a category $\CDGA_R$.\medbreak

Let $k$ be a field of characteristic $0$. We now obtain a model structure on $\CDGA_k$ from a model structure on $\mathbf{DGM}_k$.

Consider the cochain algebra of $n$-disks $(D^n)^k$ and $n$-spheres $(S^n)^k$. Then the map $\smash{i_n:(S^{n+1})^\bullet\to(D^n)^\bullet}$ sends $a$ to $c$ where $a$ is the generator of $H^n(S^n)=k$ and $c$ is the generator of $H^{n+1}(D^n)$, and the map $\smash{j_n:0\to(D^n)^\bullet}$ is acyclic.

Let weak equivalences be quasi-isomorphisms and fibrations be Serre fibrations, then $I=\{i_n:n\in\N\}$ generates cofibrations and $J=\{j_n:n\in\N\}$ generates acyclic cofibrations. This defines the $q$-model structure on $\mathbf{DGM}_k$.\medbreak

Consider the pair $(\Lambda,U)$ of free-forgetful functors between $\mathbf{DGM}_k$ and $\CDGA_k$. Note that for a DGM$_k$ $M=M_0\oplus M_1$, its free extension $\Lambda M=\Sym(M_0)\otimes\Etr(M_1)$ where $M_0$ are even-degree spaces and $M_1$ are the odd-degree spaces, and $\Sym$ is the symmetric and $\Etr$ is the exterior algebra. By a careful analysis we see that $\Lambda D^n$ are acyclic, and $\Lambda(j_n)$ are quasi-isomorphisms.

\begin{thm}[$q$-model structure on $\CDGA_k$]
There is a cofibrantly generared model structure on $\CDGA_k$ with quasi-isomorphisms as weak equivalences, and $\Lambda(I)$ and $\Lambda(J)$ generating cofibrations and acyclic cofibrations respectively.
\end{thm}
\begin{proof}
See [Moerman]
\end{proof}

By repeated use of Quillen's small object argument, the pair $(\Lambda,U)$ is a Quillen adjunction.

We elaborate on the $q$-model structure on $\CDGA_k$. A fibration is just a degreewise surjection, while a cofibration is a relative Sullivan algebra (see [Berguland]).




\newpage
\section*{Acknowledgements}

It is a pleasure to thank my mentor, Danny Xiaolin Shi, for blah. I also thank blah for helping 
me understand blah.

\newpage
\begin{thebibliography}{2}

\bibitem{Andersen-Grodal} d

\bibitem{Klaus-Kreck}

\bibitem{May}
J. P. May.
A Concise Course in Algebraic Topology.
University of Chicago Press. 1999.

\bibitem{Moerman} lah

\bibitem{Quillen} lah

\bibitem{Serre} lda

\bibitem{Sullivan} ewf

\end{thebibliography}

\section*{Appendix}

Postnikov system

\end{document}

