\documentclass[11pt]{article}

\usepackage[margin=1in]{geometry}
\usepackage{amsmath,amsthm,amsfonts,amssymb}
\usepackage[utf8]{inputenc}
\usepackage[mathscr]{eucal}
\usepackage{tikz-cd}
\usepackage[shortlabels]{enumitem}
\usepackage{xcolor}
\usepackage[OT1]{fontenc}
\usepackage{physics}
\usepackage{hyperref}

\setlength\parindent{0pt}

\setcounter{section}{+1}

\theoremstyle{definition}
\newtheorem*{defn}{Definition}
\newtheorem*{example}{Example}

\theoremstyle{plain}

\newtheorem{thm}{Theorem}[section]
\newtheorem{cor}[thm]{Corollary}
\newtheorem{prop}[thm]{Proposition}
\newtheorem{lem}[thm]{Lemma}

\newcommand{\R}{\mathbf{R}}

\begin{document}

1.2: algebra, $\sigma$-algebra, Borel $\sigma$-algebra, $\sigma$-algebra generated, product $\sigma$-algebra, $F_\sigma$ and $G_\delta$ sets

Key proposition: 1.2 (generates $\mathfrak{B}_\R$), 1.4 (product $\sigma$-algebra), 1.6 ($\mathfrak{B}_{\R^n}=\bigotimes_1^n\mathfrak{B}_\R$, by 1.5)

\begin{prop}
$\mathfrak{B}_\R$ is generated by half-open intervals $\{(a,b]:a<b\}$.
\end{prop}

\begin{prop}
Suppose that $\mathfrak{M}_\alpha$ is the $\sigma$-algebra generated by $\mathcal{E}_\alpha$ for $\alpha\in A$. Then the product $\sigma$-algebra.
\end{prop}

1.3: measure, measurable set, measure space, $\sigma$-finite, almost everywhere, complete

key proposition: 1.8 (properties of measure), 1.9 (completion of measure)

\begin{defn}
An \textbf{outer measure} on $X$ is a function $\mu^\ast:2^X\to[0,\infty]$ satisfying:\begin{enumerate}[(i)]
    \item $\mu^\ast(\emptyset)=0$;
    \item (monotonicity) $\mu^\ast(A)\leq\mu^\ast(B)$ for $A\subseteq B$;
    \item (subadditivity) $\mu^\ast(\cup_{i\geq1}A_i)\leq\sum_{i\geq1}\mu^\ast(A_i)$.
\end{enumerate}
A set $A\subseteq X$ is \textbf{$\mu^\ast$-measurable} if for any $E\subseteq X$,
\[\mu^\ast(E)=\mu^\ast(E\cap A)+\mu^\ast(E\cap A^c).\]
\end{defn}

\begin{lem}
For any $A\in2^X$, the following defines an outer measure on $X$:
\[\mu^\ast(A)=\inf\left\{\sum_{i\geq1}\rho(E_i):E_i\in\mathcal{E}\textrm{, and }A\subseteq\bigcup_{i\geq1}E_i\right\},\]
where $\mathcal{E}\subseteq2^X$ contains $\emptyset$ and $X$, and $\rho:\mathcal{E}\to[0,\infty]$ satisfies $\rho(\emptyset)=0$.
\end{lem}

\begin{thm}[Carathéodory]
Let $\mu^\ast$ be an outer measure and $\mathfrak{M}$ the collection of $\mu^\ast$-measurable sets. Then $\mathfrak{M}$ is a $\sigma$-algebra and $\mu^\ast|_\mathfrak{M}$ is a complete measure.
\end{thm}

\begin{defn}
For an algebra $\mathcal{A}\subseteq2^X$, a \textbf{premeasure} on $X$ is a function $\mu_0:\mathcal{A}\to[0,\infty]$ satisfying:\begin{enumerate}[(i)]
    \item $\mu_0(\emptyset)=0$;
    \item (countable additivity) $\mu_0(\cup_{i\geq1}A_1)=\sum_{i\geq1}\mu_0(A_i)$ for disjoint sets $\{A_i\}_{i\geq1}$ with $\cup_{i\geq1}A_i\in\mathcal{A}$.
\end{enumerate}
\end{defn}

\begin{prop}
Let $\mu_0$ be a premeasure on an algebra $\mathcal{A}$. Define $\mu^\ast$ by
\[\mu^\ast(E)=\inf\left\{\sum_{i\geq1}\mu_0(A_i):A_i\in\mathcal{A}\textrm{, and }E\subseteq\bigcup_{i\geq1}A_j\right\}.\]
Then $\mu^\ast|_\mathcal{A}=\mu_0$, and every set in $\mathcal{A}$ is $\mu^\ast$-measurable.
\end{prop}

\begin{prop}
Let $\mu_0$ be a premeasure on an algebra $\mathcal{A}$. Then there exists a measure $\mu$ on $\mathfrak{M}(\mathcal{A})$. Moreover, $\mu$ is unique if $\mu_0$ is $\sigma$-finite.
\end{prop}

\begin{defn}
A \textbf{Borel measure} on $\R$ is a measure on $\mathfrak{B}_\R$.
\end{defn}

\begin{lem}
The collection of $\mathcal{A}$ of finite disjoint unions of half-open intervals is an algebra, and $\mathfrak{B}_\R=\mathfrak{M}(\mathcal{A})$
\end{lem}

In the following, let $\mathcal{A}$ be the algebra describe above, and let $F:\R\to\R$ be an increasing and right-continuous function.

\begin{lem}
For any disjoint intervals $(a_1,b_2]$ and $(a_2,b_2]$, the following defines a premeasure on $\mathcal{A}$:
\[\mu(\emptyset)=0\textrm{, and }\mu_0((a_1,b_2]\cup(a_2,b_2])=(F(b_1)-F(a_1))+(F(b_2)-F(b_2)).\]
\end{lem}

Application: there is a unique Borel measure $\mu_F$ on $\R$ such that $\mu_F((a,b])=F(b)-F(a)$ for $a,b\in\R$. It may not be complete. The completion is called the Lebesgue-Stieltjes measure.

In the following, let $\mu$ be a Lebesgue-Stieltjes measure on $\R$ associated with $F$, and $\mathfrak{M}_\mu$ be the domain of $\mu$. Then
\[\mu(E)=\inf\left\{\sum_{i\geq1}F(b_i)-F(a_i):E\subseteq\bigcup_{i\geq1}(a_i,b_i]\right\}\]

\begin{prop}
If $E\in\mathfrak{M}_\mu$, then
\[\mu(E)=\inf\{\mu(U):E\subseteq U\textrm{ and }E\textrm{ is open}\}=\inf\{\mu(U):K\subseteq E\textrm{ and }K\textrm{ is compact}\}\]
\end{prop}

\begin{cor}
Any $E\subseteq\R$ can be written as $V\setminus N$ or $H\cup N$ for $V$ a $G_\delta$ set, $H$ an $F_\delta$ set, and $N$ a null set. Moreover, $\mu(E\delta A)<\epsilon$ for some finite union of open intervals $A$ if $\mu(E)<\infty$.
\end{cor}

We further restrict ourselves to $F(x)=x$, then we get Lebesgue measure $m=\mu_F$. We write $\mathcal{L}$ for the Lebesgue measure sets its domain $\mathfrak{M}_m$.




\end{document}