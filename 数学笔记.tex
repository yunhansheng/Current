%\documentclass{tufte-book}
\documentclass[11pt]{article}

\usepackage{xeCJK}
\usepackage{amsmath,amsthm,amsfonts}
\usepackage[margin=1in]{geometry}
\usepackage[utf8]{inputenc}
\usepackage{amssymb}
\usepackage[mathscr]{eucal}
\usepackage{graphicx}
\usepackage{physics}

\newenvironment{loggentry}[1]{\noindent#1\medbreak}{\vspace{0.5cm}}

\setlength\parindent{0pt}

\setCJKmainfont{BabelStone Han}
\setmainfont{Courier New}



\begin{document}

\begin{loggentry}{June 27, 2021}
今天读了我校Benson Farb教授的一篇关于如何做数学讲座,深有感触。Farb教授的论述非常朴实,但又十分深刻。\\

“让你的生活、呼吸、睡眠都与数学相伴,如果你不愿意这样的话,那你就不适合去做数学研究”——我需要继续反思自己到底适不适合做纯数学。然后Farb教授说要锻炼自己在单位时间(比如一个月)内学习更多的数学。Farb本人是个工作狂,一周工作80个小时。\\

Farb举了他自己和他学生的例子,说明了光读懂而不会背定义、不会计算、不会做题是远远不够的。我们得经常反省,尽量正确的认识到自己到底掌握了没有。对自己诚实这点非常难\\

要尝试着去研究基本的例子,再基本再trivial的例子也值得反复推敲,不要顾颜面而去攀高。\\

这一点我前段时间也认识到了:总是有人比自己强太多,而且对此我们无能为力。我们只能做好眼前的事情,work,and then hope for the best。\\

关于数学品味,好的数学一定能回归到基本的东西,比如微积分和线性代数,像Sullivan、Serre、Milnor的工作都是这样。不要被诸如$n$-category之类的东西牵着鼻子走。
\end{loggentry}

\begin{loggentry}{June 27, 2021}
d
\end{loggentry}


\end{document}