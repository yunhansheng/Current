\documentclass[11pt]{article}

\usepackage[margin=1in]{geometry}
\usepackage{amsmath,amsthm,amsfonts}
\usepackage[utf8]{inputenc}
\usepackage{amssymb}
\usepackage[mathscr]{eucal}
\usepackage{graphicx}
\usepackage{listings}
\usepackage{titlesec}
\usepackage{tikz-cd}
\usepackage[shortlabels]{enumitem}
\usepackage{tikz}
\usepackage{xcolor}
\usepackage[OT1]{fontenc}
\usepackage{physics}
\usepackage{xpatch}
%\usepackage{hyperref}

\setlength\parindent{0pt}

\setcounter{section}{-0}
\setcounter{secnumdepth}{2}

\theoremstyle{definition}
\newtheorem*{defin}{Definition}
\newtheorem*{example}{Example}

\theoremstyle{plain}
\newtheorem{theorem}{Theorem}[section]

\newenvironment{rcases}
  {\left.\begin{aligned}}
  {\end{aligned}\right\rbrace}

\newcommand{\id}{\textup{id}}
\newcommand{\im}{\textup{Im}\,}
\newcommand{\R}{\mathbf{R}}
\newcommand{\Z}{\mathbf{Z}}
\newcommand{\Q}{\mathbf{Q}}
\newcommand{\N}{\mathbf{N}}
\newcommand{\pt}{\textup{pt}}
\newcommand{\C}{\mathbf{C}}
\newcommand{\cone}{\textup{cone}}
\newcommand{\D}{\mathbf{D}}
\newcommand{\s}{\mathbf{S}}

\newcommand{\forceindent}{\leavevmode{\parindent=1.5em\indent}}

\begin{document}

\section*{Table of contents}

\contentsline{section}{\numberline{1}Fundamental Groups}{1}
\contentsline{subsection}{\numberline{1.1}Basic concepts}{1}
\forceindent\forceindent\forceindent Some background, fundamental groups, induced homomorphisms
\contentsline{subsection}{\numberline{1.2}Covering spaces}{3}
\forceindent\forceindent\forceindent Homotopy-lifting property, lifting criterion, universal covers, deck transformations
\contentsline{subsection}{\numberline{1.3}Calculation and application}{3}
\forceindent\forceindent\forceindent $\pi_1(\s^1)\cong\Z$ and its consequences, van Kampen's theorem

\contentsline{section}{\numberline{2}Homology}{5}
\contentsline{subsection}{\numberline{2.1}Basic concepts}{5}
\forceindent\forceindent\forceindent Simplicial homology, singular homology, cellular homology, axioms for homology
\contentsline{subsection}{\numberline{2.2}Tools}{6}
\forceindent\forceindent\forceindent Zig-zag lemma, five lemma, excision, Mayer-Vietoris sequence
\contentsline{subsection}{\numberline{2.3}Other topics}{8}
\forceindent\forceindent\forceindent Manifold and the degree of a map, Euler characteristics and Betti number, $\pi_1$ and $H_1$,\newline
\forceindent\forceindent\forceindent simplicial approximation

\contentsline{section}{\numberline{3}Cohomology}{5}

\newpage
\section{Fundamental Groups}

\subsection{Basic concepts}

\paragraph{Some background}

\paragraph{fundamental groups}

\paragraph{induced homomorphisms}

\subsection{Covering space}

\paragraph{Homotopy-lifting property}

\paragraph{Lifting criterion}

\paragraph{Universal covers}

\paragraph{Deck transformations}

\subsection{Calculation and applications}

\paragraph{$\pi_1(\s^1)\cong\Z$ and its consequences}

\paragraph{Van Kampen's theorem}

\newpage
\section{Homology}

Homology is a covariant functor from the category of topological spaces with continuous maps to the category of abelian groups with group homomorphisms, with chain complexes as an intermediate step. The central property of homology groups, like the fundamental groups, is the homotopy invariance. But homology groups are easier to compute than homotopy groups, in general.

\subsection{Basic concepts}

Simplicial homology is built from $\Delta$-complex structures and cellular homology is built upon CW complexes. These two homology theories are convenient in computation but require strict conditions on the space. Singular homology built from singular simplices is most used in proofs for its flexibility, but is less computable in general. 

\paragraph{Simplicial homology}

A $\Delta$-complex structure on space $X$ is a collection of maps $\sigma_n$ from the standard $n$-simplices $\Delta^n$ to $X$ that decomposes $X$ ``nicely'', with appropriate topology. The chain group is defined as $\Delta_n(X)=F^{ab}(\sigma_\alpha)$, and the boundary map is $\partial_n:\Delta_n(X)\to\Delta_{n-1}(X)$ by $\sigma_\alpha\mapsto\sum_i(-1)^i\sigma_\alpha|_{[v_0,\cdots,\hat{v}_i,\cdots,v_n]}$. Check that $\partial_n\partial_{n-1}=0$, and thus $\Delta_.(X)$ forms a chain complex. The corresponding homology groups $H_.^\Delta(X)$ are the simplicial homology groups.\medbreak

To calculate simplicial homology we need to subdivide for a $\Delta$-complex structure.
\[\mathbf{T}^2:\quad\quad\begin{tikzcd}[column sep=1.5cm,row sep=1.5cm]
v\arrow{r}{b}&v\\
v\arrow{u}{a}\arrow[swap]{r}{b}\arrow[ur,"U" near end,"L"' near start,"c"]&v\arrow[swap]{u}{a}
\end{tikzcd}\Longrightarrow\begin{cases}
\textrm{2-simplices: }\sigma_U,\sigma_L\\
\textrm{1-simplices: }\sigma_a,\sigma_b,\sigma_c\\
\textrm{0-simplices: }\sigma_v
\end{cases}\]
Thus we obtain the chain complex and the corresponding homology groups:
\[\begin{tikzcd}[column sep=scriptsize,row sep=scriptsize]
0\arrow{r}{\partial_3}&\Delta_2(\mathbf{T}^2)\arrow[equal]{d}\arrow{r}{\partial_2}&\Delta_1(\mathbf{T}^2)\arrow[equal]{d}\arrow{r}{\partial_1}&\Delta_0(\mathbf{T}^2)\arrow{r}{\partial_0}\arrow[equal]{d}&0\\
&\Z^2&\Z^3&\Z
\end{tikzcd}\Rightarrow\begin{cases}H_2^\Delta(\mathbf{T}^2)=\ker\partial_2\cong\Z\\
H_1^\Delta(\mathbf{T}^2)\cong\Z^3/\Z\cong\Z^2\\
H_0^\Delta(\mathbf{T}^2)=H_3^\Delta(\mathbf{T}^2)=0
\end{cases}.\]
Similarly, we can do this for $\mathbf{RP}^2$ and Klein bottle $K$.

\paragraph{Singular homology}

A singular $n$-simplex in $X$ is a map $\sigma:\Delta^n\to X$ that need not be ``nice''. The singular chain group $C_n(X)$ is the free abelian group generated by all singular $n$-simplices in $X$ with the same boundary map as before. Similarly $C_.(X)$ forms a chain complex with corresponding singular homology groups $H_.(X)$.\medbreak

For example, let $X$ be nonempty and path-connected, then define augmented map $\varepsilon:C_0(X)\to\Z$ by $\sum_in_i\sigma_i\mapsto\sum_in_i$, which is surjective since $X$ is nonempty. If we can show that $\ker\varepsilon=\im\partial_1$, then $H_0(X)=\ker\partial_0/\im\partial_1=C_0(X)/\ker\varepsilon\cong\Z$. $\im\partial_1\subseteq\ker\varepsilon$ is easy, for the reverse inclusion construct appropriate singular $1$-simplex.\medbreak

Sometimes we use reduced homology $\widetilde{H}$ by attaching the augmented map $\varepsilon$, so that $\widetilde{H}_n(X)=H_n(X)$ for $n>0$ and $\widetilde{H}_0(X)\oplus\Z=H_0(X)$. Now $\widetilde{H}_n(\pt)=0$ for all $n$.\medbreak

For $A\subseteq X$, define $C_n(X,A)=C_n(X)/C_n(A)$ to be the $n$-th relative chain group. With the naturally inherited boundary map we obtain a complex, and thus relative homology groups $H_.(X,A)$. It's not hard to see that $H_n(X,A)=\widetilde{H}_n(X,A)$ if $A$ is nonempty, and $H_n(X,A)=\widetilde{H}_n(X)$ for all $n$.\medbreak

Now we've reached the heart of homology: homotopy invariance. For $f:X\to Y$ there is a chain map $f_\#:C_.(X)\to C_.(Y)$ by $\sigma\mapsto f\sigma$, and since $f_\#(\partial\sigma)=\partial f_\#(\sigma)$, the chain ladder commutes. Two chain maps $f_\#$ and $g_\#$ are homotopic if there exists $h_n:C_n(X)\to C_{n+1}(Y)$ such that $f_\#-g_\#=\partial'_{n+1}h_n+h_{n-1}\partial_n$ for all $n$. Chain map $f_\#$ induces $f_\ast:H_.(X)\to H_.(Y)$. As a functors, $(fg)_\#=f_\#g_\#$, $\id_\#=\id$, and $(fg)_\ast=f_\ast g_\ast$, $\id_\ast=\id$.

\[\begin{tikzcd}[column sep=scriptsize]
\cdots\arrow{r}&C_n(X)\arrow{d}{h_n}\arrow{r}{\partial_n}&C_{n-1}(X)\arrow{d}{h_{n-1}}\arrow{r}&\cdots\\
\cdots\arrow{r}&C_n(Y)\arrow{r}{\partial_n}&C_{n-1}(Y)\arrow{r}&\cdots
\end{tikzcd}\quad\quad\quad\begin{tikzcd}
\cdots\arrow[r]&C_n(X)\arrow[d,xshift=0.5ex,"f_\#"'{xshift=-0.6ex}]\arrow[d,xshift=-0.5ex,"g_\#"{xshift=0.6ex}]\arrow[swap]{dl}{h_n}\arrow{r}{\partial_n}&C_{n-1}(X)\arrow{dl}{h_{n-1}}\\
C_{n+1}(Y)\arrow[swap]{r}{\partial'_{n+1}}&C_n(Y)\arrow[r]&\cdots
\end{tikzcd}\]

\begin{theorem}
Let $f,g:X\to Y$. If $f\simeq g$, then $f_\#\simeq g_\#$, and thus $f_\ast=g_\ast$.
\end{theorem}
\begin{proof}
We split the proof into two steps:\medbreak

\textit{$f\simeq g\Rightarrow f_\#\simeq g_\#$.} Let $i,i':X\hookrightarrow X\times[0,1]$ be the inclusions $x\mapsto(x,0)$ and $x\mapsto(x,1)$. Let $\Delta^n\times\{0\}=[u_0,\cdots,u_n]$ and $\Delta^n\times\{1\}=[v_0,\cdots,v_n]$, then we can subdivide $\Delta^n\times[0,1]$ into $n+1$ $(n+1)$-simplices of the form $s_i=[u_0,\cdots,u_i,v_i,\cdots,v_n]$. Define $P_n=\sum_i(-1)^is_i$, then by calculation $\partial P_n=\Delta^n\times\{1\}-\Delta^n\times\{0\}-P_n(\partial\Delta^n)$. Let $h_n:C_n(X)\to C_{n+1}(Y)$ by $\sigma\mapsto(\sigma\times\id)_\#(P_n)$ where $\sigma:\Delta^n\to X$. Check that $i_\#\simeq i'_\#$ under $h$. Finally, notice that if $f\simeq g$ by $F$, then $f=Hi$ and $g=Hi'$.\medbreak

\textit{$f_\#\simeq g_\#\Rightarrow f_\ast=g_\ast$.} Let $z\in\ker\partial_n$, then $f_\#(z)-g_\#(z)=(\partial'_{n+1}h_n+h_{n-1}\partial_n)(z)=(\partial'_{n+1}h_n)(z)$ which is in $\im\partial'_{n+1}$, hence $f_\ast([z])=[f_\#(z)]=[g_\#(z)]=g_\ast([z])$.
\end{proof}

$C_n(X)$ is a much bigger group then $\Delta_n(X)$, but in terms of homology they are equivalent:

\begin{theorem}
Let $X$ be equipped with a $\Delta$-complex structure, then the inclusion $\Delta_.(X)\hookrightarrow C_.(X)$ of chain complexes induces an isomorphism $H_.^\Delta(X)\cong H_.(X)$.
\end{theorem}

\paragraph{Cellular homology}

CW structure is less strict then $\Delta$-complex structure. Define cellular chain group to be $C_n^{CW}(X)=H_n(X^n,X^{n-1})=F^{ab}(n\textrm{-cells of }X)$, and with boundary map $d_n=j_{n-1}\partial_n$ we obtain a cellular chain complex:
\[\begin{tikzcd}[column sep=small]
&&0\arrow[d]&&&&&&\\
&&H_n(X^n)\arrow[rrd, "j_n"]\arrow[rr]&&H_n(X^{n+1})\cong H_n(X)\arrow[rr]&&0&&\\
\cdots\arrow[rr]&&{H_{n+1}(X^{n+1},X^n)}\arrow[u,"\partial_{n+1}"]\arrow[rr,"d_{n+1}"]&& {H_n(X^n,X^{n-1})}\arrow[rr, "d_n"] \arrow[d, "\partial_n"] &  & {H_{n-1}(X^{n-1},X^{n-2})} \arrow[rr]&&\cdots\\
&&&&H_{n-1}(X^{n-1})\arrow[rru,"j_{n-1}"']&&&&\\
&&&&0\arrow[u]&&&&       
\end{tikzcd}.\]
The corresponding homology groups $H_.^{CW}(X)$ are the cellular homology groups.\medbreak

Above we assumed the following observations when $X$ is a CW-complex:
\begin{enumerate}[(i)]
\item $H_n(X^k,X^{k-1})\cong\begin{cases}\bigoplus\Z\ (\textrm{one for each }n\textrm{-cell of }X),&n=k\\0,&\textrm{else}
\end{cases}$.

\item $i_\ast:H_k(X^k)\to H_k(X)$ induced by $i:X^k\hookrightarrow X$ is surjective.

\item $H_n(X^k)\cong\begin{cases}H_n(X)\ (\textrm{induced by }i:X^k\hookrightarrow X),&n<k\\0,&n>k
\end{cases}$.
\end{enumerate}
(i) follows from $H_n(X^k,X^{k-1})\cong\widetilde{H}_n(X^k/X^{k-1})\cong\widetilde{H}_n(\vee_i\mathbf{S}^k)$ since $(X^k,X^{k-1})$ is a good pair. (ii) and (iii) when $X$ is finite-dimensional follows from considering the exact sequence of $(X^k,X^{k-1})$ and induction on $H_k(X^0)=0$ when $k>0$. When $X$ is infinite-dimensional recall that\medbreak

Now we make explicit of the boundary maps $d_n$, with the help of mapping degree.\medbreak

Observe that $d_{n+1}:H_{n+1}(X^{n+1},X^n)\to H_n(X^n,X^{n-1})$ sends $e_\alpha^{n+1}\mapsto\sum d_{\alpha\beta}e^n_\beta$. The claim is that $d_{\alpha\beta}=\deg f_{\alpha\beta}$ where $f_{\alpha\beta}=q_\beta\circ\varphi_\alpha$, with $\varphi_\alpha:\partial\mathbf{D}^n\to X$ the attaching map and $q_\beta:X\to\mathbf{S}^n_\beta$ collapsing all of $X$ except for $e_\beta^n$.

\[\begin{tikzcd}
\widetilde{H}_{n+1}(\mathbf{D}_\alpha^{n+1},\partial\mathbf{D}^{n+1}_\alpha)\arrow{d}{\Phi_{\alpha\ast}}\arrow{r}{\partial}&\widetilde{H}_n(\partial\mathbf{D}_\alpha^{n+1})\arrow{d}{\varphi_{\alpha\ast}}\arrow{r}{f_{\alpha\beta\ast}}&\widetilde{H}_n(\mathbf{S}_\beta^n)\\
\widetilde{H}_{n+1}(X^{n+1},X^n)\arrow[swap]{drr}{d_{n+1}}\arrow{r}{\partial}&\widetilde{H}_n(X^n)\arrow{ur}{q_{\beta\ast}}\arrow{r}{q_{\beta_1\ast}}&\widetilde{H}_n(X^n/X^{n-1})\arrow[swap]{u}{q_{\beta_2\ast}}\\
&&\widetilde{H}_n(X^n,X^{n-1})\arrow[equal]{u}
\end{tikzcd}\]

Take a generator $[\mathbf{D}^{n+1}]\in\widetilde{H}_{n+1}(\mathbf{D}_\alpha^{n+1},\partial\mathbf{D}^{n+1}_\alpha)$, $\partial$ sends it to a generator in $\widetilde{H}_n(\partial\mathbf{D}_\alpha^{n+1})$ which has the image $\deg f_{\alpha\beta}$ under $f_{\alpha\beta\ast}$. On the other hand, the characteristic map (extension of attaching map) $\Phi_{\alpha\ast}$ sends $[\mathbf{D}^{n+1}]$ to $[e_\alpha^{n+1}]$, and $d_{n+1}$ further sends it to $\sum d_{\alpha\beta}e_\beta^n$, which projects to the $\beta^{\mathrm{th}}$ factor $d_{\alpha\beta}$ by $q_{\beta_2\ast}$. Since the diagram commutes, $d_{\alpha\beta}=\deg f_{\alpha\beta}$.

\begin{theorem}
The inclusion $C.^{CW}(x)\hookrightarrow C.(X)$ induces an isomorphism
$H_.^{CW}(X)\cong H.(X)$.
\end{theorem}
\begin{proof}
Since $j_n$ is injective, $\im\partial_{n+1}=\im d_{n+1}$ and $H_n(X^n)=\im j_n$. By exactness $\im j_n=\ker\partial_n$, and since $j_{n-1}$ is injective, $\ker\partial_n=\ker d_n$. Together we have $H_n^{CW}(X)\cong H_n(X^n)/\im\partial_{n+1}$, which by exactness is precisely $H_n(X)$.
\end{proof}

Now we no longer have to subdivide for a $\Delta$-complex structure like we did when computing the simplicial homology of $\mathbf{T}^2$.\medbreak

$\mathbf{M}_g$, $\mathbf{RP}^n$, for $\mathbf{CP}^n$: recall that $\mathbf{CP}^n=\mathbf{S}^{2n+1}/\sim$ where $v\sim\lambda v$ when $\abs{\lambda}=1$. The ``upper hemisphere'' of $\mathbf{CP}^n$ consists of points $(\omega,(1-\abs{\omega}^2)^{1/2})$ where $\omega\in\mathbf{C}^n\cong\mathbf{D}^{2n}$, the boundary of which corresponds to $(\omega,0)$ with the identification of $\mathbf{CP}^{n-1}$. Hence inductively $\mathbf{CP}^n$ is obtained from $\mathbf{CP}^{n-1}$ by attaching a $2n$-cell: $\mathbf{CP}^n=e_0\cup e_2\cup\cdots\cup e_{2n}$. Hence the cellular chain complex is an alternation between $0$ and $\Z$ with trivial boundary maps.
\[H_k(\mathbf{CP}^n)\cong\begin{cases}
\Z,&k=0,2,\cdots,2n\\0,&\textrm{else}
\end{cases}.\]

\paragraph{Axioms for homology}

In general a

\newpage
\subsection{Tools}

\paragraph{Zig-zag lemma}

The zig-zag lemma provides a way of constructing long exact sequence of homology groups from short exact sequence of chain complexes.

\begin{theorem}[zig-zag lemma]
A short exact sequence of chain complexes
\[\begin{tikzcd}[column sep=small]
&\vdots\arrow{d}&\vdots\arrow{d}&\vdots\arrow{d}\\
0\arrow{r}&C_n(A)\arrow{d}{\partial_A}\arrow{r}{i}&C_n(B)\arrow{d}{\partial_B}\arrow{r}{j}&C_n(C)\arrow{d}{\partial_C}\arrow{r}&0\\
0\arrow{r}&C_{n-1}(A)\arrow{d}\arrow{r}{i}&C_{n-1}(B)\arrow{d}\arrow{r}{j}&C_{n-1}(C)\arrow{d}\arrow{r}&0\\
&\vdots&\vdots&\vdots
\end{tikzcd}\]
induces a long exact sequence of homology groups
\[\begin{tikzcd}
\cdots \arrow[r]
& H_n(B) \arrow{r}{j_\ast}
\arrow[d, phantom, ""{coordinate, name=Z}]
& H_n(C) \arrow[dll,
"\partial",
rounded corners,
to path={ -- ([xshift=2ex]\tikztostart.east)
|- (Z) [near end]\tikztonodes
-| ([xshift=-2ex]\tikztotarget.west)
-- (\tikztotarget)}] \\
H_{n-1}(A) \arrow{r}{i_\ast}
& H_{n-1}(B) \arrow[r]
& \cdots
\end{tikzcd}.\]
\end{theorem}
\begin{proof}
We construct the connecting homomorphism $\partial:H_{n+1}(C.)\to H_n(A.)$ as follows. Choose $[c]\in H_{n+1}(C.)$, then $c\in C_{n+1}$ and $\partial_C c=0$. Since $j$ is surjective, there exists $b\in B_{n+1}$ such that $j(b)=c$ and $j(\partial_Bb)=\partial_Cj(b)=0$. Now since $\partial_Bb\in\ker j\cong\im i$, there exists $a\in A_n$ such that $i(a)=\partial_Bb$. Finally, $i(\partial_Aa)=\partial_Bi(a)=\partial_B\partial_Bb=0$, and since $i$ is injective, $\partial_Aa=0$. Hence let $\partial:[c]\mapsto[a]$, and check that it is well defined, i.e., the choices of $[c]$ and $b$ does not matter.
\[\begin{tikzcd}
&&B_{n+1}\arrow{d}{\partial}\arrow{r}{j}&C_{n+1}\arrow{d}{\partial}\arrow{r}&0\\
&A_n\arrow{d}{\partial}\arrow{r}{i}&B_n\arrow{d}{\partial}\arrow{r}{j}&C_n\\
0\arrow{r}&A_{n-1}\arrow{r}{i}&B_{n-1}
\end{tikzcd}.\]
Now what's left is to check for exactness by diagram chasing.
\end{proof}

The zig-zag lemma is most used in space tuples $(X,A)$:
\[\begin{tikzcd}
\cdots \arrow[r]
& H_n(X,A) \arrow{r}{j_\ast}
\arrow[d, phantom, ""{coordinate, name=Z}]
& H_n(X) \arrow[dll,
"\partial",
rounded corners,
to path={ -- ([xshift=2ex]\tikztostart.east)
|- (Z) [near end]\tikztonodes
-| ([xshift=-2ex]\tikztotarget.west)
-- (\tikztotarget)}] \\
H_{n-1}(A) \arrow{r}{i_\ast}
& H_{n-1}(X,A) \arrow[r]
& \cdots
\end{tikzcd},\]
where the connecting homomorphism corresponds with the boundary map: $\partial[\alpha]=[\partial\alpha]$.\medbreak

For example, one can easily compute
\[\widetilde{H}_k(\mathbf{D}^n,\partial\mathbf{D}^n)\cong \widetilde{H}_k(\s^{n-1})\cong\begin{cases}\Z,&k=n\\0,&\textrm{else}
\end{cases},\]
and from there deduce that $\R^m$ is not homeomorphic to $\R^n$. As expected, another nice corollary is the Brouwer fixed-point theorem: Every map $f:\mathbf{D}^n\to\mathbf{D}^n$ has a fixed point.
\[\begin{tikzcd}
\partial\mathbf{D}^n\arrow[swap]{dr}{\id}\arrow{r}{i}&\mathbf{D}^n\arrow{d}{r}\\
&\partial\mathbf{D}^n
\end{tikzcd}\Longrightarrow\begin{tikzcd}
\widetilde{H}_{n-1}(\mathbf{S}^{n-1})\cong\Z\arrow[swap]{dr}{\id}\arrow{r}{i_\ast}&\widetilde{H}_{n-1}(\mathbf{D}^n)\cong0\arrow{d}{r_\ast}\\
&\Z
\end{tikzcd}\]

\paragraph{Five lemma} Another elementary yet useful tool from homological algebra.

\begin{theorem}[five lemma]
In the diagram bellow if the rows are exact, then $\gamma$ is an isomorphism:
\[\begin{tikzcd}
A\arrow[r]\arrow[two heads]{d}{\alpha}&B\arrow[equal]{d}{\beta}\arrow[r]&C\arrow{d}{\gamma}\arrow[r]&D\arrow[equal]{d}{\delta}\arrow[r]&E\arrow[hook]{d}{\varepsilon}\\
A'\arrow[r]&B'\arrow[r]&C'\arrow[r]&D'\arrow[r]&E'
\end{tikzcd}.\]
\end{theorem}
\begin{proof}
$\beta,\delta$ surjective, $\varepsilon$ injective $\Rightarrow$ $\gamma$ surjective. $\beta,\delta$ injective, $\alpha$ surjective $\Rightarrow$ $\gamma$ injective.
\end{proof}

\paragraph{Excision} Excision is another fundamental property of homology, being one of the three axioms. It is also helpful in calculation. Excision theorem is remarkably easy when $X$ has a $\Delta$-complex structure with $A,X\setminus Z,A\setminus Z$ as $\Delta$-subcomplexes:
\[\begin{tikzcd}
\Delta_n(X\setminus Z)\arrow[bend right=25]{rr}{\varphi}\arrow{r}&\Delta_n(X)\arrow[r]&\Delta_n(X,A)
\end{tikzcd}\]
$\varphi$ is surjective since a basis of $\Delta_n(X,A)$ is given by subcomplexes of $X\setminus A\subseteq X\setminus Z$. Hence $\varphi$ induces an isomorphism $\Delta_n(X,A)\cong\Delta_n(X\setminus Z)/\ker\varphi=\Delta_n(X\setminus Z)/\Delta_n(A\setminus Z)=\Delta_n(X\setminus Z,A\setminus Z)$.

\begin{theorem}[excision theorem]
If $\overline{Z}\subseteq A^\circ$, then $H_n(X,A)\cong H_n(X\setminus Z,A\setminus Z)$ for all $n$.
\end{theorem}
\begin{proof}
Let $\mathcal{U}=\{U_\alpha\}_{\alpha\in A}$ be an open cover of $X$. Define $C_n^\mathcal{U}(X)\subseteq C_n(X)$ to be the subcomplex generated by $n$-simplices of $X$ such that $\sigma(\Delta^n)\subseteq U_\alpha$ for some $\alpha$. The boundary map inherited from $\partial:C_n(X)\to C_{n-1}(X)$ makes $(C.^\mathcal{U}(X),\partial)$ into a chain complex.\medbreak

By so-called barycentric subdivision $S:C.(X)\to C.(X)$ one can divide simplices so that each small simplex lies inside some $U_\alpha$, and by showing that $S$ is chain homotopic to the identity map, make sense of $H_n(C.^\mathcal{U}(X))\cong H_n(C.(X))$ for all $n$.\medbreak

Now let $Y=X\setminus Z$ and $\mathcal{U}=\{Y,A\}$. We have $C_n(X\setminus Z)/C_n(A\setminus Z)=C_n(Y)/C_n(Y\cap A)$, which is generated by the simplices that lie in $X\setminus A$. Hence $C_n(Y)/C_n(Y\cap A)\cong C_n^\mathcal{U}(X)/C_n(A)$. Now we have the exact sequences of homology by the zig-zag lemma:
\[\begin{tikzcd}[column sep=scriptsize]
H_n(C.(A))\arrow{d}{\cong}\arrow[r]&H_n(C.^\mathcal{U}(X))\arrow{d}{\cong}\arrow[r]&H_n(C.^\mathcal{U}(X)/C.(A))\arrow[d]\arrow[r]&H_{n-1}(C.(A))\arrow{d}{\cong}\arrow[r]&H_{n-1}(C.^\mathcal{U}(X))\arrow{d}{\cong}\\
H_n(C.(A))\arrow[r]&H_n(C.(X))\arrow[r]&H_n(C.(X)/C.(A))\arrow[r]&H_{n-1}(C.(A))\arrow[r]&H_{n-1}(C.(X))
\end{tikzcd}.\]
By the five lemma $H_n(C.^\mathcal{U}(X)/C.(A))\cong H_n(C.(X)/C.(A))$. Hence $H_n(X\setminus Z,A\setminus Z)\cong H_n(X,A)$.
\end{proof}

The reason cone, 2.25, generated, wedge inclusion, etc

\paragraph{Mayer-Vietoris sequence}

\subsection{Other topics}

\paragraph{Manifold and the degree of a map}

\paragraph{Euler characteristics and Betti number}

\paragraph{$\pi_1$ and $H_1$}

\paragraph{Simplicial approximation}




\end{document}