\documentclass[11pt]{article}

\usepackage[margin=1in]{geometry}
\usepackage{amsmath,amsthm,amsfonts}
\usepackage[utf8]{inputenc}
\usepackage{amssymb}
\usepackage[mathscr]{eucal}
\usepackage{graphicx}
\usepackage{listings}
\usepackage{titlesec}
\usepackage{tikz-cd}
\usepackage[shortlabels]{enumitem}
\usepackage{tikz}
\usepackage{xcolor}
\usepackage[OT1]{fontenc}
\usepackage{physics}
\usepackage{xpatch}
%\usepackage{hyperref}

\setlength\parindent{0pt}

\setcounter{section}{-0}
\setcounter{secnumdepth}{2}

\theoremstyle{definition}
\newtheorem*{defin}{Definition}
\newtheorem*{example}{Example}

\theoremstyle{plain}
\newtheorem{theorem}{Theorem}[section]

\newenvironment{rcases}
  {\left.\begin{aligned}}
  {\end{aligned}\right\rbrace}

\newcommand{\id}{\textup{id}}
\newcommand{\im}{\textup{Im}\,}
\newcommand{\R}{\mathbf{R}}
\newcommand{\Z}{\mathbf{Z}}
\newcommand{\Q}{\mathbf{Q}}
\newcommand{\N}{\mathbf{N}}
\newcommand{\pt}{\textup{pt}}
\newcommand{\Hom}{\textup{Hom}}
\newcommand{\Ext}{\textup{Ext}}
\DeclareMathOperator{\coker}{coker}
\DeclareMathOperator{\Char}{char}
\newcommand{\1}{\mathbf{1}}
\newcommand{\C}{\mathbf{C}}
\newcommand{\cone}{\textup{Cone}\,}
\newcommand{\D}{\mathbf{D}}
\newcommand{\s}{\mathbf{S}}

\newcommand{\forceindent}{\leavevmode{\parindent=1.5em\indent}}

\begin{document}

\section*{Table of contents}

\contentsline{section}{\numberline{1}Fundamental Groups}{1}
\contentsline{subsection}{\numberline{1.1}Basic concepts}{1}
\forceindent\forceindent\forceindent Some background, fundamental groups, induced homomorphisms
\contentsline{subsection}{\numberline{1.2}Covering spaces}{3}
\forceindent\forceindent\forceindent Homotopy-lifting property, lifting criterion, universal covers, deck transformations
\contentsline{subsection}{\numberline{1.3}Calculation and application}{3}
\forceindent\forceindent\forceindent $\pi_1(\s^1)\cong\Z$ and its consequences, van Kampen's theorem

\contentsline{section}{\numberline{2}Homology}{5}
\contentsline{subsection}{\numberline{2.1}Basic concepts}{5}
\forceindent\forceindent\forceindent Simplicial homology, singular homology, cellular homology, axioms for homology
\contentsline{subsection}{\numberline{2.2}Tools}{6}
\forceindent\forceindent\forceindent Zig-zag lemma, five lemma, excision, Mayer-Vietoris sequence
\contentsline{subsection}{\numberline{2.3}Other topics}{8}
\forceindent\forceindent\forceindent Manifold and the degree of a map, Euler characteristics and Betti number, $\pi_1$ and $H_1$,\newline
\forceindent\forceindent\forceindent simplicial approximation

\contentsline{section}{\numberline{3}Cohomology}{5}

\newpage
\section{Fundamental Groups}

\subsection{Basic concepts}

\paragraph{Some background}

\paragraph{fundamental groups}

\paragraph{induced homomorphisms}

\subsection{Covering space}

\paragraph{Homotopy-lifting property}

\paragraph{Lifting criterion}

\paragraph{Universal covers}

\paragraph{Deck transformations}

\subsection{Calculation and applications}

\paragraph{$\pi_1(\s^1)\cong\Z$ and its consequences}

\paragraph{Van Kampen's theorem}

\newpage
\section{Homology}

Homology is a covariant functor from the category of topological spaces with continuous maps to the category of abelian groups with group homomorphisms, with chain complexes as an intermediate step. The central property of homology groups, like the fundamental groups, is the homotopy invariance. But homology groups are easier to compute than homotopy groups, in general.

\subsection{Basic concepts}

Simplicial homology is built from $\Delta$-complex structures and cellular homology is built upon CW complexes. These two homology theories are convenient in computation but require strict conditions on the space. Singular homology built from singular simplices is most used in proofs for its flexibility, but is less computable in general. 

\paragraph{Simplicial homology}

A $\Delta$-complex structure on space $X$ is a collection of maps $\sigma_n$ from the standard $n$-simplices $\Delta^n$ to $X$ that decomposes $X$ ``nicely'', with appropriate topology. The chain group is defined as $\Delta_n(X)=F^{ab}(\sigma_\alpha)$, and the boundary map is $\partial_n:\Delta_n(X)\to\Delta_{n-1}(X)$ by $\sigma_\alpha\mapsto\sum_i(-1)^i\sigma_\alpha|_{[v_0,\cdots,\hat{v}_i,\cdots,v_n]}$. Check that $\partial_n\partial_{n-1}=0$, and thus $\Delta_.(X)$ forms a chain complex. The corresponding homology groups $H_.^\Delta(X)$ are the simplicial homology groups.\medbreak

To calculate simplicial homology we need to subdivide for a $\Delta$-complex structure.
\[\mathbf{T}^2:\quad\quad\begin{tikzcd}[column sep=1.5cm,row sep=1.5cm]
v\arrow{r}{b}&v\\
v\arrow{u}{a}\arrow[swap]{r}{b}\arrow[ur,"U" near end,"L"' near start,"c"]&v\arrow[swap]{u}{a}
\end{tikzcd}\Longrightarrow\begin{cases}
\textrm{2-simplices: }\sigma_U,\sigma_L\\
\textrm{1-simplices: }\sigma_a,\sigma_b,\sigma_c\\
\textrm{0-simplices: }\sigma_v
\end{cases}\]
Thus we obtain the chain complex and the corresponding homology groups:
\[\begin{tikzcd}[column sep=scriptsize,row sep=scriptsize]
0\arrow{r}{\partial_3}&\Delta_2(\mathbf{T}^2)\arrow[equal]{d}\arrow{r}{\partial_2}&\Delta_1(\mathbf{T}^2)\arrow[equal]{d}\arrow{r}{\partial_1}&\Delta_0(\mathbf{T}^2)\arrow{r}{\partial_0}\arrow[equal]{d}&0\\
&\Z^2&\Z^3&\Z
\end{tikzcd}\Rightarrow\begin{cases}H_2^\Delta(\mathbf{T}^2)=\ker\partial_2\cong\Z\\
H_1^\Delta(\mathbf{T}^2)\cong\Z^3/\Z\cong\Z^2\\
H_0^\Delta(\mathbf{T}^2)=H_3^\Delta(\mathbf{T}^2)=0
\end{cases}.\]
Similarly, we can do this for $\mathbf{RP}^2$ and Klein bottle $K$.

\paragraph{Singular homology}

A singular $n$-simplex in $X$ is a map $\sigma:\Delta^n\to X$ that need not be ``nice''. The singular chain group $C_n(X)$ is the free abelian group generated by all singular $n$-simplices in $X$ with the same boundary map as before. Similarly $C_.(X)$ forms a chain complex with corresponding singular homology groups $H_.(X)$.\medbreak

For example, let $X$ be nonempty and path-connected, then define augmented map $\varepsilon:C_0(X)\to\Z$ by $\sum_in_i\sigma_i\mapsto\sum_in_i$, which is surjective since $X$ is nonempty. If we can show that $\ker\varepsilon=\im\partial_1$, then $H_0(X)=\ker\partial_0/\im\partial_1=C_0(X)/\ker\varepsilon\cong\Z$. $\im\partial_1\subseteq\ker\varepsilon$ is easy, for the reverse inclusion construct appropriate singular $1$-simplex.\medbreak

Sometimes we use reduced homology $\widetilde{H}$ by attaching the augmented map $\varepsilon$, so that $\widetilde{H}_n(X)=H_n(X)$ for $n>0$ and $\widetilde{H}_0(X)\oplus\Z=H_0(X)$. Now $\widetilde{H}_n(\pt)=0$ for all $n$.\medbreak

For $A\subseteq X$, define $C_n(X,A)=C_n(X)/C_n(A)$ to be the $n$-th relative chain group. With the naturally inherited boundary map we obtain a complex, and thus relative homology groups $H_.(X,A)$. It's not hard to see that $H_n(X,A)=\widetilde{H}_n(X,A)$ if $A$ is nonempty, and $H_n(X,A)=\widetilde{H}_n(X)$ for all $n$.\medbreak

Now we've reached the heart of homology: homotopy invariance. For $f:X\to Y$ there is a chain map $f_\#:C_.(X)\to C_.(Y)$ by $\sigma\mapsto f\sigma$, and since $f_\#(\partial\sigma)=\partial f_\#(\sigma)$, the chain ladder commutes. Two chain maps $f_\#$ and $g_\#$ are homotopic if there exists $h_n:C_n(X)\to C_{n+1}(Y)$ such that $f_\#-g_\#=\partial'_{n+1}h_n+h_{n-1}\partial_n$ for all $n$. Chain map $f_\#$ induces $f_\ast:H_.(X)\to H_.(Y)$. As a functors, $(fg)_\#=f_\#g_\#$, $\id_\#=\id$, and $(fg)_\ast=f_\ast g_\ast$, $\id_\ast=\id$.

\[\begin{tikzcd}[column sep=scriptsize]
\cdots\arrow{r}&C_n(X)\arrow{d}{h_n}\arrow{r}{\partial_n}&C_{n-1}(X)\arrow{d}{h_{n-1}}\arrow{r}&\cdots\\
\cdots\arrow{r}&C_n(Y)\arrow{r}{\partial_n}&C_{n-1}(Y)\arrow{r}&\cdots
\end{tikzcd}\quad\quad\quad\begin{tikzcd}
\cdots\arrow[r]&C_n(X)\arrow[d,xshift=0.5ex,"f_\#"'{xshift=-0.6ex}]\arrow[d,xshift=-0.5ex,"g_\#"{xshift=0.6ex}]\arrow[swap]{dl}{h_n}\arrow{r}{\partial_n}&C_{n-1}(X)\arrow{dl}{h_{n-1}}\\
C_{n+1}(Y)\arrow[swap]{r}{\partial'_{n+1}}&C_n(Y)\arrow[r]&\cdots
\end{tikzcd}\]

\begin{theorem}
Let $f,g:X\to Y$. If $f\simeq g$, then $f_\#\simeq g_\#$, and thus $f_\ast=g_\ast$.
\end{theorem}
\begin{proof}
We split the proof into two steps:\medbreak

\textit{$f\simeq g\Rightarrow f_\#\simeq g_\#$.} Let $i,i':X\hookrightarrow X\times[0,1]$ be the inclusions $x\mapsto(x,0)$ and $x\mapsto(x,1)$. Let $\Delta^n\times\{0\}=[u_0,\cdots,u_n]$ and $\Delta^n\times\{1\}=[v_0,\cdots,v_n]$, then we can subdivide $\Delta^n\times[0,1]$ into $n+1$ $(n+1)$-simplices of the form $s_i=[u_0,\cdots,u_i,v_i,\cdots,v_n]$. Define $P_n=\sum_i(-1)^is_i$, then by calculation $\partial P_n=\Delta^n\times\{1\}-\Delta^n\times\{0\}-P_n(\partial\Delta^n)$. Let $h_n:C_n(X)\to C_{n+1}(Y)$ by $\sigma\mapsto(\sigma\times\id)_\#(P_n)$ where $\sigma:\Delta^n\to X$. Check that $i_\#\simeq i'_\#$ under $h$. Finally, notice that if $f\simeq g$ by $F$, then $f=Hi$ and $g=Hi'$.\medbreak

\textit{$f_\#\simeq g_\#\Rightarrow f_\ast=g_\ast$.} Let $z\in\ker\partial_n$, then $f_\#(z)-g_\#(z)=(\partial'_{n+1}h_n+h_{n-1}\partial_n)(z)=(\partial'_{n+1}h_n)(z)$ which is in $\im\partial'_{n+1}$, hence $f_\ast([z])=[f_\#(z)]=[g_\#(z)]=g_\ast([z])$.
\end{proof}

$C_n(X)$ is a much bigger group then $\Delta_n(X)$, but in terms of homology they are equivalent:

\begin{theorem}
Let $X$ be equipped with a $\Delta$-complex structure, then the inclusion $\Delta_.(X)\hookrightarrow C_.(X)$ of chain complexes induces an isomorphism $H_.^\Delta(X)\cong H_.(X)$.
\end{theorem}
\begin{proof}
If $X$ is finite dimensional, then with excision and the fact that $H_n(\Delta^n,\partial\Delta^n)\cong\Z$ is generated by identity maps $\id_n:\Delta^n\to\Delta^n$, we find the same description of $H_n^\Delta(X^k,X^{k-1})$ and $H_n(X^k,X^{k-1})$: free abelian with basis the $k$-simplices of $X$ when $n=k$ and trivial otherwise.
\[\begin{tikzcd}[column sep=scriptsize]
H^\Delta_{n+1}(X^k,X^{k-1})\arrow[d,equal]\arrow[r]&H^\Delta_n(X^{k-1})\arrow[d,equal,"\textrm{induction}"']\arrow[r]&H^\Delta_n(X^k)\arrow[d]\arrow[r]&H^\Delta_n(X^k,X^{k-1})\arrow[d,equal]\arrow[r]&H^\Delta_{n-1}(X^{k-1})\arrow[d,equal,"\textrm{induction}"']\\
H_{n+1}(X^k,X^{k-1})\arrow[r]&H_n(X^{k-1})\arrow[r]&H_n(X^k)\arrow[r]&H_n(X^k,X^{k-1})\arrow[r]&H_{n-1}(X^{k-1})
\end{tikzcd}\]
Apply the five lemma to the diagram above. If $X$ is infinite-dimensional, we use the \textcolor{red}{compactness argument}
\end{proof}

\paragraph{Cellular homology}

CW structure is less strict then $\Delta$-complex structure. Define cellular chain group to be $C_n^{CW}(X)=H_n(X^n,X^{n-1})=F^{ab}(n\textrm{-cells of }X)$, and with boundary map $d_n=j_{n-1}\partial_n$ we obtain a cellular chain complex:
\[\begin{tikzcd}[column sep=small]
&&0\arrow[d]&&&&&&\\
&&H_n(X^n)\arrow[rrd, "j_n"]\arrow[rr]&&H_n(X^{n+1})\cong H_n(X)\arrow[rr]&&0&&\\
\cdots\arrow[rr]&&{H_{n+1}(X^{n+1},X^n)}\arrow[u,"\partial_{n+1}"]\arrow[rr,"d_{n+1}"]&& {H_n(X^n,X^{n-1})}\arrow[rr, "d_n"] \arrow[d, "\partial_n"] &  & {H_{n-1}(X^{n-1},X^{n-2})} \arrow[rr]&&\cdots\\
&&&&H_{n-1}(X^{n-1})\arrow[rru,"j_{n-1}"']&&&&\\
&&&&0\arrow[u]&&&&       
\end{tikzcd}.\]
The corresponding homology groups $H_.^{CW}(X)$ are the cellular homology groups.\medbreak

Above we assumed the following observations when $X$ is a CW-complex:
\begin{enumerate}[(i)]
\item $H_n(X^k,X^{k-1})\cong\begin{cases}\bigoplus\Z\ (\textrm{one for each }n\textrm{-cell of }X),&n=k\\0,&\textrm{else}
\end{cases}$.

\item $i_\ast:H_k(X^k)\to H_k(X)$ induced by $i:X^k\hookrightarrow X$ is surjective.

\item $H_n(X^k)\cong\begin{cases}H_n(X)\ (\textrm{induced by }i:X^k\hookrightarrow X),&n<k\\0,&n>k
\end{cases}$.
\end{enumerate}
(i) follows from $H_n(X^k,X^{k-1})\cong\widetilde{H}_n(X^k/X^{k-1})\cong\widetilde{H}_n(\vee_i\mathbf{S}^k)$ since $(X^k,X^{k-1})$ is a good pair. (ii) and (iii) when $X$ is finite-dimensional follows from considering the exact sequence of $(X^k,X^{k-1})$ and induction on $H_k(X^0)=0$ when $k>0$. When $X$ is infinite-dimensional recall that\medbreak

Now we make explicit of the boundary maps $d_n$, with the help of mapping degree.\medbreak

Observe that $d_{n+1}:H_{n+1}(X^{n+1},X^n)\to H_n(X^n,X^{n-1})$ sends $e_\alpha^{n+1}\mapsto\sum d_{\alpha\beta}e^n_\beta$. The claim is that $d_{\alpha\beta}=\deg f_{\alpha\beta}$ where $f_{\alpha\beta}=q_\beta\circ\varphi_\alpha$, with $\varphi_\alpha:\partial\mathbf{D}^n\to X$ the attaching map and $q_\beta:X\to\mathbf{S}^n_\beta$ collapsing all of $X$ except for $e_\beta^n$.

\[\begin{tikzcd}
\widetilde{H}_{n+1}(\mathbf{D}_\alpha^{n+1},\partial\mathbf{D}^{n+1}_\alpha)\arrow{d}{\Phi_{\alpha\ast}}\arrow{r}{\partial}&\widetilde{H}_n(\partial\mathbf{D}_\alpha^{n+1})\arrow{d}{\varphi_{\alpha\ast}}\arrow{r}{f_{\alpha\beta\ast}}&\widetilde{H}_n(\mathbf{S}_\beta^n)\\
\widetilde{H}_{n+1}(X^{n+1},X^n)\arrow[swap]{drr}{d_{n+1}}\arrow{r}{\partial}&\widetilde{H}_n(X^n)\arrow{ur}{q_{\beta\ast}}\arrow{r}{q_{\beta_1\ast}}&\widetilde{H}_n(X^n/X^{n-1})\arrow[swap]{u}{q_{\beta_2\ast}}\\
&&\widetilde{H}_n(X^n,X^{n-1})\arrow[equal]{u}
\end{tikzcd}\]

Take a generator $[\mathbf{D}^{n+1}]\in\widetilde{H}_{n+1}(\mathbf{D}_\alpha^{n+1},\partial\mathbf{D}^{n+1}_\alpha)$, $\partial$ sends it to a generator in $\widetilde{H}_n(\partial\mathbf{D}_\alpha^{n+1})$ which has the image $\deg f_{\alpha\beta}$ under $f_{\alpha\beta\ast}$. On the other hand, the characteristic map (extension of attaching map) $\Phi_{\alpha\ast}$ sends $[\mathbf{D}^{n+1}]$ to $[e_\alpha^{n+1}]$, and $d_{n+1}$ further sends it to $\sum d_{\alpha\beta}e_\beta^n$, which projects to the $\beta^{\mathrm{th}}$ factor $d_{\alpha\beta}$ by $q_{\beta_2\ast}$. Since the diagram commutes, $d_{\alpha\beta}=\deg f_{\alpha\beta}$.

\begin{theorem}
The inclusion $C.^{CW}(x)\hookrightarrow C.(X)$ induces an isomorphism
$H_.^{CW}(X)\cong H.(X)$.
\end{theorem}
\begin{proof}
Since $j_n$ is injective, $\im\partial_{n+1}=\im d_{n+1}$ and $H_n(X^n)=\im j_n$. By exactness $\im j_n=\ker\partial_n$, and since $j_{n-1}$ is injective, $\ker\partial_n=\ker d_n$. Together we have $H_n^{CW}(X)\cong H_n(X^n)/\im\partial_{n+1}$, which by exactness is precisely $H_n(X)$.
\end{proof}

Now we no longer have to subdivide for a $\Delta$-complex structure like we did when computing the simplicial homology of $\mathbf{T}^2$.\medbreak

$\mathbf{M}_g$, $\mathbf{RP}^n$, for $\mathbf{CP}^n$: recall that $\mathbf{CP}^n=\mathbf{S}^{2n+1}/\sim$ where $v\sim\lambda v$ when $\abs{\lambda}=1$. The ``upper hemisphere'' of $\mathbf{CP}^n$ consists of points $(\omega,(1-\abs{\omega}^2)^{1/2})$ where $\omega\in\mathbf{C}^n\cong\mathbf{D}^{2n}$, the boundary of which corresponds to $(\omega,0)$ with the identification of $\mathbf{CP}^{n-1}$. Hence inductively $\mathbf{CP}^n$ is obtained from $\mathbf{CP}^{n-1}$ by attaching a $2n$-cell: $\mathbf{CP}^n=e_0\cup e_2\cup\cdots\cup e_{2n}$. Hence the cellular chain complex is an alternation between $0$ and $\Z$ with trivial boundary maps.
\[H_k(\mathbf{CP}^n)\cong\begin{cases}
\Z,&k=0,2,\cdots,2n\\0,&\textrm{else}
\end{cases}.\]

\paragraph{Axioms for homology}

In general a

\newpage
\subsection{Tools}

\paragraph{Zig-zag lemma}

The zig-zag lemma provides a way of constructing long exact sequence of homology groups from short exact sequence of chain complexes.

\begin{theorem}[zig-zag lemma]
A short exact sequence of chain complexes
\[\begin{tikzcd}[column sep=small]
&\vdots\arrow{d}&\vdots\arrow{d}&\vdots\arrow{d}\\
0\arrow{r}&C_n(A)\arrow{d}{\partial_A}\arrow{r}{i}&C_n(B)\arrow{d}{\partial_B}\arrow{r}{j}&C_n(C)\arrow{d}{\partial_C}\arrow{r}&0\\
0\arrow{r}&C_{n-1}(A)\arrow{d}\arrow{r}{i}&C_{n-1}(B)\arrow{d}\arrow{r}{j}&C_{n-1}(C)\arrow{d}\arrow{r}&0\\
&\vdots&\vdots&\vdots
\end{tikzcd}\]
induces a long exact sequence of homology groups
\[\begin{tikzcd}
\cdots \arrow[r]
& H_n(B) \arrow{r}{j_\ast}
\arrow[d, phantom, ""{coordinate, name=Z}]
& H_n(C) \arrow[dll,
"\partial",
rounded corners,
to path={ -- ([xshift=2ex]\tikztostart.east)
|- (Z) [near end]\tikztonodes
-| ([xshift=-2ex]\tikztotarget.west)
-- (\tikztotarget)}] \\
H_{n-1}(A) \arrow{r}{i_\ast}
& H_{n-1}(B) \arrow[r]
& \cdots
\end{tikzcd}.\]
\end{theorem}
\begin{proof}
We construct the connecting homomorphism $\partial:H_{n+1}(C.)\to H_n(A.)$ as follows. Choose $[c]\in H_{n+1}(C.)$, then $c\in C_{n+1}$ and $\partial_C c=0$. Since $j$ is surjective, there exists $b\in B_{n+1}$ such that $j(b)=c$ and $j(\partial_Bb)=\partial_Cj(b)=0$. Now since $\partial_Bb\in\ker j\cong\im i$, there exists $a\in A_n$ such that $i(a)=\partial_Bb$. Finally, $i(\partial_Aa)=\partial_Bi(a)=\partial_B\partial_Bb=0$, and since $i$ is injective, $\partial_Aa=0$. Hence let $\partial:[c]\mapsto[a]$, and check that it is well defined, i.e., the choices of $[c]$ and $b$ does not matter.
\[\begin{tikzcd}
&&B_{n+1}\arrow{d}{\partial}\arrow{r}{j}&C_{n+1}\arrow{d}{\partial}\arrow{r}&0\\
&A_n\arrow{d}{\partial}\arrow{r}{i}&B_n\arrow{d}{\partial}\arrow{r}{j}&C_n\\
0\arrow{r}&A_{n-1}\arrow{r}{i}&B_{n-1}
\end{tikzcd}.\]
Now what's left is to check for exactness by diagram chasing.
\end{proof}

The zig-zag lemma is most used in space pairs $(X,A)$:
\[\begin{tikzcd}
\cdots \arrow[r]
& H_n(X,A) \arrow{r}{j_\ast}
\arrow[d, phantom, ""{coordinate, name=Z}]
& H_n(X) \arrow[dll,
"\partial",
rounded corners,
to path={ -- ([xshift=2ex]\tikztostart.east)
|- (Z) [near end]\tikztonodes
-| ([xshift=-2ex]\tikztotarget.west)
-- (\tikztotarget)}] \\
H_{n-1}(A) \arrow{r}{i_\ast}
& H_{n-1}(X,A) \arrow[r]
& \cdots
\end{tikzcd},\]
where the connecting homomorphism corresponds with the boundary map: $\partial[\alpha]=[\partial\alpha]$.\medbreak

For example, one can easily compute
\[\widetilde{H}_k(\mathbf{D}^n,\partial\mathbf{D}^n)\cong \widetilde{H}_k(\s^{n-1})\cong\begin{cases}\Z,&k=n\\0,&\textrm{else}
\end{cases},\]
and deduce therefore the Brouwer fixed-point theorem: Every map $f:\mathbf{D}^n\to\mathbf{D}^n$ has a fixed point.
\[\begin{tikzcd}
\partial\mathbf{D}^n\arrow[swap]{dr}{\id}\arrow{r}{i}&\mathbf{D}^n\arrow{d}{r}\\
&\partial\mathbf{D}^n
\end{tikzcd}\Longrightarrow\begin{tikzcd}
\widetilde{H}_{n-1}(\mathbf{S}^{n-1})\cong\Z\arrow[swap]{dr}{\id}\arrow{r}{i_\ast}&\widetilde{H}_{n-1}(\mathbf{D}^n)\cong0\arrow{d}{r_\ast}\\
&\Z
\end{tikzcd}\]

\paragraph{Five lemma} Another elementary yet useful tool from homological algebra.

\begin{theorem}[five lemma]
In the diagram bellow if the rows are exact, then $\gamma$ is an isomorphism:
\[\begin{tikzcd}
A\arrow[r]\arrow[two heads]{d}{\alpha}&B\arrow[equal]{d}{\beta}\arrow[r]&C\arrow{d}{\gamma}\arrow[r]&D\arrow[equal]{d}{\delta}\arrow[r]&E\arrow[hook]{d}{\varepsilon}\\
A'\arrow[r]&B'\arrow[r]&C'\arrow[r]&D'\arrow[r]&E'
\end{tikzcd}.\]
\end{theorem}
\begin{proof}
$\beta,\delta$ surjective, $\varepsilon$ injective $\Rightarrow$ $\gamma$ surjective. $\beta,\delta$ injective, $\alpha$ surjective $\Rightarrow$ $\gamma$ injective.
\end{proof}

\paragraph{Excision} Excision is another fundamental property of homology, being one of the three axioms. It is also helpful in calculation. Excision theorem is remarkably easy when $X$ has a $\Delta$-complex structure with $A,X\setminus Z,A\setminus Z$ as $\Delta$-subcomplexes:
\[\begin{tikzcd}
\Delta_n(X\setminus Z)\arrow[bend right=25]{rr}{\varphi}\arrow{r}&\Delta_n(X)\arrow[r]&\Delta_n(X,A)
\end{tikzcd}\]
$\varphi$ is surjective since a basis of $\Delta_n(X,A)$ is given by subcomplexes of $X\setminus A\subseteq X\setminus Z$. Hence $\varphi$ induces an isomorphism $\Delta_n(X,A)\cong\Delta_n(X\setminus Z)/\ker\varphi=\Delta_n(X\setminus Z)/\Delta_n(A\setminus Z)=\Delta_n(X\setminus Z,A\setminus Z)$.

\begin{theorem}[excision theorem]
If $\overline{Z}\subseteq A^\circ$, then $H_n(X,A)\cong H_n(X\setminus Z,A\setminus Z)$ for all $n$.
\end{theorem}
\begin{proof}
Let $\mathcal{U}=\{U_\alpha\}_{\alpha\in A}$ be an open cover of $X$. Define $C_n^\mathcal{U}(X)\subseteq C_n(X)$ to be the subcomplex generated by $n$-simplices of $X$ such that $\sigma(\Delta^n)\subseteq U_\alpha$ for some $\alpha$. The boundary map inherited from $\partial:C_n(X)\to C_{n-1}(X)$ makes $(C.^\mathcal{U}(X),\partial)$ into a chain complex.\medbreak

By so-called barycentric subdivision $S:C.(X)\to C.(X)$ one can divide simplices so that each small simplex lies inside some $U_\alpha$, and by showing that $S$ is chain homotopic to the identity map, make sense of $H_n(C.^\mathcal{U}(X))\cong H_n(C.(X))$ for all $n$.\medbreak

Now let $Y=X\setminus Z$ and $\mathcal{U}=\{Y,A\}$. We have $C_n(X\setminus Z)/C_n(A\setminus Z)=C_n(Y)/C_n(Y\cap A)$, which is generated by the simplices that lie in $X\setminus A$. Hence $C_n(Y)/C_n(Y\cap A)\cong C_n^\mathcal{U}(X)/C_n(A)$. Now we have the exact sequences of homology by the zig-zag lemma:
\[\begin{tikzcd}[column sep=scriptsize]
H_n(C.(A))\arrow[equal]{d}\arrow[r]&H_n(C.^\mathcal{U}(X))\arrow[equal]{d}\arrow[r]&H_n(C.^\mathcal{U}(X)/C.(A))\arrow[d]\arrow[r]&H_{n-1}(C.(A))\arrow[equal]{d}\arrow[r]&H_{n-1}(C.^\mathcal{U}(X))\arrow[equal]{d}\\
H_n(C.(A))\arrow[r]&H_n(C.(X))\arrow[r]&H_n(C.(X)/C.(A))\arrow[r]&H_{n-1}(C.(A))\arrow[r]&H_{n-1}(C.(X))
\end{tikzcd}.\]
By the five lemma $H_n(C.^\mathcal{U}(X)/C.(A))\cong H_n(C.(X)/C.(A))$. Hence $H_n(X\setminus Z,A\setminus Z)\cong H_n(X,A)$.
\end{proof}

Equivalently, excision theorem says that if $X=A^\circ\cup B^\circ$, then $(B,A\cap B)\hookrightarrow(X,A)$ induces $H_n(B,A\cap B)\cong H_n(X,A)$ for all $n$. To see this let $Z=X\setminus B$ and $B=X\setminus Z$ for the converse.\medbreak

Excision theorem tells us that for a ``good'' pair $(X,A)$, i.e., $A\subseteq X$ is closed, nonempty, and deformation retract to some some neighborhood $V\subseteq X$ of $A$ (e.g., CW pairs are ``good"), then $H_n(X,A)\cong\widetilde{H}_n(X/A)$. Otherwise $\widetilde{H}_n(X,A)\cong H_n(X\cup\cone_A,\cone_A)\cong\widetilde{H}_n(X\cup\cone_A)$ is always true for arbitrary pairs, where the first isomorphism is obtained by excising the cone tip.
\[\begin{tikzcd}
H_n(X,A)\arrow[equal,r,"\textrm{excision}"]\arrow[d]&H_n(X\setminus A,V\setminus A)\cong H_n(X,V)\arrow[d,equal]\\
H_n(X/A,A/A)\arrow[r,equal]&H_n(X/A\setminus A/A,V/A\setminus A/A)\cong H_n(X/A,V/A)
\end{tikzcd}\]

For example, $H_n(\Delta^n,\partial\Delta^n)\cong\Z$ is generated by identity maps $\id_n:\Delta^n\to\Delta^n$, a key ingredient in the proof of Theorem 2.2. When $n=0$ the statement is trivial. We proceed by induction on $n$. Let $\Lambda^n$ be $\partial\Delta^n$ setminus the last face (in the case of a triangle $\Delta^2$, $\Lambda^2$ looks exactly like a triangle without the bottom side) and consider the exact sequence of the triple $(\Delta^{n+1},\partial\Delta^{n+1},\Lambda^{n+1})$
\[\begin{tikzcd}[column sep=small]
H_{n+1}(\Delta^{n+1},\Lambda^{n+1})\arrow[d,equal]\arrow[r]&H_{n+1}(\Delta^{n+1},\partial\Delta^{n+1})\arrow{r}{\partial}&H_n(\partial\Delta^{n+1},\Lambda^{n+1})\arrow[equal,d,"\textrm{"good"}"']\arrow[r]&H_n(\Delta^{n+1},\Lambda^{n+1})\arrow[equal,d]\\
0&&H_n(\Delta^n,\partial\Delta^n)&0
\end{tikzcd}\]
to conclude that $\partial$ is an isomorphism. Let $\id_{n+1}\in H_{n+1}(\Delta^{n+1},\partial\Delta^{n+1})$, then
\begin{align*}
\partial \id_{n+1}=[\sum_k(-1)^k\id_{n+1}|_{k^{\textrm{th}}\textrm{ face}}]=[\pm\id_n]\quad\quad\quad\textrm{(the last face)}
\end{align*}
which generates $H_n(\Delta^n,\partial\Delta^n)$ by induction hypothesis. Hence $\id_{n+1}$ generates $H_{n+1}(\Delta^{n+1},\partial\Delta^{n+1})$.\medbreak

The ``invariance of dimension": if nonempty $U\subseteq\R^m$ and $V\subseteq\R^n$ are homeomorphic, then $m=n$. 

\paragraph{Mayer-Vietoris sequence}

wedge sum

\newpage
\subsection{Other topics}

\paragraph{Manifold and the degree of a map}

An \textbf{n-dimensional manifold} is a second countable Hausdorff space that is locally homeomorphic to $\R^n$.\medbreak

Let $p\in U\subseteq M$ such that $U$ is homeomorphic to $\R^n$, then $H_k(U,U\setminus\{p\})\cong\widetilde{H}_k(\R^n,\R^n\setminus\{p\})$. Considering the pair $(\R^n,\R^n\setminus\{p\})$ we have $\widetilde{H}_k(\R^n,\R^n\setminus\{p\})\cong \widetilde{H}_k(\R^n\setminus\{p\})\cong\widetilde{H}_k(\mathbf{S}^{n-1})$. Finally by excision, $H_k(M,M\setminus\{p\})\cong H_k(U,U\setminus\{p\})$. Together we have 
\[\widetilde{H}_k(M,M\setminus\{p\})\cong\widetilde{H}_k(\mathbf{S}^{n-1})\cong\begin{cases}
\Z,&k=n\\
0,&\textrm{otherwise}
\end{cases}.\]
Hence if two manifolds $M$ and $N$ are homeomorphic, then $\dim M=\dim N$.\medbreak

The \textbf{local orientation} $\mu_p$ of a manifold $M$ at $p\in M$ is a choice of generator $\{\pm1\}$ of $H_n(M,M\setminus\{p\})\cong\Z$. Two orientations are \textbf{consistent} if. \textbf{global orientation}

\paragraph{Euler characteristics and Betti number}

Euler Char: alternating sum of rank

theorem: number of k-cells,

proof: flatness of q, split exact sequence.

eg. $\chi(\mathbf{M}_g)=2-2g$, $\chi(\mathbf{S}^n)=\begin{cases}0,&n\textrm{ odd}\\2,&\textrm{otherwise}
\end{cases},\chi(\mathbf{RP}^n)=\begin{cases}0,&n\textrm{ odd}\\1,&\textrm{otherwise}
\end{cases}$,$\chi(\mathbf{CP}^n)=2n$

\paragraph{$\pi_1$ and $H_1$}

\paragraph{Simplicial approximation}

\newpage
\section{Cohomology}

\subsection{Basic constructions}

We dualize the chain complex by $\Hom(-,G)$ where $G$ is an abelian group, so that $C^\ast_n=\Hom(C_n,G)$ and $\partial_n^\ast$, the dual object of $\partial_{n+1}$, sends $\varphi\in\Hom(C_n,G)$ to $\varphi\partial_{n+1}\in\Hom(C_{n+1},G)$.
\begin{center}\begin{tikzcd}
\cdots\arrow[r]&C_{n+1}\arrow{r}{\partial_{n+1}}&C_n\arrow[d,Rightarrow,"\textrm{dual}"']\arrow{r}{\partial_n}&C_{n-1}\arrow[r]&\cdots\\
\cdots&C^\ast_{n+1}\arrow[l]&C^\ast_n\arrow[swap]{l}{\partial_n^\ast}&C^\ast_{n-1}\arrow[swap]{l}{\partial^\ast_{n-1}}&\cdots\arrow[l]
\end{tikzcd}\end{center}

Construct $h:H^n(C;G)\to\Hom(H_n(C),G)$ by $\varphi\mapsto\overline{\varphi}_0$, where $\varphi_0=\varphi|_{\ker\partial_n}$ and $\overline{\varphi}_0:H_n(C)\to G$. This is appropriate since $\varphi$ vanishes on $\im\partial_{n+1}$. $h$ is surjective by the following argument.\medbreak

The exact sequence $0\rightarrow\ker\partial_n\xrightarrow{i}C_n\xrightarrow{\partial_{n+1}}\im\partial_{n+1}\rightarrow0$ splits, since $\im\partial_{n+1}\subseteq C_n$ is free. Hence there exists $p:C_n\to\ker\partial_n$ such that $pi=\id_{\ker\partial_n}$. Then the extension $\varphi p:C_n\to G$ also vanishes on $\im\partial_{n+1}$, and $h':\Hom(H_n(C),G)\rightarrow\ker\partial^\ast_n\rightarrow H^n(C;G)$ makes $h$ surjective. Hence
\begin{center}\begin{tikzcd}[column sep=scriptsize]
0\arrow[r]&\ker h\arrow[r]&H^n(C;G)\arrow{r}{h}&\Hom(H_n(C),G)\arrow[r]&0
\end{tikzcd}.\end{center}

Now we dualize by $\Hom(-,G)$ and apply the zig-zag lemma to the canonical decomposition of $C_n$
\[\begin{tikzcd}[column sep=scriptsize]
& \cdots \arrow[r]
\arrow[d, phantom, ""{coordinate, name=X}]
& \im\partial_{n+1} \arrow[dll,
"i_{n+1}",
rounded corners,
to path={ -- ([xshift=2ex]\tikztostart.east)
|- (X) [near end]\tikztonodes
-| ([xshift=-2ex]\tikztotarget.west)
-- (\tikztotarget)}] \\
\ker\partial_n \arrow[r]
& H_n(C) \arrow{r}{\partial_n}
\arrow[d, phantom, ""{coordinate, name=Z}]
& \im\partial_n \arrow[dll,
"i_n",
rounded corners,
to path={ -- ([xshift=2ex]\tikztostart.east)
|- (Z) [near end]\tikztonodes
-| ([xshift=-2ex]\tikztotarget.west)
-- (\tikztotarget)}]\\
\ker\partial_{n-1} \arrow[r]
& \cdots
\end{tikzcd}\quad\overset{\textrm{dual}}{\Longrightarrow}\quad
\begin{tikzcd}[column sep=scriptsize]
& \cdots \arrow[r]
\arrow[d, phantom, ""{coordinate, name=X}]
& \ker\partial^\ast_{n-1} \arrow[dll,
"i^\ast_{n-1}",
rounded corners,
to path={ -- ([xshift=2ex]\tikztostart.east)
|- (X) [near end]\tikztonodes
-| ([xshift=-2ex]\tikztotarget.west)
-- (\tikztotarget)}] \\
\im\partial^\ast_n \arrow{r}{\partial_n^\ast}
& H^n(C;G) \arrow{r}
\arrow[d, phantom, ""{coordinate, name=Z}]
& \ker\partial^\ast_n \arrow[dll,
"i^\ast_n",
rounded corners,
to path={ -- ([xshift=2ex]\tikztostart.east)
|- (Z) [near end]\tikztonodes
-| ([xshift=-2ex]\tikztotarget.west)
-- (\tikztotarget)}]\\
\im\partial^\ast_{n+1} \arrow[r]
& \cdots
\end{tikzcd}\]
and obtain with $\ker h=\coker i^\ast_{n-1}=\im\partial^\ast_n/\im i^\ast_{n-1}=\im\partial^\ast_n/\ker\partial^\ast_{n-1}$,
\begin{center}\begin{tikzcd}[column sep=scriptsize]
0\arrow[r]&\coker i^\ast_{n-1}\arrow[r]&H^n(C;G)\arrow{r}{h}&\ker i^\ast_n=\Hom(H_n(C),G)\arrow[r]&0
\end{tikzcd}.\end{center}

Consider the dual of the free resolution $0\rightarrow\im\partial_n\rightarrow\ker\partial_{n-1}\rightarrow H_{n-1}(C)\rightarrow0$ of $H_{n-1}(C)$, it happens to be that $\Ext(H_{n-1}(C),G)=H^1(F;G)=\coker i^\ast_{n-1}$. Hence $\coker i^\ast_{n-1}$ only depends on $H_{n-1}(C)$ and $G$. And since the exact sequence splits, the cohomology $H^n(C;G)$ only depends on the homology $H_{n-1}(C)$ and $G$. To summarize:

\begin{theorem}[universal coefficient theorem, cohomology]
For a chain complex $C$ of free abelian groups and its homology groups $H_n(C)$, the cohomology groups $H^n(C;G)$ are determined by the split exact sequences
\begin{equation}\begin{tikzcd}[column sep=scriptsize]
0\arrow[r]&\Ext(H_{n-1}(C),G)\arrow[r]&H^n(C;G)\arrow{r}{h}&\Hom(H_n(C),G)\arrow[r]&0
\end{tikzcd}.\end{equation}
\end{theorem}

To calculate the $\Ext$ term we have the following\begin{enumerate}[(i)]
    \item $\Ext(H\oplus H',G)=\Ext(H,G)\oplus\Ext(H',G)$
    \item $\Ext(H,G)=0$ if $H$ is free (free resolution $0\rightarrow H\rightarrow H\rightarrow0$)
    \item $\Ext(\Z/n\Z,G)=G/nG$ (dualizing $0\rightarrow\Z\xrightarrow{n}\Z\rightarrow\Z/n\Z\rightarrow0$)
\end{enumerate}
As a corollary, $\Hom(H,\Z)$ is the torsion subgroup of $H$ if $H$ is finitely generated.

Calculation example of torus, M2, Klein bottle\bigbreak

Hatcher Example 3.16 says that the exterior algebra $\Lambda_R[\alpha_1,\cdots,\alpha_n]$ is the graded tensor product over $R$ of $\Lambda_R[\alpha_i]$ where all $\alpha_i$ has odd degree/dimension. I want to make sure that the antisymmetry $\alpha_i\alpha_j=-\alpha_j\alpha_i$ in $\Lambda_R[\alpha_1,\cdots,\alpha_n]$ is guaranteed by the odd degree like this:
\[\alpha_i\alpha_j=(\1_{\Lambda_R[\alpha_i]}\otimes\alpha_i)(\alpha_j\otimes\1_{\Lambda_R[\alpha_i]})=(-1)^{1\cdot1}\alpha_j\alpha_i\]

\newpage
\subsection{The additional ring structure}

The additional ring structure obtained from contravariance is useful in distinguishing spaces that the additive structure of (co)homology is incapable of doing.

\paragraph{The construction} For ring $R$, define cup product $C^m(X;R)\times C^n(X;R)\xrightarrow{\smile} C^{m+n}(X;R)$ by $(\varphi\smile\psi)(\sigma)=\varphi(\sigma|_{[v_0,\cdots,v_m}])\psi(\sigma|_{[v_m,\cdots,v_{m+n}]})$. Then $\partial^\ast(\varphi\smile\psi)=\partial^\ast\varphi\smile\psi+(-1)^m\varphi\smile\partial^\ast\psi$ implies that the cup product of cocycles is a cocycle, and the cup product of a cocycle and a coboundary is a coboundary. Hence $\smile$ induces $H^m(X;R)\times H^n(X;R)\xrightarrow{\smile}H^{m+n}(X;R)$, with the identity element $1\in H^0(X;R)$ that sends every singular $0$-simplex to $1_R$.

\begin{theorem}
$f^\ast:H^n(Y;R)\to H^n(X;R)$ satisfy $f^\ast(\alpha\smile\beta)=f^\ast(\alpha)\smile f^\ast(\beta)$ for $f:X\to Y$.
\end{theorem}

One can generalize to the relative version $H^m(X,A;R)\times H^n(X,B;R)\xrightarrow{\smile}H^{m+n}(X,A\cup B;R)$, by first $C^m(X,A;R)\times C^n(X,B;R)\to C^{m+n}(X,A+B;R)$ where cochains in $C^{m+n}(X,A+B;R)$ vanishes on sums of chains in $A$ and chains in $B$. Then $C^{m+n}(A+B;R)\cong C^{m+n}(A\cup B;R)$ by barycentric subdivision, and $C^{m+n}(X,A+B;R)\cong C^{m+n}(X,A\cup B;R)$ by the five lemma.\medbreak

Since cup product is associative and distributive, the cohomology ring $H^\ast(X;R)=\bigoplus_{i\geq0}H^0(X;R)$ is a graded ring that contains a more compact description.

\begin{theorem}
If $R$ is commutative, then $\alpha\smile\beta=(-1)^{mn}\beta\smile\alpha$, for $\abs{a}=m$ and $\abs{b}=n$.
\end{theorem}

Define a bilinear map $H^\ast(X;R)\times H^\ast(Y;R)\xrightarrow{\times}H^\ast(X\times Y;R)$ by $a\times b=p_1^\ast(a)\smile p_2^\ast(b)$ where $p_1$ and $p_2$ are projection from $X\times Y$ to $X$ and $Y$, respectively. This induces, for commutative ring $R$, $H^\ast(X;R)\otimes_RH^\ast(Y;R)\xrightarrow{\mu}H^\ast(X\times Y;R)$ a homomorphism of $R$-modules $a\otimes b\mapsto a\times b$. Define multiplication $(a\otimes b)(c\otimes d)=(-1)^{\abs{b}\abs{c}}ac\otimes bd$ in $H^\ast(X\times Y;R)$ so that commutativity (Theorem 3.3) is satisfied.

\begin{theorem}[Künneth formula]
If $H^n(Y;R)$ is finitely-generated for all $n$, then the cross product $\mu$ induces $H^\ast(X;R)\otimes_RH^\ast(Y;R)\cong H^\ast(X\times Y;R)$.
\end{theorem}
\begin{proof}
For CW complexes $X$ and $Y$ define functors $h^n(X,A)=\bigoplus_i(H^i(X,A;R)\otimes_RH^{n-i}(Y;R))$ and $k^n(X,A)=H^n(X\times Y,A\times Y;R)$. Check that $h$ and $k$ are (unreduced) cohomology theories:\begin{enumerate}[(i)]
    \item homotopy invariance is trivial;
    \item excision is trivial if we use the alternate formulation;
    \item existence of the long exact sequence
    \item disjoint union
\end{enumerate}

Let $\mu:h^n(X,A)\to k^n(X,A)$, then $\mu$ is natural in the sense that it commutes in the following way\medbreak

That $\mu$ is an isomorphism for CW pair $(X,A)$ follows from a reduction argument. Firstly it suffices to prove for $A=\varnothing$ by the five-lemma. Then proceed by induction on the dimension for finite-dimensional $X$. When $\dim X=0$ the statement is trivial invoking axiom (iv). The inductive step $X^n$, which is reduced to $(X^n,X^{n-1})$ by the five-lemma. This is further reduced to $\bigsqcup_\alpha(e_\alpha^n,\partial e^n_\alpha)$ by characteristic map $\Phi$, noticing that $\Phi^\ast$ is an isomorphism in $h$ and $k$ cohomology theories. Once again by the axiom disjoint union this is reduced to $(e_\alpha^n,\partial e^n_\alpha)$, which is trivial by the five-lemma and induction hypothesis.\medbreak

If $X$ is infinite-dimensional, then it reduces to the finie-dimensional case by the telescope argument.


\end{proof}

\paragraph{Examples}\begin{enumerate}[(i)]
\item Let $M_2$ be an orientable surface of genus $2$, then $H^n(M_2;\Z)\cong\Hom(H_n(M_2),\Z)$ by cellular cohomology, and $H^0(M_2;\Z)=H^2(M_2;\Z)=\Z$ and $H^1(M_2;\Z)=\Z^4$. Let $\{\alpha_i\}_{1\leq i\leq4}$ be the basis of $H^1(M_2;\Z)$, then $\alpha_i$ sends $a_i$ to $\1_R$ and other generators to $0$. One can check that the vanishing condition $\partial^\ast_1\alpha_i=0$ makes $\alpha_i$ a traversing line between two $a_i$. Applying cup product we find that $\alpha_1\smile\alpha_2=1$ on $[0,2,1]$ and $\alpha_2\smile\alpha_1=1$ on the right adjacent one. This $1$ is the generator of $H^2(M_2;\Z)$.

\item Let $T^2$ be a torus, then $H^0(T^2;\Z)=H^2(T^2;\Z)=\Z$ and $H^1(T^2;\Z)=\Z^2=(\alpha,\beta)$. On the $\Delta$-complex structure of $T^2$ we see that $\alpha\smile\beta=-(\beta\smile\alpha)$ generates $H^2(T^2;\Z)$. Hence $H^\ast(T^2;\Z)=\Z[\alpha,\beta]/(\alpha^2,\beta^2,\alpha\beta=-\beta\alpha)=\Lambda_\Z[\alpha,\beta]$, a result also obtainable by the Künneth formula $H^\ast(\s^1\times\s^1;\Z)=\Lambda_\Z[\alpha]\otimes_\Z\Lambda_\Z[\beta]=\Lambda_\Z[\alpha,\beta]$, since $\abs{\alpha}=\abs{\beta}=1$ is odd.\medbreak

In general, since $H^\ast(\s^n;\Z)=\Lambda_\Z[\alpha]=\Z[\alpha]/(\alpha^2)$ where $\abs{\alpha}=n$, by the Künneth formula $H^\ast(\s^{2m}\times\s^{2n+1};\Z)=\Z[\alpha]/(\alpha^2)\otimes_\Z\Lambda_\Z[\beta]$ where $\abs{\alpha}=2m$ and $\abs{\beta}=2n+1$. The distinction of parity comes from the commutativity criteria\footnote{When $\abs{\alpha}$ and $\abs{\beta}$ is odd, $\Lambda_R[\alpha,\beta]$ guarantees $\alpha\beta=-\beta\alpha$ but $\Z[\alpha]/(\alpha^2)\otimes_\Z\Z[\beta]/(\beta^2)$ does not. The distinction becomes irrelevant if $\Char R=2$.}.

\item Let $K$ be a Klein bottle, $\partial_1^\ast:\alpha\mapsto\eta+\zeta$ since $\partial_1^\ast\alpha(U)=\partial_1^\ast\alpha(L)=1$, and similarly, $\beta\mapsto\eta-\zeta$ and $\gamma\mapsto\zeta-\eta$. Hence $\im\partial_1^\ast=(\eta+\zeta,\eta-\zeta,\zeta-\eta)=(\eta+\zeta)$, taking $R=\Z_2$, and $\ker\partial^\ast_1=(\alpha+\beta,\alpha+\gamma)$. Since $\partial_0^\ast\mu$ on any of $a,b,c$ is $0$, $\im\partial^\ast_0=0$. In summary, $H^2(K;\Z_2)=(\eta,\zeta)/(\eta+\zeta)=(\eta)$ and $H^1(K;\Z_2)=(\alpha+\beta,\alpha+\gamma)$. Finally let $x=\alpha+\beta$ and $y=\beta+\gamma$, then $x\smile x(U)=x\smile y(U)=1$ and all other terms are $0$. Hence $x\smile x=x\smile y$ acting on $U$ gives $\eta$. We obtain $H^\ast(K,\Z_2)=\Z_2[x,y]/(x^3,y^3,xy,x^2-y^2)$ with basis $1,x,x^2,y$.


\end{enumerate} 


\subsection{Application to manifolds}

\subsection{Tools}

\paragraph{CW approximation}


\end{document}