\documentclass[psamsfonts]{amsart}

\usepackage{amssymb,amsfonts}
\usepackage[all,arc]{xy}
\usepackage{enumerate}
\usepackage{mathrsfs}
\usepackage{tikz-cd}
\usetikzlibrary{decorations.pathmorphing}

\usepackage{hyperref}  
\hypersetup{
  bookmarksnumbered=true,%
  bookmarks=true,%
  colorlinks=true,%
  linkcolor=magenta,%
  citecolor=blue,%
  filecolor=blue,%
  menucolor=blue,%
  pagecolor=blue,%
  urlcolor=blue,%
  pdfnewwindow=true,%
  pdfstartview=FitBH}
  
\let\fullref\autoref
%
%  \autoref is very crude.  It uses counters to distinguish environments
%  so that if say {lemma} uses the {theorem} counter, then autrorefs
%  which should come out Lemma X.Y in fact come out Theorem X.Y.  To
%  correct this give each its own counter eg:
%                 \newtheorem{theorem}{Theorem}[section]
%                 \newtheorem{lemma}{Lemma}[section]
%  and then equate the counters by commands like:
%                 \makeatletter
%                   \let\c@lemma\c@theorem
%                  \makeatother
%
%  To work correctly the environment name must have a corrresponding 
%  \XXXautorefname defined.  The following command does the job:
%
\def\makeautorefname#1#2{\expandafter\def\csname#1autorefname\endcsname{#2}}

\def\equationautorefname~#1\null{(#1)\null}
\makeautorefname{footnote}{footnote}%
\makeautorefname{item}{item}%
\makeautorefname{figure}{Figure}%
\makeautorefname{table}{Table}%
\makeautorefname{part}{Part}%
\makeautorefname{appendix}{Appendix}%
\makeautorefname{chapter}{Chapter}%
\makeautorefname{section}{Section}%
\makeautorefname{subsection}{Section}%
\makeautorefname{subsubsection}{Section}%
\makeautorefname{theorem}{Theorem}%
\makeautorefname{thm}{Theorem}%
\makeautorefname{cor}{Corollary}%
\makeautorefname{lem}{Lemma}%
\makeautorefname{prop}{Proposition}%
\makeautorefname{defn}{Definition}%
\makeautorefname{rem}{Remark}%
\makeautorefname{rems}{Remarks}%
\makeautorefname{quest}{Question}%
\makeautorefname{exmp}{Example}%
\makeautorefname{ax}{Axiom}%
\makeautorefname{claim}{Claim}%
\makeautorefname{notn}{Notation}
\makeautorefname{notns}{Notations}%

\newtheorem{thm}{Theorem}[section]
\newtheorem{cor}{Corollary}[section]
\newtheorem{prop}{Proposition}[section]
\newtheorem{lem}{Lemma}[section]

\theoremstyle{definition}
\newtheorem{defn}{Definition}[section]
\newtheorem{ax}{Axiom}[section]
\newtheorem{con}{Construction}[section]
\newtheorem{exmp}{Example}[section]
\newtheorem{quest}{Question}[section]
\newtheorem{rem}{Remark}[section]
\newtheorem{notn}{Notation}[section]

\newcommand{\Q}{\mathbb{Q}}
\newcommand{\Z}{\mathbb{Z}}
\newcommand{\Top}{\mathbf{Top}}
\newcommand{\sSet}{\mathbf{sSet}}
\newcommand{\dgcAlg}{\mathbf{dgcAlg}}
\newcommand{\Hom}{\mathrm{Hom}}
\newcommand{\Ho}{\mathrm{Ho}}
\newcommand{\W}{\mathrm{W}}
\newcommand{\Fib}{\mathrm{Fib}}
\newcommand{\Cof}{\mathrm{Cof}}
\newcommand{\cof}{\mathrm{cof}}
\newcommand{\fib}{\mathrm{fib}}
\newcommand{\id}{\mathrm{id}}

\DeclareMathOperator{\colim}{colim}
\DeclareMathOperator{\coker}{coker}

\makeatletter
\let\c@obs=\c@thm
\let\c@cor=\c@thm
\let\c@prop=\c@thm
\let\c@lem=\c@thm
\let\c@prob=\c@thm
\let\c@con=\c@thm
\let\c@conj=\c@thm
\let\c@defn=\c@thm
\let\c@notn=\c@thm
\let\c@notns=\c@thm
\let\c@exmp=\c@thm
\let\c@ax=\c@thm
\let\c@pro=\c@thm
\let\c@ass=\c@thm
\let\c@warn=\c@thm
\let\c@rem=\c@thm
\let\c@conv=\c@thm
\let\c@sch=\c@thm
\let\c@equation\c@thm
\numberwithin{equation}{section}
\makeatother

\bibliographystyle{plain}

\title{On Realizing Rational and Polynomial Cohomology Rings}

\author{Yunhan (Alex) Sheng}

\begin{document}

\begin{abstract}

This is an expository paper written during the $\textrm{\smash{2021 REU}}$ program at the University of Chicago. In this paper we address the following question: which commutative graded $R$-algebras can be realized as the cohomology ring of a space with coefficients in $R$? We discuss two solved cases: for $R=\Q$, and for $R$ is the polynomial algebra over $\Z$ or $\Z_p$ for prime $p$. Model categories play an important role in our discussion.

\end{abstract}

\maketitle

\tableofcontents

\section{Introduction}

Some historical information.

\section{Rational Homotopy Theory}

The main gaol of this section is to show, restricting to rational and $1$-connected objects of finite type, an equivalence between homotopy categories:
\[\Ho(\Top_{\Q,1,f})\cong\Ho(\sSet_{\Q,1,f})\cong\Ho(\dgcAlg_{\Q,1,f}).\]
where $\Top$, $\sSet$ and $\dgcAlg$ denote the category of topological spaces, simplicial sets, and differential graded-commutative algebras respectively.
Therefore, any $1$-connected differential graded-commutative algebra over $\Q$ of finite type can be realized as a rational cohomology ring $H^\ast(X;\Q)$ of some space $X$.

To prove the desired equivalence, we need the concept of model categories, which impose an additional structure on the original categories. We develop this machinery in Section 2.2. We will construct a pair of adjoint functors between $\Top$ and $\sSet$ and prove that they induce equivalence on the respective homotopy categories. In Section 2.3 we describe properties of differential graded-commutative algebras and the model structure on $\dgcAlg$. Finally in Section 2.4 we construct and prove the similar equivalence between $\sSet$ and $\dgcAlg$.

\subsection{Rationalization and localization}

In this section we introduce some basic notions of rational homotopy theory. The main result is Proposition 2.10.

\begin{defn}
A $1$-connected space is \textbf{rational} if $\pi_n(X)$ is a $\Q$-vector space for all $n\geq1$. A map $f:X\to Y$ between $1$-connected spaces is a \textbf{rational homotopy equivalence} (in which case we write $X\simeq_\Q Y$) if the map
\[\pi_n(f)\otimes\Q:\pi_n(X)\otimes\Q\to\pi_n(Y)\otimes\Q\]
is an isomorphism for all $n\geq1$.
\end{defn}

\begin{rem}
In rational homotopy theory we study spaces up to rational homotopy equivalence. By tensoring with $\Q$ we ``kill" the torsion abelian groups. In order for the tensor product to be well-defined, we need the homotopy groups to be abelian. That's why we will be working with $1$-connected spaces exclusively. 
\end{rem}

The notion of a Serre class deals with the ``killing'' of abelian groups efficiently.

\begin{defn}
A collection $\mathfrak{C}$ of abelian groups is a \textbf{Serre class} if for a short exact sequence of abelian groups
\[0\rightarrow A\to B\to C\to0,\]
$A\in\mathfrak{C}$ and $C\in\mathfrak{C}$ implies $B\in\mathfrak{C}$. A map $f\in\Hom(A,B)$ is a \textbf{$\mathfrak{C}$-isomorphism} if both $\ker f\in\mathfrak{C}$ and $\coker f\in\mathfrak{C}$.
\end{defn}

\begin{rem}
We can think of a Serre class as the class of groups we wish to ``kill". In our case, take $\mathfrak{C}$ to be the class of torsion abelian groups, then $f$ is a rational homotopy equivalence if and only if $\pi_n(f)$ is a $\mathfrak{C}$-isomorphism.
\end{rem}

\begin{thm}[Serre-Hurewicz]
Let $\mathfrak{C}$ e a Serre class and $X$ be a $1$-connected space. Then $\pi_i(X)\in\mathfrak{C}$ for all $i<n$ if and only if $H_i(X)\in\mathfrak{C}$ for all $i<n$. Moreover, the Hurewicz map $\pi_n(X)\to H_n(X)$ is a $\mathfrak{C}$-isomorphism.
\end{thm}

\begin{thm}[Serre-Whitehead]
Let $\mathfrak{C}$ be a Serre class and $f:X\to Y$ be a map between $1$-connected spaces. Then $\pi_i(f):\pi_i(X)\to\pi_i(Y)$ is a $\mathfrak{C}$-isomorphism for $i<n$ if and only if $H_i(f):H_i(X)\to H_i(Y)$ is a $\mathfrak{C}$-isomorphism for $i<n$.
\end{thm}

We would like to to associate an arbitrary space $X$ to a rational space $X_\Q$ such that $X\simeq_\Q X_\Q$. This process is called \textbf{rationalization}, and it can be done via localization. Let's consider a heuristic example.

\begin{exmp}

In algebra we obtain $\Q$ from $\Z$ by inverting all primes:
\[\Q\cong\colim\left(\Z\xrightarrow{\times2}\Z\xrightarrow{\times3}\Z\xrightarrow{\times5}\cdots\right).\]
We mimic this construction, but instead of using colimit, we use \textit{homotopy colimit}, which can be thought as first passing to cofibrations and then taking colimits. %We will develop this concept rigorously in Section 3.
Consider the limit sequence
\[\begin{tikzcd}
X_0\arrow[d,equal]\arrow[r,"f_1"]&X_1\arrow[d,shift right]\arrow[r,"f_2"]&X_2\arrow[d,shift right]\arrow[r,"f_3"]&\cdots\\
M_0\arrow[r]&M_1\arrow[u,"r_1"',shift right]\arrow[r]&M_2\arrow[u,"r_2"',shift right]\arrow[r]&\cdots
\end{tikzcd}\]
where each $M_n$ is the mapping cylinder of the map $f_nr_{n-1}:M_{n-1}\to X_n$. The homotopy colimit of $X_n$ is the colimit of the cofibrations $M_n$, which is also the rationalization of $X_0$.

Intuitively we can also think of the telescopic argument, and in the $n$-th step of a sequence we ``collapse'' to $X_n$. In the case of an $m$-sphere $S^m$, the rationalization $S^m_\Q$ is obtained by $\colim_n{S^m(n)}$, where each $S^m(n)$ is constructed inductively by gluing a copy of $S^m$ to $S^m(n-1)$ via a degree-$n$ map.
\end{exmp}

\begin{defn}
Let $\mathbf{C}$ be a category and $\mathrm{L}$ a subclass of morphisms in $\mathbf{C}$. An object $A$ of $\mathbf{C}$ is \textbf{$\mathrm{L}$-local} if for every $f\in\Hom_\mathbf{C}(X,Y)$ such that $f\in\mathrm{L}$, the induced map $f^\ast:\Hom_\mathbf{C}(Y,A)\to\Hom_\mathbf{C}(X,A)$ is a bijection. Diagrammatically,
\[\begin{tikzcd}
Y\arrow[dr]\arrow[rr,"f^\ast"]&&X\arrow[dl]\\
&A
\end{tikzcd}.\]
A map $l:A\to A'$ is an \textbf{$\mathrm{L}$-localization} if $l\in\mathrm{L}$ and $A'$ is $\mathrm{L}$-local.
\end{defn}

We now take $\mathrm{Q}$ to be the class of rational homotopy equivalences. We show that $\mathrm{Q}$-localization is equivalent to rationalization in the category of $1$-connected spaces.

\begin{prop}
An object of $\Top_1$ is $\mathrm{Q}$-local if it is rational.
\end{prop}
\begin{proof}
Let $A$ be a rational space and $f:X\to Y$ a map between $1$-connected spaces. We show that $f^\ast:[Y,A]\to[X,A]$ is a bijection.\medbreak

\textit{Surjectivity.} Given any $X\to A$, we wish to find a lift $Y\to A$. Consider the Postnikov tower of $A$. By the $1$-connectedness the base case $A_1=\ast$, and the lift is trivial. Proceed by induction and suppose there is a lift $Y\to A_{n-1}$. Then the problem of finding $Y\to A_n$ amounts to finding a map $Y\to\ast$ below:
\[\begin{tikzcd}[column sep=large, row sep=large]
X\arrow[d,"f"']\arrow[r]&A_n\arrow[d]\arrow[r]&\ast\arrow[d]\\
Y\arrow[r]\arrow[urr,dashed]\arrow[ur,dashed]&A_{n-1}\arrow[r,"k^{n+1}"]&K(\pi_n(A),n+1)
\end{tikzcd}\]
where $k^{n+1}$ is the $k$-invariant. We only need to check that the obstruction class $\omega\in H^{n+1}(Y,X;\pi_n(A))$ is trivial. This is indeed the case since $\pi_nA$ is a $\Q$-vector space and $X\simeq_\Q Y$.\medbreak

\textit{Injectivity.} Given any $X\to A$ and suppose that $\lambda_0,\lambda_1:Y\to A$, we wish to find a homotopy between $\lambda_0$ and $\lambda_1$. This is equivalent to another lifting problem:
\[\begin{tikzcd}[row sep=large]
(X\times[0,1])\cup_{X\times\partial[0,1]}(Y\times\partial[0,1])\arrow[r]\arrow[d,"f'"']&A\\
Y\times[0,1]\arrow[ur,dashed]
\end{tikzcd}\]
where $X\times[0,1]\to A$ and $Y\times\partial[0,1]\to A$ are defined naturally. Note that $f'$ is a rational homotopy equivalence, as it induces isomorphisms on all rational homology groups. Then the same obstruction argument in surjectivity applies.
\end{proof}

\begin{prop}
Every $1$-connected space $X$ admits a rationalization $X_\Q$.
\end{prop}
\begin{proof}
Fist note that we can rationalize an Eilenberg-MacLane space $K(G,n)$ by simply taking $K(G\otimes\Q,n)$. This motivates us to construct $X_\Q$ inductively from its Postnikov tower and associated $k$-th invariants.

By $1$-connectedness, the base case starts from $X_2=K(\pi_2(X),2)$. Suppose that the rationalization $(X^{n+1})_\Q$ of $X_{n-1}$ has already been constructed. Since $K(\pi_n(X)\otimes\Q,n+1)$ is rational, it is $\mathrm{Q}$-local by Proposition 2.9. Hence the rationalization $r:X_{n-1}\to(X_{n-1})_\Q$ induces a bijective correspondence
\[[(X_{n-1})_\Q,K(\pi_n(X)\otimes\Q,n+1)]\cong[X_{n-1},K(\pi_n(X)\otimes\Q,n+1)]\]
which identifies the $k$-th invariant $k^{n+1}$ with a map $(k^{n+1})_\Q$. Since $X_n$ is the fiber of $k^{n+1}$, we define $(X_n)_\Q$ to be the fiber of $(k^{n+1})_\Q$:
\[\begin{tikzcd}
X_n\arrow[r]&X_{n-1}\arrow[d]\arrow[r,"k^{n+1}"]&K(\pi_n(X),n+1)\arrow[d]\\
(X_n)_\Q\arrow[r,dashed]&(X_{n-1})_\Q\arrow[r,"(k^{n+1})_\Q"]&K(\pi_n(X)\otimes\Q,n+1)
\end{tikzcd}.\]
Finally, we define $X_\Q$ to be the inverse limit $\lim_n(X_n)_\Q$ of the Postnikov tower.
\end{proof}

Now we have the following converse of Proposition 2.9

\begin{prop}
An object of $\Top_1$ is rational if it is $\mathrm{Q}$-local.
\end{prop}
\begin{proof}
Suppose $A$ is $\mathrm{Q}$-local, then the rationalization $r:A\to A_\Q$ induces a bijection $[A_\Q,A]\cong[A,A]$. Since $A_\Q$ is rational, it is $\mathrm{Q}$-local by Proposition 2.9. Hence there is another bijection $[A_\Q,A_\Q]\cong[A,A_\Q]$. These two bijections together imply that $A\simeq_\Q A_\Q$. By the Serre-Whitehead theorem $A$ is rational as well.
\end{proof}

\begin{cor}
Rationalization is equivalent to $\mathrm{Q}$-localization.
\end{cor}
\begin{proof}
This follows from Proposition 2.9 and Proposition 2.11.
\end{proof}

\subsection{The first Quillen equivalence}

We begin by introducing simplicial sets.\medbreak

Let $\mathbf{\Delta}$ be the category with order sets $[n]=\{0,1,\cdots,n\}$ as objects and order-preserving functions $\varphi:[m]\to[n]$ (that is, functions satisfying $\varphi(i)\leq\varphi(i+1)$ for all $i\in[m]$)
as morphisms.

\begin{defn}
A \textbf{simplicial object} in a category $\mathbf{C}$ is a functor $\mathbf{\Delta}^\textrm{op}\to\mathbf{C}$.
\end{defn}

\begin{rem}
Simplicial objects provides a combinatorial way of viewing things.

Simplicial sets are simplicial objects in the category $\mathbf{Set}$ of sets. It generalizes the idea of simplicial complexes.

Cosimplicial spaces are the cosimplicial objects in the category $\Top$. It is a covariant functor $\mathbf{\Delta}\to\Top$ (or a contravariant functor $\mathbf{\Delta}^\textrm{op}\to\Top$). It assigns each $[n]$ with a standard topological $n$-simplex $\Delta^n$.
\end{rem}

Let $\sSet$ be the category with simplicial sets as objects and natural transformations between simplicial sets as morphisms. Then there is a natural functor $\Delta:\mathbf{\Delta}^\textrm{op}\to\sSet$ defined by $[n]\mapsto\Delta[n]$. The cosimplicial space defines a (contravariant) functor $\Phi:\mathbf{\Delta}^\textrm{op}\to\Top$. The \textit{Kan extension} then establishes functors between $\sSet$ and $\Top$:
\[\begin{tikzcd}
\mathbf{\Delta}^\textrm{op}\arrow[dr,"\Delta"']\arrow[rr,"\Phi"]&&\Top\arrow[dl,shift right,dashed]\\
&\sSet\arrow[ur,shift right,dashed]
\end{tikzcd}\]

We now make the definition precise.

\begin{defn}
For any space $X$, the \textbf{singular simplicial complex} $\mathcal{S}_\bullet$ is a hom-functor $\Top\to\sSet$ given by
\[\mathcal{S}_\bullet(X)=\Hom_\Top(\Phi,X)=\Hom_\Top(\Delta^\bullet,X)\]
where $\Phi$ is the cosimplicial space and $\Delta^\bullet$ are the standard topological simplices.
\end{defn}

\begin{rem}
The subscript $_\bullet$ encodes simplicial data as follows:
\[\begin{tikzcd}
{[m]}\arrow[d,"\varphi"]\arrow[r]&\Delta^m\arrow[r]\arrow[d,"\varphi_\ast"]&\mathcal{S}_m(X)=\Hom_{\Top}(\Delta^m,X)\\
{[n]}\arrow[r]&\Delta^n\arrow[r]&\mathcal{S}_n(X)=\Hom_{\Top}(\Delta^n,X)\arrow[u,"\varphi^\ast"']
\end{tikzcd}.\]
For $\psi\in\mathcal{S}_n(X)$, the induced map $\varphi^\ast$ is given by $\varphi^\ast(\psi)=\psi\varphi_\ast$.
\end{rem}

\begin{defn}
For any simplicial set $S$, its \textbf{geometric realization} $|S|$ is a hom-functor $\sSet\to\Top$ given by
\[|S|=\Hom_\sSet(S,\Phi)=\colim_{\Delta[n]\to S}\Phi[n]\]
where $\Phi$ is the cosimplicial space $\mathbf{\Delta}\to\Top$.
\end{defn}

\begin{prop}
The singular simplicial complex functor is left adjoint to the geometric realization functor.
\end{prop}
\begin{proof}
Let $S$ be a simplicial set and $X$ be a space. Then
\begin{align*}
\sSet(S,\mathcal{S}_\bullet(X))&\cong\textstyle{\lim_{\Delta[n]\to S}\sSet(\Delta[n],\mathcal{S}_\bullet(X))}\\
&\cong\textstyle{\lim_{\Delta[n]\to S}\mathcal{S}_n(X)}\\
&\cong\textstyle{\lim_{\Delta[n]\to S}\Top(\Phi[n],X)}\cong\Top(|S|,X)
\end{align*}
where $\sSet(\Delta[n],\mathcal{S}_\bullet(X))\cong\mathcal{S}_n(X)$ is given by the Yoneda lemma.
\end{proof}

\begin{rem}
There is another more intuitive definition of geometric realization: for a simplicial set $S$, define
\[|S|=\coprod_{n\geq0}(\Delta[n]\times\Delta^n)/_{\sim}\]
where $\Delta[n]$ are the $n$-simplices of $S$. In plain terms, we associate a topological $n$-simplex to each $n$-simplex of $S$ and glue them together under the identification $(\varphi^\ast(x),t)\sim(x,\varphi_\ast(t))$. These two definitions are equivalent.
\end{rem}

We now turn to model categories. Model category abstracts certain common homotopy-theoretical properties that exist in many different categories.

\begin{notn}
For a class of morphisms $\mathrm{M}$, we write $\mathrm{M}^l$ (resp. $\mathrm{M}^r$) for the class of morphisms that have the left (resp. right) lifting property with respect to morphisms in $\mathrm{M}$.
\end{notn}

\begin{defn}
A \textbf{model category} if a category $\mathbf{C}$ with three subclasses of morphisms specified: the weak equivalences $\W$, the fibrations $\Fib$ and the cofibrations $\Cof$. They satisfy the following axioms:\begin{enumerate}[(M1)]
    \item The category $\mathbf{C}$ admits finite limits and colimits;
    \item The class of weak equivalences $\W$ contains all isomorphisms and satisfy the 2-out-of-3 property: if any two of the three morphisms $f$, $g$, and $fg$ belongs to $\W$, then the third also belongs to $\W$;
    \item Any two classes determine the third as follows:
    \[W=\Fib^r\circ\Cof^l,\quad\Fib=(\W\cap\Cof)^r,\quad\Cof=(\W\cap\Fib)^l;\]
    \item Any morphism in $\mathbf{C}$ can be factored in two ways: either $\Fib\circ(\W\cap\Cof)$ or $(\W\cap\Fib)\circ\Cof$.
\end{enumerate}
\end{defn}

\begin{exmp}
Reader should be familiar with fibrations and cofibrations in $\Top$. In fact, the following defines a model category structure on $\Top$:\begin{itemize}
    \item weak homotopy equivalences as weak equivalences $\W$;
    \item Serre fibrations as fibrations $\Fib$;
    \item the set of maps $I=\{S^{n-1}\hookrightarrow D^n:n\geq0\}$ generating cofibrations $\Cof$.
\end{itemize}
This is sometimes referred to as the $q$-model structure on $\Top$. It is \textit{cofibrantly generated}. The verification is not readily straightforward, and we refer to Quillen's original tome \cite{Quillen} for a detailed treatment.
\end{exmp}

\begin{exmp}
The following classes of morphisms defines a model category structure on $\sSet$:
\begin{itemize}
    \item The weak equivalences $\W$ are those morphisms $f:S\to T$ that induce weak homotopy equivalences $f_\ast:|S|\to|T|$ on the geometric realization;
    \item Kan fibrations as fibrations $\Fib$;
    \item degreewise inclusions as cofibrations $\Cof$.
\end{itemize}
This is sometimes referred to as the $q$-model structure on $\sSet$. A proof due to Bousfield and May is available in \cite{May-Ponto}. The proof is similar to the proof the model structure on $\dgcAlg_k$, of which we develop a careful treatment in Section 2.3.
\end{exmp}

\begin{defn}
Let $\mathbf{C}$ and $\mathbf{D}$ be model categories. An adjunction $\mathcal{F}\dashv\mathcal{G}$ is a \textbf{Quillen adjunction} if it satisfy the following equivalent conditions:\begin{enumerate}[(i)]
    \item $\mathcal{F}$ preserves cofibrations and acyclic cofibrations;
    \item $\mathcal{G}$ preserves fibrations and acyclic fibrations.
\end{enumerate}
\end{defn}

If $\mathcal{F}\dashv\mathcal{G}$ is a Quillen adjunction between $\mathbf{C}$ and $\mathbf{D}$, then the total derived functors $\mathbf{L}\mathcal{F}$ and $\mathbf{R}\mathcal{G}$ are defined by taking cofibrant and fibrant replacements:
\[\mathbf{L}\mathcal{F}(X)=\mathcal{F}(X^\cof),\quad\mathbf{R}\mathcal{G}(Y)=\mathcal{G}(Y^\fib).\]
These derived functors form an adjunction on the homotopy categories $\Ho(\mathbf{C})$ and $\Ho(\mathbf{D})$ obtained by localizing with respect to weak equivalences:
\[\Ho(\mathbf{C})=\mathbf{C}[\W_\mathbf{C}^{-1}],\quad\Ho(\mathbf{D})=\mathbf{D}[\W_\mathbf{D}^{-1}].\]

\begin{defn}
An Quillen adjunction $\mathcal{F}\dashv\mathcal{G}$ is a \textbf{Quillen equivalence} if the following conditions hold:\begin{enumerate}[(i)]
    \item the derived adjunction unit $\Bar{\eta}:X\to\mathbf{R}\mathcal{G}(\mathcal{L}(X))$ is a weak equivalence in $\mathbf{C}$;
    \item the derived adjunction counit $\Bar{\epsilon}:\mathbf{L}\mathcal{F}(\mathcal{G}(Y))\to Y$ is a weak equivalence in $\mathbf{D}$.
\end{enumerate}
\end{defn}

If $\mathcal{F}\dashv\mathcal{G}$ is a Quillen equivalence, then the derived functor $\mathbf{L}\mathcal{F}:\Ho(\mathbf{C})\to\Ho(\mathbf{D})$ is an equivalence on homotopy categories. We omit the details of the proof.

\begin{prop}
The singular simplicial complex $\mathcal{S}_\bullet$ and the geometric realization functor forms a Quillen equivalence between $\Top$ and $\sSet$.
\end{prop}
\begin{proof}
Again, we only provide a brief sketch. We refer to xxx and xx for details.\medbreak

\textit{Quillen adjunction.} Since both $\Top$ and $\sSet$ are cofibrantly generated, by Proposition A.88 we only need to check that $|i|in\Cof_{\Top}$ for $i\in I_{\sSet}$ and $|j|\in\Cof_{\Top}\cap\W_{\Top}$ for $j\in J_{\sSet}$. Both are immediate since $|\partial\Delta[n]|=S^{n-1}$ and $|\Delta[n]|=D^n$.\medbreak

\textit{Quillen equivalence.} By Proposition A.87, we only need to check the following:\begin{enumerate}[(i)]
    \item if $|f|\in\W_{\Top}$ for cofibrant objects in $\Top$, then $f\in\W_{\sSet}$ for cofibrant objects in $\sSet$;
    \item the map $\smash{|\mathcal{S}_\bullet(Y)^{\cof}|\to Y}$ is a weak equivalence in $\Top$ for $Y\in\Fib_\Top$
\end{enumerate}
(i) is immediate by definition of $\W_{\sSet}$.
\end{proof}

\subsection{A model structure on $\dgcAlg_k$}

In this section we describe the algebraic model of our interest. A key result is Proposition 2.33.\medbreak

In what follows, let $R$ be a ring and $k$ be a field with characteristic $0$.

\begin{defn}
A \textbf{differential graded module} $M$ over $R$ is a $R$-module endowed with a cochain structure
\[\begin{tikzcd}
M_0\arrow[r,"d"]&M_1\arrow[r,"d"]&M_2\arrow[r,"d"]&\cdots
\end{tikzcd}.\]
\end{defn}

Write $\mathbf{DGM}_R$ for the category with differential graded modules over $R$ as objects and chain maps (maps that respects differentials) as morphisms.

\begin{rem}
Let $M$ and $N$ be differential graded modules over $R$. The (graded) tensor product $M\otimes_RN$ endows $\mathbf{DGM}_R$ with a  symmetric monoidal structure. The symmetric braiding is given by
\[a\otimes b=(-1)^{|a||b|}b\otimes a,\]
where $|a|$ is the degree of $a$. In addition, the Leibniz rule holds:
\[\smash{d(a\otimes b)=da\otimes b+(-1)^{|a|}a\otimes db}.\]
%We see higher-categorical analog in Section 3.
\end{rem}

\begin{defn}
A \textbf{differential graded-commutative algebra} $A$ over $R$ is a commutative monoid in $\mathbf{DGM}_R$.

Explicitly, $A$ is an object of $\mathbf{DGM}_R$ together with a unit $\eta:R\to A$ and an associative product $\mu:A\otimes_RA\to R$ given by $a\otimes b\to a\cdot b$:
\[\begin{tikzcd}[row sep=0.8em]
	{(A\otimes_RA)\otimes_RA} & {A\otimes_RA} \\
	&& A \\
	{A\otimes_R(A\otimes_RA)} & {A\otimes_RA}
	\arrow[from=1-1, to=3-1]
	\arrow["{\mu\otimes_R\id}"', from=1-1, to=1-2]
	\arrow["{\id\otimes_R\mu}", from=3-1, to=3-2]
	\arrow["\mu"', from=3-2, to=2-3]
	\arrow["\mu", from=1-2, to=2-3]
\end{tikzcd}\quad
\begin{tikzcd}[column sep=1.20em]
	& {M\otimes_RM} \\
	{R\otimes_RM} & M & {M\otimes_RR}
	\arrow["{\eta\otimes_R\id}", from=2-1, to=1-2]
	\arrow["{\id\otimes_R\eta}"', from=2-3, to=1-2]
	\arrow[from=2-1, to=2-2]
	\arrow[from=2-3, to=2-2]
	\arrow["\mu", from=1-2, to=2-2]
\end{tikzcd}.\]
The monoid is commutative in the sense of $\smash{a\cdot b=(-1)^{|a||b|}b\cdot a}$.
\end{defn}

Chain maps $f$ between differential graded-commutative algebras respect the units $f\eta=\eta$ and products $f\mu=\mu(f\otimes f)$. Write $\dgcAlg_R$ for the category with differential graded-commutative algebras as objects and chain maps as morphisms.\medbreak

Now we describe a model structure on $\dgcAlg_k$. Note that we switch form a ring $R$ to field $k$ of characteristic 0. (Why?)

\begin{prop}
The following defines a model category structure on $\dgcAlg_k$:\begin{enumerate}[(i)]
    \item quasi-isomorphisms as weak equivalences $\W$;
    \item degreewise surjections as fibrations $\Fib$.
\end{enumerate}
Note that the cofibrations $\Cof$ are thence uniquely determined to be
$(\Fib\cap\W)^r$.
\end{prop}

Before proving the theorem, we describe the models that mimic the notion of spheres $S^n$ and disks $D^{n-1}$ in $\Top$. This is crucial for inductive arguments.\medbreak

For a graded vector space $V$, write $\Lambda V$ for the free graded-commutative algebra generated by $V$. It is equivalent to the tensor product
\[\textrm{Sym}(V^\textrm{even})\otimes\textrm{Etr}(V^\textrm{odd})\]
of symmetric algebra on vectors of even degrees and exterior algebra on odd.

\begin{con}
Let $S(n)$ be the free differential graded-commutative algebra with on one generator of degree $n$:
\[(\Lambda x,dx=0),\quad|x|=n.\] Let $D(n-1)$ be the free differential graded-commutative algebra $\Lambda(x,y)$:
\[(\Lambda(x,y),dx=0,dy=x),\quad|x|=n,\quad|y|=n-1.\]
Write $Z^n(A)$ for the space of $n$-cocycles of $A$. We make the following observations:\begin{enumerate}[(i)]
    \item $S(n)\hookrightarrow D(n-1)$ is a natural inclusion;
    \item $\Hom_{\dgcAlg_k}(S(n),A)\cong Z^n(A)$;
    \item $\Hom_{\dgcAlg_k}(D(n-1),A)\cong A^{n-1}$.
\end{enumerate}
More generally, for any graded vector space $V$, define $S(V)$ to be $\Lambda V$ with $dv=0$ for all $v\in V$. Define $D(V)$ to be $\Lambda(V\oplus sV)$ with $dv=0$ and $d(sv)=v$ for $v\in V$. Then similarly the following holds:\begin{enumerate}[(i)]
    \item $S(V)\hookrightarrow D(V)$ is a natural inclusion;
    \item $\Hom_{\dgcAlg_k}(S(V),A)\cong\Hom_k(V,Z(A))$;
    \item $\Hom_{\dgcAlg_k}(D(V),A)\cong\Hom_k(V,A)$.
\end{enumerate}
We will see later that $S(n)$ and $D(n-1)$ functions like $S^n$ and $D^{n-1}$ in inductive gluing constructions.
\end{con}

\begin{lem}
For any graded vector spaces $V$ and $V'$ and any cdg-algebra $A$, the morphism $i:A\to A\otimes_kS(V)\otimes_kD(V')$ given by $a\mapsto a\otimes\eta\otimes\eta$ is a cofibration.
\end{lem}
\begin{proof}
The morphism $j:A\to A\otimes_kD(V)$ given by $a\mapsto a\otimes\eta$ is a weak equivalence. The inclusion $S(V)\hookleftarrow D(V)$ is a cofibration.
\end{proof}

\begin{proof}[Proof of Proposition 2.15]
The proof outlined here is taken from \cite{Gelfand-Manin}.\medbreak

\textit{Limit Axiom.} We show that pushout and pullback exist in $\CDGA_k$. It's standard to verify that $\{(b,c):f(b)=g(c)\}$ is the pullback of $\smash{B\xrightarrow{f}A\xleftarrow{g}C}$. For pushout of $\smash{B\xleftarrow{f}A\xrightarrow{g}C}$, consider $B\otimes_AC$ together with inclusions $\eta\otimes\id:C\to B\otimes_AC$ and $\id\otimes\eta:B\to B\otimes_AC$ where $\eta$ is the unit.\medbreak

\textit{Factorization Axiom.} Let $f:A\to B$ be a morphism in $\CDGA_k$. We show that $\Fib\circ(\Cof\cap\W)=f=(\Fib\cap\W)\circ\Cof$.

For the second equality consider the base case
\[A\xrightarrow{j_0}C_0=A\otimes_kS(Z(B))\otimes_kD(B)\xrightarrow{q_0}B\]
where $j_0=\id\otimes\eta\otimes\eta$ and $q_o:a\otimes z\otimes b\mapsto f(a)\alpha(z)\beta(b)$ and $\alpha$ corresponds to $\id:Z(B)\to Z(B)$ by previous discussion. Then $j_0\in\Cof$ by Lemma 2.16 and $q_0\in\Fib$ by definition. But instead of $q_0\in\W$, we can only guarantee that $q_0$ is surjective on cohomology.

To fix this issue, we add generators to $C_n$ to kill cocycles in $C_n$. Suppose $\smash{A\xrightarrow{j_n}C_n\xrightarrow{q_n}B}$ has already been constructed. Let
\[V_n=\{(c,b)\in C_n\oplus B:dc=0\textrm{ and }q_n(c)=db\}\textrm{ (with grading }\abs{(c,b)}=\abs{c})\]
contain the information we want to erase. We add generators to $C_n$ to construct $C_{n+1}$ in analogy to adding generators to $S(V_n)$ to construct $D(V_n)$. Formally, we construct $C_{n+1}$ as the pushout
\[\begin{tikzcd}[column sep=large,row sep=large]
S(V_n)\arrow[d,hookrightarrow]\arrow{r}{\alpha'}&C_n\arrow{d}{i_n}\arrow[ddr,bend left,"q_n"]&A\arrow[l,"j_n"']\\
D(V_n)\arrow[drr,bend right,"\beta'"]\arrow[r]&C_{n+1}\arrow[ul,phantom,"\ulcorner",very near start]\arrow[dr,dashed,"q_{n+1}"']\\
&&B
\end{tikzcd}\]
where $\alpha'$ is the composition $S(V_n)\to S(V_n')\to C_n$ and $V_n'=\im(V_n\to C_n)$. Let $j_{n+1}=i_nj_n:A\to C_{n+1}$. The map $q_{n+1}:C_{n+1}\to B$ comes from the universality of pushout. The map $\beta'$ is the composition $D(V_n)\to D(V_n'')\to B$ where $V_n''=\im(V_n\to B)$. We have thus constructed $\smash{A\xrightarrow{j_{n+1}}C_{n+1}\xrightarrow{q_{n+1}}B}$.

Finally let $C=\lim_n C_n$ and $j=\filtlim j_n:A\to C$. By Lemma 2.16 $\lambda\in\Cof$, and by xxx $i_n\in\Cof$. Hence $j_n\in\Cof$ for all $n$. Similarly let $q=\inverselim q_n:C\to B$, then $q$ is surjective on cohomology. By our construction $q$ is also injective on cohomology: any kernel in $H^\cdot(q_n)$ is becomes a boundary in $C_{n+1}$, and hence is trivial in $C$.
\medbreak

\textit{Determination Axiom.} We forced $\Cof=(\Fib\cap\W)^r$ in the construction. It remains to show that $\smash{\Fib=(\Cof\cap\W)^l}$ and $\smash{\W=\Fib^r\circ\Cof^l}$. Let $f:A\to B$ and factorize $f$ to $\smash{A\xrightarrow{i}C\xrightarrow{p}B}$ where $i\in\Cof\cap\W$ and $p\in\Fib$. If $f\in(\Cof\cap\W)^r$, then there exists a lift $h:C\to A$ such that $p=fh$. Since $p$ is surjective, $f$ is surjective and $f\in\Fib$. Hence $(\Cof\cap\W)^r\subseteq\Fib$.

For the reverse inclusion, factorize $j:A\to B$ through $\smash{A\xrightarrow{i}A\otimes_kD(B)\xrightarrow{p}B}$. If $j\in\Cof\cap\W$, then for any surjection $f:C\to D$ we construct a lift $h:B\to C$ as follows:
\[\begin{tikzcd}[column sep=huge, row sep=scriptsize]
A\arrow[d,"i"']\arrow[r,"h"]&C\arrow[dd,"f"]\\
A\otimes_kD(B)\arrow[ur,"k", near start]\arrow[d,"p"]\\
B\arrow[uur,"\ell"']\arrow[u,bend left,"q"]\arrow[uur,dashed,bend right,"h"']\arrow[r,"g"']&D
\end{tikzcd}\]
Since $f$ is surjective, there exists a degreewise map $\ell:B\to C$ such that $g=f\ell$. Then $k:A\otimes_kD(B)\to C$ is determined by restricting it to $h$ and $\ell$ on $A$ and $B$ respectively. The map $q:B\to A\otimes_kD(B)$ is constructed as the lift in the diagram
\[\begin{tikzcd}
A\arrow[d,"j"]\arrow[r,"i"]&A\otimes_kD(B)\arrow[d,"p"]\\
B\arrow[ur,dashed]\arrow[r,equal]&B
\end{tikzcd}\]
We define $h=kq$. It's standard to check that $h$ is indeed the lift required. Hence $f\in(\Cof\cap\W)^r$ and $\Fib\subseteq(\Cof\cap\W)^r$.

The inclusion $\smash{\W\subseteq\Fib^r\circ\Cof^l}$ is immediate. For the converse, we need to prove that $\smash{\Cof^r\subseteq\W}$ and $\smash{\Fib^l\subseteq\W}$. Factorize $f:A\to B$ through $\smash{A\xrightarrow{i}C\xrightarrow{p}B}$ where $i\in\Cof$ and $p\in\Fib\cap\W$. If $f\in\Cof^r$, then it follows that $f\in\W$. The other inclusion is proven similarly.\medbreak

\textit{2-out-of-3 Axiom.} All isomorphisms of cdg-algebras are quasi-isomorphisms. The 2-out-of-3 property holds for quasi-isomorphisms.
\end{proof}

We want to be able to describe the cofibrations in $\CDGA_k$ explicitly, instead of defining it
to satisfy the axioms of model category. These cofibrations are so-called relative Sullivan algebras.

\begin{defns}
A \textbf{relative Sullivan algebra} is an inclusion $A\hookrightarrow A\otimes_k\Lambda V$ of cdga$_k$ where $V=\{V^i\}_{i\geq1}$ is a DGM$_k$ with filtration
\[0=V(-1)\subseteq V(0)\subseteq V(1)\subseteq\cdots\]
satisfying $V=\cup_{j\geq0}V(j)$ and $dV(n)\subseteq A\otimes_k\Lambda V(n-1)$ for all $n\geq0$.

A map $f:A\otimes_k\Lambda V\to A'\otimes_k\Lambda V'$ is a morphism of relative Sullivan algebras if it's restriction to $A$ and $A'$ is a morphism of cdga$_k$s:
\[\begin{tikzcd}
A\arrow[d,hookrightarrow]\arrow[r,"f|_A"]&A'\arrow[d,hookrightarrow]\\
A\otimes_k\Lambda V\arrow[r,"f"]&A'\otimes_k\Lambda'
\end{tikzcd}.\]

A \textbf{Sullivan algebra} is a relative Sullivan algebra with $A=k$. A relative Sullivan algebra is \textbf{minimal} if $\im d\subseteq\Lambda^{\geq2}V$, subspace of words of length at least two.
\end{defns}

Relative Sullivan algebras can also be built inductively similar to CW complexes by gluing disks algebras along sphere algebras. In the proof of factorization axiom in Proposition 2.15 this equivalent to adding generators to kill cocycles: if $A$ is an differential graded module over $k$ with $x\in Z^n(A)$, then $A\otimes_k\Lambda V$ with $V=\{y\}$ and $dy=x$ is the pushout of $D(V)\leftarrow S(V)\rightarrow A$.

A relative Sullivan algebra is an inclusion $A\hookrightarrow X$ of differential graded algebras over $k$ with a filtration
\[A=X(0)\subseteq X(1)\subseteq\cdots\]
satisfying $X=\cup_{j\geq0}X(j)$, where each $X(j)$ is obtained as the pushout of $D(V^j)\leftarrow S(V^j)\rightarrow X(j-1)$ for some differential graded module $V=\{V^i\}_{i\geq1}$.

\begin{prop}
Relative Sullivan algebras are cofibrations and Sullivan algebras are cofibrant in the $q$-model structure of $\CDGA_k$.
\end{prop}

\begin{defn}
Let $f\in\Hom_{\CDGA_k}(A,B)$. A \textbf{Sullivan model} for $f$ is a relative Sullivan algebra $A\hookrightarrow M$ together with a quasi-isomorphism $M\sim B$.
\[\begin{tikzcd}
A\arrow[dr,hookrightarrow]\arrow[rr,"f"]&&B\\
&M\arrow[ur,"\sim"']
\end{tikzcd}\]
A Sullivan model is \textbf{minimal} if $A$
\end{defn}

Minimal Sullivan model plays.

\begin{prop}
Every $f\in\Hom_{\CDGA_k,1}(A,B)$ admits a minimal Sullivan model which is unique up to isomorphism.
\[\begin{tikzcd}[row sep=large]
	A && B \\
	M && {M'}
	\arrow["f","\cong"', from=1-1, to=1-3]
	\arrow["\sim" near start, from=2-1, to=1-3]
	\arrow["\sim"' near start, from=2-3, to=1-1]
	\arrow["\cong"',dashed, from=2-1, to=2-3]
	\arrow[from=1-1,hookrightarrow, to=2-1]
	\arrow[from=1-3,hookrightarrow, to=2-3]
\end{tikzcd}\]
\end{prop}

\newpage
\subsection{The second Quillen equivalence}

In this section we prove the equivalence
\[\Ho(\sSet_{\Q,1,f})\cong\Ho(\dgcAlg_{\Q,1,f})\]
of homotopy theories. We begin by constructing a functor from $\sSet$ to $\dgcAlg_k$. For the whole section $k$ will be a field with characteristic $0$.

\begin{defn}
The \textbf{polynomial differential form} $\Omega^\bullet_\textrm{poly}$ is a simplicial object in $\dgcAlg_k$ (that is, a functor $\mathbf{\Delta}^\textrm{op}\to\dgcAlg_k$) given by

\[\Omega^\bullet[n]=\Omega^n_\textrm{poly}=\Lambda(t_0,\cdots,t_n,dt_0,\cdots,dt_n)/\left(\sum_{i=0}^nt_i=1,\sum_{i=0}^ndt_i=0\right)\]
where $|t_i|=0$ and $|dt_i|=1$.
\end{defn}

The superscript $^\bullet$ encodes the simplicial information specified by face maps $\smash{\delta_i:\Omega^n_\textrm{poly}\to\Omega^{n-1}_\textrm{poly}}$ and degeneracy maps $\smash{\sigma_i:\Omega^n_\textrm{poly}\to\Omega^{n+1}_\textrm{poly}}$ below:

\[\delta_i:t_k\mapsto\begin{cases}
t_k,&k<i\\0,&k=i\\t_{k-1},&k>i
\end{cases},\quad\quad s_i:t_k\mapsto\begin{cases}
t_k,&k<i\\t_k+t_{k+1},&k=i\\t_{k+1},&k>i
\end{cases}.\]

In addition to the simplicial structure, $\Omega^\bullet_\textrm{poly}$ also carries a grading structure as an algebra. We encode this structure in the subscript and write $\Omega^\bullet_{\textrm{poly},p}$ for the subspace of homogeneous elements of $\Omega^\bullet_\textrm{poly}$ of degree $p$. This subspace is a $k$-module.

\begin{lem}
For any $p\geq0$, $\Omega^\bullet_{\mathrm{poly},p}$ is a trivial Kan complex.
\end{lem}
\begin{proof}
For any $p\geq0$, $\Omega^\bullet_{\textrm{poly},p}$ is a simplicial $k$-module, and thus a Kan complex. By a theorem of Eilenberg-Mac Lane, the homotopy groups of $\Omega^\bullet_{\textrm{poly},p}$ corresponds to the homology groups of the alternating face map complex

\[\begin{tikzcd}
\cdots\arrow[r]&\Omega^2_p\arrow[r,"\partial_2"]&\Omega^1_p\arrow[r,"\partial_1"]&\Omega^0_p\end{tikzcd}\]
with differential given by $\partial_n=\sum_{i=1}^n(-1)^i\delta_i$. Define shifting chain maps $\{s_n:\Omega^n_{\textrm{poly},p}\to\Omega^{n+1}_{\textrm{poly},p}\}_{n}$ given by $s_n(1)=(1-t_0)^2$ and $s_n(t_i)=(1-t_0)t_{i+1}$ for $0\leq i\leq n$. It's standard to verify that $s_n$ is a degeneracy map, Goerss-Jardin 201. 
\end{proof}

Now we can extend $\Omega^\bullet_\textrm{poly}:\mathbf{\Delta}^\textrm{op}\to\dgcAlg_k$ along $\mathbf{\Delta}^\textrm{op}\to\sSet$ via the left Kan extension. This establishes a functor $\sSet\to\dgcAlg_k$:

\[\begin{tikzcd}
\mathbf{\Delta}^\textrm{op}\arrow[swap]{dr}{\Delta}\arrow{rr}{\Omega^\bullet_\textrm{poly}}&&\dgcAlg_k\\
&\sSet\arrow[swap,near start,dashed]{ur}
\end{tikzcd}.\]

\begin{defn}
For any simplicial set $S$, the \textbf{PL de Rham complex} $\Omega^\bullet$ is a hom-functor $\sSet\to\dgcAlg_k^\mathrm{op}$ given by
\[\Omega^\bullet(S)=\Hom_\sSet(S,\Omega^\bullet_\textrm{poly}).\]
Alternatively, we can define $\Omega^\bullet(S)=\lim_{\Delta[n]\to S}\Omega^n_\textrm{poly}$. See xxx for the detail.
\end{defn}

The integration $\int$ maps smooth differential $p$-forms to $p$-cochains. By Stokes' theorem, $\int$ is a chain map between the de Rham complex and the singular cochain complex. The classical de Rham's theorem asserts that $\int$ induces isomorphism on cohomology. The analogy for polynomial forms is explain below.

\begin{con}
Let $S$ be a simplicial set and $x\in S$ a simplex. Let $f\in\Omega^p(S)$ be a polynomial $p$-form on $S$. Define an integration map $\oint_p:\Omega^p(S)\to C^p(S;k)$ by

\[(\oint_pf)(x)=\int_{\Delta[p]}f(x)\]
where $f(x)=\hat{f}(x)dt_1\cdots dt_p$ and $\hat{f}(x)\in\Q[t_1,\cdots,t_p]$. The collection $\oint=\{\oint_p\}_{p\geq0}$ defines a chain map from $\Omega^\bullet(S)$ to $C^\bullet(S;k)$ by the Stokes' theorem.
\end{con}

\begin{prop}[PL de Rham's theorem]
For a simplicial set $S$, the integration map $\oint:\Omega^\bullet(S)\to C^\bullet(S;k)$ is a quasi-isomorphism.
\end{prop}
\begin{proof}
Since any simplicial set $S$ has a skeletal filtration
\[\emptyset=S^{(-1)}\subseteq S^{(0)}\subseteq\cdots\subseteq S=\textstyle{\colim_nS^{(n)}}\]
where $S^{(n)}$ is generated by non-degenerate simplices of $S$ of degree at most $n$, we will use skeleta induction for the proof.\medbreak

\textit{First step.} The goal is to prove the theorem for $S=\Delta[n]$ for any $n$. The limit construction gives $\smash{\Omega^\bullet_\textrm{PLdR}(\Delta[n])=\Omega^n_\textrm{poly}}$, and the natural identification
\begin{align*}
\Lambda(t_0,\cdots,t_n,dt_0,\cdots,dt_n)/\left(\sum_{i=0}^nt_i=1,\sum_{i=0}^ndt_i=0\right)=\Lambda(t_1,\cdots,t_n,dt_1,\cdots,dt_n)
\end{align*}
identifies $\Omega^n_\textrm{poly}$ with the tensor product of $n$ copies of $\Lambda(t,dt)=\Omega^1_\textrm{poly}$. Since $\Omega^1_\textrm{poly}$ has trivial cohomology, so does $\smash{\Omega^n_\textrm{poly}}$ by the Künneth formula. The chain map $\smash{\oint}_n$ takes the trivial class in $\smash{\Omega^n_\textrm{poly}}$ to the trivial class in $\smash{C^n(\Delta[n],k)}$, thus inducing isomorphism on cohomology.\medbreak

\textit{Second step.}
Any simplicial set $S$ has a skeletal filtration
\[\emptyset=S^{(-1)}\subseteq S^{(0)}\subseteq\cdots\subseteq S=\textstyle{\colim_nS^{(n)}}\]
where $S^{(n)}$ is generated by non-degenerate simplices of $S$ of degree at most $n$. The base case $\Omega^\bullet_\textrm{PLdR}(S^{(0)})\simeq C^\bullet(S^{(0)};k)$ is trivial. Now suppose the claim is true for $S^{(n-1)}$, then $S^{(n)}$ is obtained by gluing copies of $\Delta[n]$, which appears as a pushout of the form $\coprod\Delta[n]\leftarrow\coprod\partial\Delta[n]\rightarrow S$. Consider the cube diagram:
\[\begin{tikzcd}[column sep=scriptsize]
\Omega^\bullet_\textrm{PLdR}(S^{(n)})\arrow[dd]\arrow[dr]\arrow[rr]&&\Omega^\bullet_\textrm{PLdR}(\Delta[n])\arrow[dd,dashed,two heads]\arrow[dr,"\simeq"]\\
&C^\bullet(S^{(n)};k)\arrow[dd]\arrow[rr]&&C^\bullet(\Delta[n];k)\arrow[dd,two heads]\\
\Omega^\bullet_\textrm{PLdR}(S^{(n-1)})\arrow[dr,"\simeq"']\arrow[rr,dashed]&&\Omega^\bullet_\textrm{PLdR}(\partial\Delta[n])\arrow[dr,dashed,"\simeq"]\\
&C^\bullet(S^{(n-1)};k)\arrow[rr]&&C^\bullet(\partial\Delta[n];k)
\end{tikzcd}.\]
The front and back squares are pullbacks, since both $\Omega^\bullet_\textrm{PLdR}$ and $C^\bullet$ are contravariant functors that preserve (co)limits. The contravariance also sends inclusions (which are cofibrations in $\sSet$) to surjections (which are fibrations in $\dgcAlg_k$). The front and back squares are connected by $\oint$. The bottom horizontal quasi-isomorphisms are obtained from inductive assumption (noticing that $\partial\Delta[n]$ is an $(n-1)$-skeleton. The upper-right quasi-isomorphism follows from the First step. The cube lemma then completes the induction $\Omega^\bullet_\textrm{PLdR}(S^{(n)})\simeq C^\bullet(S^{(n)};k)$.
\end{proof}

We have hitherto constructed a functor $\Omega^\bullet:\sSet\to\dgcAlg^\textrm{op}_k$. We now define its right adjoint functor. This can also be done by Kan extension.

\begin{con}
For $A$ a differential graded-commutative algebra over $k$, define $\mathcal{K}_\bullet$ to be the hom-functor $\dgcAlg_k^\textrm{op}\to\sSet$ given by
\[\mathcal{K}_\bullet(A)=\Hom_{\dgcAlg_k}(A,\Omega^\bullet_\textrm{poly}).\]
Then $\mathcal{K}_\bullet$ is right adjoint to $\Omega^\bullet$, as shown by the following calculation:
\begin{align*}
\dgcAlg_k(A,\Omega^\bullet(S))&\cong\dgcAlg_k(A,\textstyle{\lim_{\Delta[n]\to S}\Omega_\textrm{poly}^\bullet}) \\
&\cong\colim_{\Delta[n]\to S}\mathcal{K}_n(A)\\
&\cong\colim_{\Delta[n]\to S}\sSet(\mathcal{K}(A),\Delta[n])\quad\textrm{(Yoneda lemma)}\\
&\cong\sSet(\mathcal{K}(A),S).
\end{align*}
\end{con}

\begin{prop}
The adjunction $\Omega^\bullet\dashv\mathcal{K}_\bullet$ is a Quillen adjunction.
\end{prop}
\begin{proof}
We show that $\Omega^\bullet$ preserves cofibrations and acyclic cofibrations. We only need to check this for generating sets $I$ and $J$.\medbreak

\textit{Preserving cofibrations.} We need to show that the inclusions $\partial\Delta[n]\hookrightarrow\Delta[n]$ are sent to degreewise surjections in $\dgcAlg_k$. Moerdijk 29.\medbreak

\textit{Preserving acyclic cofibrations.} f
\end{proof}

In PL de Rham's theorem we see how $\Omega^\bullet$ relates to ordinary cohomology. In the following lemma we investigate the homotopy groups of $\mathcal{K}_\bullet$. We omit the proof.

\begin{lem}
Let $A=\Lambda V$ be a cofibrant object in $\dgcAlg_{k,1}$. Then there is a group isomorphism $\pi_n\mathcal{K}_\bullet(A)\cong\Hom_k(V^n,k)$.
\end{lem}

\begin{cor}
If $V$ is generated by one element of degree $n$ and $A=\Lambda V$ is a minimal algebra, then $\mathcal{K}_\bullet(A)$ has the homotopy type of an Eilenberg-MacLane space $K(V^\ast,n)$, where $V^\ast$ is the dual space of $V$.
\end{cor}

Recall that for a Quillen functor, its left and right derived functor are obtained by restricting to cofibrant and fibrant objects respectively. The following construction allows to compare the homotopy theories of $\sSet$ and $\dgcAlg_k$.

\begin{con}
Since every object $S$ of $\sSet$ is cofibrant, the left derived functor of $\Omega^\bullet$
\[\mathbf{L}\Omega^\bullet:\Ho(\sSet)\to\Ho(\dgcAlg_k^\textrm{op})\]
corresponds with $\Omega^\bullet$. That is, $\mathbf{L}\Omega^\bullet(S)=\Omega^\bullet(S^\textrm{cof})=\Omega^\bullet(S)$.

As proved in Corollary 4.12, every minimal model $M_A$ is cofibrant in $\dgcAlg_k$ (and hence fibrant in the opposite category). The right derived functor of $\mathcal{K}_\bullet$
\[\mathbf{R}\mathcal{K}_\bullet:\Ho(\dgcAlg_k^\textrm{op})\to\Ho(\sSet)\]
is obtained by $\mathbf{R}\mathcal{K}_\bullet(A)=\mathcal{K}_\bullet(A^\textrm{fib})=\mathcal{K}_\bullet(M_A)$.
\end{con}

Finally, we wish to show that the adjunction $\Omega^\bullet\dashv\mathcal{K}_\bullet$ is a Quillen equivalence, so that the derived functors induces an equivalence on the homotopy categories. We show that by proving that the adjunction counit and unit are weak equivalences.

\begin{prop}
The derived adjunction counit $\Bar{\epsilon}:\mathbf{L}\Omega^\bullet(\mathcal{K}_\bullet(A))\to A$ is a weak equivalence in $\dgcAlg_{\Q,1,f}$.  
\end{prop}
\begin{proof}

The derived adjunction counit is the composition of the normal adjunction counit $\epsilon$ with a cofibrant replacement:
\[\Bar{\epsilon}:\mathbf{L}\Omega^\bullet(\mathcal{K}_\bullet(A))=\Omega^\bullet(\mathcal{K}_\bullet(A)^\textrm{cof})\longrightarrow\Omega^\bullet(\mathcal{K}_\bullet(A))\xrightarrow{\ \epsilon\ }A.\]
The cofibrant replacement is trivial since every object is cofibrant in $\sSet$. The proof is by degreewise induction, bearing similarities to that of Proposition 2.12.\medbreak

\textit{Base case.} Let $A=\Lambda V$ be a minimal algebra with $V$ generated by one element $v$ of degree $n$. By Corollary 2.16 we have $\mathcal{K}_\bullet(A)\simeq K(\Q,n)$, and the cohomology of which is $\Q[x]$, the free graded-commutative algebra on one generator $x$. This can be calculated via spectral sequences, first replacing $K(\Q,n)$ with $K(\Z,n)$ since $\Z\hookrightarrow\Q$ induces rational homotopy equivalence. Let $z\in\Omega^\bullet(\mathcal{K}_\bullet(A))$ by a cycle representing $x$, and the map $\Omega^\bullet(\mathcal{K}_\bullet(A))\to A$ defined by $z\mapsto v$ is a quasi-isomorphism.\medbreak

\textit{Inductive step.} Let $A$ be a cofibrant object for which the theorem holds. If $B$ is obtained by the pushout of $\Lambda D(m-1)\leftarrow\Lambda S(m)\rightarrow A$, we prove that the theorem also holds for $B$. Apply $\Omega^\bullet\mathcal{K}_\bullet$ to the pushout square to get a cube:
\[\begin{tikzcd}[column sep=scriptsize]
\Lambda S(m)\arrow[rr]\arrow[dr,"\simeq"{description}]\arrow[dd]&& A\arrow[dd]\arrow[dr,"\simeq"{description}]\\
&\Omega^\bullet(\mathcal{K}_\bullet(\Lambda S(m)))\arrow[rr]\arrow[dd]&&\Omega^\bullet(\mathcal{K}_\bullet(A))\arrow[dd]\\
\Lambda D(m-1)\arrow[rr]\arrow[dr,"\simeq"{description}]&& B\arrow[dr]\\
&\Omega^\bullet(\mathcal{K}_\bullet(\Lambda D(m-1)))\arrow[rr]&&\Omega^\bullet(\mathcal{K}_\bullet(B))
\end{tikzcd}.\]
Since both $\Omega^\bullet$ and $\mathcal{K}_\bullet$ are contravariant, the front square is a homotopy pushout by Lemma xxx. By xxx the back pushout square is also a homotopy pushout square, since every object is cofibrant and $\Lambda S(m)\hookrightarrow\Lambda D(m-1)$ is a cofibration.

The bottom left quasi-isomorphism comes from the face that both algebras are acyclic. The top left quasi-isomorphism comes from the base step, and the top right quasi-isomorphism is by assumption. The desired statement follows by the cube lemma.\medbreak

Now we adapt the inductive step to our situation. First note that we can substitute $A$ for its minimal model $M_A$, since every object in $\dgcAlg_\Q$ admits a unique minimal model. Moreover, $M_A$ is $1$-connected and of finite type if $A$ has the same properties. We only need to prove that $\Lambda V\to\Omega^\bullet\mathcal{K}_\bullet(\Lambda V)$ is a quasi-isomorphism for a minimal model $\Lambda V$.\medbreak

By Proposition 2.33 we can filter $\Lambda V$ by degree, that is, $\Lambda V^n=\Lambda V(n)$. Suppose $V(n)=V(n-1)\oplus V'$ and that the theorem holds for $V(n-1)$, then for each generator $v\in V'$ of degree $p$, the inductive step claims that the theorem also holds for the pushout in the following diagram
\[\begin{tikzcd}
S(k)\arrow[d]\arrow{r}{t\ \mapsto\ dv}&\Lambda V(n-1)\arrow[d]\\
D(k-1)\arrow[r]&\Lambda(V(n-1)\oplus\Q\cdot v) \arrow[ul, phantom, "\ulcorner",very near start]
\end{tikzcd}.\]
Since $\Lambda V$ is of finite type, repeating this procedure for finitely many times we see that the theorem also holds for $V(n)$. Finally, the map $H^i(\Lambda V)\to H^i(\Omega^\bullet\mathcal{K}_\bullet(\Lambda V))$ is same as the map $H^i(\Lambda V^{\leq i})\to H^i(\Omega^\bullet\mathcal{K}_\bullet(\Lambda V^{\leq i}))$, which is an isomorphism.
\end{proof}

\begin{prop}
The derived adjunction unit $\Bar{\eta}:S\to\mathbf{R}\mathcal{K}_\bullet(\Omega^\bullet(S))$ is a weak equivalence in $\sSet_{\Q,1,f}$
\end{prop}
\begin{proof}
The derived adjunction unit is the composition of the normal adjunction unit $\eta$ with a fibrant replacement:
\[S\xrightarrow{\ \eta\ }\mathcal{K}_\bullet(\Omega^\bullet(S))\longrightarrow\mathcal{K}_\bullet(\Omega^\bullet(S)^\textrm{fib})=\mathbf{R}\mathcal{K}_\bullet(\Omega^\bullet(S))=\mathcal{K}_\bullet(M_{\Omega^\bullet(S)}).\]
In prop 2.14 we've shown that $\Omega^\bullet$ preserves weak equivalences. Hence $\Bar{\eta}$ is a weak equivalence if and only if
\[\Omega^\bullet(\Bar{\eta}):\Omega^\bullet(S)\to\Omega^\bullet(\mathcal{K}_\bullet(M_{\Omega^\bullet(S)}))\]
is a quasi-isomorphism. Since $M_{\Omega^\bullet(S)}$ is a minimal algebra, by Proposition 2.17 $\Omega^\bullet(\mathcal{K}_\bullet(M_{\Omega^\bullet(S)}))\simeq M_{\Omega^\bullet(S)}$. Finally, $M_{\Omega^\bullet(S)}\simeq\Omega^\bullet(S)$ by definition of minimal model. Hence $\Omega^\bullet(\Bar{\eta})$ is a quasi-isomorphism.
\end{proof}

\begin{cor}
efe
\end{cor}

\begin{remark}
We only restrict
\end{remark}

%\section{Rational Homotopy Theory: A Higher Perspective}

\section{Realizing polynomial cohomology rings}

\section*{Acknowledgments}

It is a pleasure to thank my mentor, Danny Xiaolin Shi, for blah. Big thanks to Peter for not only organizing this event but also recommending me I also thank blah for helping me understand blah.

\begin{thebibliography}{99}

\bibitem{Andersen-Grodal}
Andersen, K. K., \& Grodal, J. (2008). \textit{The Steenrod problem of realizing polynomial cohomology rings}. Journal of Topology, 1(4), 747–760. https://doi.org/10.1112/jtopol/jtn021 

\bibitem{May} 
May, J. P., \& Ponto, K.. \textit{More Concise Algebraic Topology}. 2011.

\end{thebibliography}

\end{document}

